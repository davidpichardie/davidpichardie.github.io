\documentclass[12pt,a4paper]{article}
\usepackage{fancyheadings}
\usepackage{pstricks,pst-node,pst-tree}
\usepackage{pstcol}
\usepackage{amsmath,theorem}
\usepackage[ansinew]{inputenc}
\usepackage[frenchb]{babel}
%\usepackage[francais]{layout}


\title{TP4 {\sc Caml} : Formules bool�ennes}
\author{}
\date{}

%MISE EN PAGE------------------------------------------------------
%\setlength{\hoffset}{-1.90cm}%2.54
%\setlength{\voffset}{-1.7cm} 
\setlength{\oddsidemargin}{0cm}
\addtolength{\textwidth}{70pt}
\setlength{\topmargin}{0cm}
% \setlength{\headsep}{0cm}
% \setlength{\headheight}{0cm}
\addtolength{\textheight}{3cm}
% \setlength{\columnseprule}{1pt}
%\addtolength{\parskip}{-0.1cm}

\theoremstyle{break}
\newtheorem{definition}{D�finition}
\newtheorem{proposition}{Proposition}
 
\pagestyle{fancy}
\lhead{Corrig� TP4 {\sc Caml}}
\rhead{Formules bool�ennes}

\newcounter{numquestion}
\setcounter{numquestion}{1}
\newenvironment{question}{\noindent{\bf \thenumquestion.}}%
{\stepcounter{numquestion}\medskip}

\setlength{\parindent}{0cm}

\renewcommand{\O}{\mathcal{O}}

\begin{document}

\section{Repr�sentation des formules bool�ennes}

\begin{verbatim}
type binop = Et | Ou | Oubien | Impl | Equiv;;

type formule =
  | Vrai 
  | Faux
  | Var  of string
  | Non  of formule
  | Bin  of binop * formule * formule
;;

let rec affiche par f = match f with
  | Vrai   -> print_string "Vrai"
  | Faux   -> print_string "Faux"
  | Var(x) -> print_string x
  | Non(g) -> print_string "Non("; affiche false g; print_string ")"
  | Bin(op,g,h) ->
      if par then print_string "(";
      affiche true g;
      print_string(match op with
                     | Et     -> " . "
                     | Ou     -> " + "
                     | Oubien -> " ++ "
                     |Impl    -> " => " 
                     |Equiv   -> " <=> ");
      affiche true h;
      if par then print_string ")"
;;
let print_formule f = affiche false f; print_newline();;
\end{verbatim}

\section{V�rificateur de tautologie}

\begin{question}
\begin{verbatim}
let rec evalue f = match f with
  | Vrai   -> true
  | Faux   -> false
  | Var x  -> failwith "la formule contient une inconnu"
  | Non(g) -> not(evalue g)
  | Bin(op,g,h) ->
      let a = evalue g and b = evalue h in
      match op with
        | Et     -> a && b
        | Ou     -> a || b
        | Oubien -> a <> b
        | Impl   -> (not a) or b
        | Equiv  -> a = b
;;
\end{verbatim}
\end{question}

\begin{question}
\begin{verbatim}
let rec subs x y f = match f with
  | Var(v) -> if v = x then y else f
  | Non(g) -> Non(subs x y g)
  | Bin(op,g,h) -> Bin(op, subs x y g, subs x y h)
  | _ -> f
;;
\end{verbatim}
\end{question}

\begin{question}
\begin{verbatim}
let rec tautologie f vars = match vars with
  | [] -> evalue(f)
  | x::suite -> tautologie (subs x Vrai f) suite
              & tautologie (subs x Faux f) suite
;;
\end{verbatim}
\end{question}

\begin{question}

Il faut bien prendre garde � d'abord simplifier les op�randes d'une
expression avant de simplifier l'expression elle-m�me, sans quoi la
simplification effectu�e est tr�s limit�e.

\begin{verbatim}
let rec simplifie f = match f with
  | Non(g) -> begin
      match (simplifie g) with
       | Vrai   -> Faux
       | Faux   -> Vrai
       | Non(h) -> h
       | h      -> Non h
      end
  | Bin(op,g,h) -> begin match op,simplifie(g),simplifie(h) with
      | Et, Vrai, b     -> b
      | Et, Faux, b     -> Faux
      | Et, a, Vrai     -> a
      | Et, a, Faux     -> Faux
    
      | Ou, Vrai, b     -> Vrai
      | Ou, Faux, b     -> b
      | Ou, a, Vrai     -> Vrai
      | Ou, a, Faux     -> a

      | Oubien, Faux, b -> b
      | Oubien, Vrai, b -> Non b
      | Oubien, a, Faux -> a
      | Oubien, a, Vrai -> Non a

      | Impl, Faux, b   -> Vrai
      | Impl, Vrai, b   -> b
      | Impl, a, Faux   -> Non a
      | Impl, a, Vrai   -> Vrai
      
      | Equiv, Vrai, b  -> b
      | Equiv, Faux, b  -> Non b
      | Equiv, a, Vrai  -> a
      | Equiv, a, Faux  -> Non a

      | _,g,h -> Bin(op,g,h)
    end
  | _ -> f
;;
\end{verbatim}
\end{question}

\begin{question}
\begin{verbatim}
let rec tautologie2 f vars = match vars with
  | [] -> evalue(f)
  | x::suite -> tautologie (simplifie (subs x Vrai f)) suite
              & tautologie (simplifie (subs x Faux f)) suite
;;
\end{verbatim}

En y r�flechissant bien, cette nouvelle fonction n'est pas beaucoup plus
efficace que la pr�c�dente car elle ne tire pas partie de l'�ventuelle
disparition d'une variable bool�enne dans une formule simplifi�e. Ce
probl�me sera r�gl� avec la question $7$.
\end{question}

\begin{question}
\begin{verbatim}
let rec tautologie3 f vars = match vars with
  | [] -> (evalue(f),[])
  | x::suite ->
      match tautologie3 (subs x Vrai f) suite with
        | (false,l) -> (false,(x^"=Vrai")::l)
        | (true,_) ->
        match tautologie3 (subs x Faux f) suite with
          | (false,l) -> (false,(x^"=Faux")::l)
          | (true,_)  -> (true,[])
;;
\end{verbatim}
\end{question}

\begin{question}

Cette question est assez difficile. Elle demande de programmer
plusieurs fonctions auxiliaires.

La premi�re permet de {\it fusionner} deux liste de variables, sans
g�n�rer de doublons. Elle requiert que les listes de d�part soient
tri�es pour l'ordre alphab�tique. 

\begin{verbatim}
let rec fusion = fun
  | [] l2             -> l2
  | l1 []             -> l1
  | (x1::q1) (x2::q2) -> if x1=x2 then fusion q1 (x2::q2)
                         else if x1<x2 then x1::(fusion q1 (x2::q2))
                                       else x2::(fusion (x1::q1) q2)
;;

#fusion ["a";"b";"c"] ["b";"d";"e";"f"];;
- : string list = ["a"; "b"; "c"; "d"; "e"; "f"]
\end{verbatim}

La deuxi�me permet de calculer la liste des variables contenues dans
une formule, en utilisant la fonction {\tt fusion}.

\begin{verbatim}
let rec liste_vars f = match f with
  | Vrai   -> []
  | Faux   -> []
  | Var x  -> [x]
  | Non(g) -> liste_vars g
  | Bin(op,g,h) -> fusion (liste_vars g) (liste_vars h)
;;

#liste_vars (Bin (Et,Bin (Ou,Var "b",Var "c"),Bin (Impl,Var "b",Var "a")));;
- : string list = ["a"; "b"; "c"]
\end{verbatim}

On en d�duit une nouvelle version de {\tt tautologie}~:

\begin{verbatim}
let rec tautologie4 f = match (liste_vars f) with
  | [] -> evalue(f)
  | x::suite -> tautologie4 (simplifie (subs x Vrai f))
              & tautologie4 (simplifie (subs x Faux f))
;;
\end{verbatim}
\end{question}

\begin{question}

Cette question demande beaucoup d'initiative. 

On suppose que la formule � simplifier ne contient plus de constante
{\tt Vrai} ou {\tt Faux}.
On commence par exprimer les op�rateurs {\tt Oubien}, {\tt Impl} et
{\tt Equiv} � l'aide de {\tt Et}, {\tt Ou} et {\tt Non}.

\begin{verbatim}
let rec enleve_Oubien_Impl_Equiv f = match f with
  | Non(g) -> Non (enleve_Oubien_Impl_Equiv g)
  | Bin(op,g,h) -> 
      begin
        let a = enleve_Oubien_Impl_Equiv g 
        and b = enleve_Oubien_Impl_Equiv h in
        match op with
          | Et     -> Bin(Et,a,b)
          | Ou     -> Bin(Ou,a,b)
          | Oubien -> Bin(Ou,Bin (Et,Non a,b),Bin (Et,a,Non b))
          | Impl   -> Bin(Ou,Non a,b)
          | Equiv  -> Bin(Et,Bin (Ou,Non a,b),Bin (Ou,a,Non b))
      end
  | f -> f
;;

#let f=(Bin (Impl,Var "a",Bin(Oubien,Var "b",Var "c")));;
#let g=(enleve_Oubien_Impl_Equiv f);;
#print_formule g;;
Non(a) + ((Non(b) . c) + (b . Non(c)))
- : unit = ()
#print_formule f;;
a => (b ++ c)
- : unit = ()
#tautologie4 (Bin (Equiv,f,g)) ;;
- : bool = true
\end{verbatim}

On peut alors faire la transformation demand�e, en se basant sur les
formules de distributivit� suivantes
$$a + (b\cdot c)\equiv (a+b)\cdot (a+c)\quad\text{et}\quad
  (b\cdot c)+a \equiv (b+a)\cdot (c+a)$$

\begin{verbatim}
let rec transforme f = match f with
  | Non(Non g) -> transforme g
  | Non(Bin (Ou,g,h)) -> Bin (Et,transforme (Non g),transforme (Non h))
  | Non(Bin (Et,g,h)) -> transforme (Bin (Ou,Non g,Non h))
  | Bin(Et,g,h) -> Bin (Et,transforme g,transforme h)
  | Bin(Ou,g,h) -> begin
       match (transforme g) with
       | Bin(Et,a1,a2) -> Bin(Et,transforme (Bin(Ou,a1,h)),
                                 transforme (Bin(Ou,a2,h)))
       | a -> begin
              match (transforme h) with
              | Bin(Et,b1,b2) -> Bin(Et,transforme (Bin(Ou,a,b1)),
                                        transforme (Bin(Ou,a,b2)))
              | b -> Bin(Ou,a,b) 
              end
       end
  | f -> f
;;

#let h=(transforme g);;
#print_formule h;;
((Non(a) + (Non(b) + b)) . (Non(a) + (Non(b) + Non(c)))) 
   . ((Non(a) + (c + b)) . (Non(a) + (c + Non(c))))
- : unit = ()
#tautologie4(Bin(Equiv,h,f));;
- : bool = true
\end{verbatim}
  
Une conjonction de formules $f_1,\ldots,f_n$ est une tautologie si et
seulement si chaque $f_i$ est une tautologie. Une disjonction de
formules atomiques du type $p$ ou $\overline{p}$, avec $p$ une
variable bool�enne, est une tautologie si et seulement si il existe
une variable bool�enne $q$ telle que, � la fois $q$ et $\overline{q}$
apparaissent dans la disjonction. Gr�ce � ces remarques on 
obtient une nouvelle m�thode pour v�rifier les tautologies. Il reste
cependant du travail � accomplir car il faut pouvoir d�tecter si une
disjonction contient une variable bool�enne et sa n�gation. Je vous
laisse ce travail en exercice...
\end{question} 

\end{document}
