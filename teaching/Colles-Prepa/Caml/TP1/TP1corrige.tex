\documentclass[12pt,a4paper]{article}
\usepackage{fancyheadings}
\usepackage[ansinew]{inputenc}
%\usepackage[francais]{layout}
\usepackage{multicol}

%MISE EN PAGE------------------------------------------------------
%\setlength{\hoffset}{-1.90cm}%2.54
%\setlength{\voffset}{-1.7cm} 
\setlength{\oddsidemargin}{0cm}
\addtolength{\textwidth}{70pt}
\setlength{\topmargin}{0cm}
% \setlength{\headsep}{0cm}
% \setlength{\headheight}{0cm}
\addtolength{\textheight}{3cm}
% \setlength{\columnseprule}{1pt}
%\addtolength{\parskip}{-0.1cm}

\pagestyle{fancy}
\lhead{Corrig� TP1 {\sc Caml}}
\rhead{Arbres}


\setlength{\parindent}{0cm}

\begin{document}

J'ai choisi d'utiliser dans ce corrig� la commande {\tt function}
plut�t que la commande {\tt fun} pour programmer les fonctions du {\sc
  tp}. Contrairement � la commande {\tt fun}, la commande {\tt
  function} n'admet qu'un param�tre, mais demande moins d'efforts de
parenth�sage. 

\section{Op�rations de base sur les arbres binaires}

On remarque que les quatre fonctions suivantes ont toutes la m�me
structure. 

\begin{verbatim}
let rec hauteur = function
   Feuille _ -> 0
 | Noeud (_,g,d) -> 1 + max (hauteur g) (hauteur d) ;;

let rec nombre_feuille = function
   Feuille _ -> 1
 | Noeud (_,g,d) -> nombre_feuille g + nombre_feuille d ;;

let rec nombre_noeud = function
   Feuille _ -> 0
 | Noeud (_,g,d) -> 1 + nombre_noeud g + nombre_noeud d ;;

let rec miroir = function
   Feuille f -> Feuille f
 | Noeud (n,g,d) -> Noeud (n,miroir d,miroir g) ;;
\end{verbatim}

\section{Arbres binaires de recherche}

\begin{verbatim}
let rec insere comp x = function
   Vide -> Noeud (x,Vide,Vide)
 | Noeud (n,g,d) -> if x=n then Noeud (n,g,d)
                    else if comp x n then Noeud (n,insere comp x g,d)
                         else Noeud (n,g,insere comp x d) ;;

let rec retire_plus_grand = function
   Vide -> failwith "arbre vide"
 | Noeud (n,g,Vide) -> (n,g)
 | Noeud (n,g,d) -> let (m,a) = retire_plus_grand d in (m,Noeud (n,g,a)) ;;
\end{verbatim}

D'un point de vue complexit�, il serait catastrophique d'�crire pour
cette derni�re ligne~:
\begin{verbatim}
 ...
 | Noeud (n,g,d) -> (fst (retire_plus_grand d),
                     Noeud (n,g,snd (retire_plus_grand d))) ;;  
\end{verbatim}

En effet, on fait alors deux appels r�cursifs au lieu d'un et la
complexit� passe d'un $\mathcal{O}(h)$, � un $\mathcal{O}(h^2)$ (o�
$h$ est la hauteur de l'arbre).
 
\begin{verbatim}
Let retire_racine = function
   Vide -> failwith "arbre vide"
 | Noeud (n,Vide,d) -> d
 | Noeud (n,g,d) -> let (m,a) = retire_plus_grand g in Noeud (m,a,d) ;;

let rec retire comp x = function
  Vide -> Vide
| Noeud (n,g,d) -> if x=n then retire_racine (Noeud (n,g,d))
                   else if comp x n then Noeud (n,retire comp x g,d)
                        else Noeud (n,g,retire comp x d) ;;

let rec separe comp x = function
  Vide -> (Vide,Vide)
| Noeud (n,g,d) -> if x=n then (g,d)
                   else if comp x n then 
                        let (a,b)=separe comp x g in (a,Noeud (n,b,d))
                   else let (a,b)=separe comp x d in (Noeud (n,g,a),b) ;;

let insere_racine comp x arbre =
  let (g,d)=separe comp x arbre in Noeud (x,g,d) ;;

let rec test comp a b = function
  Vide -> true
| Noeud (n,g,d) -> comp a n & comp n b &
                   test comp a n g & test comp n b d;; 

let test_int = test (prefix <) min_int max_int;;
\end{verbatim}

On utilise ici les valeurs {\tt min\_int} et {\tt max\_int} pr�d�finies
en {\sc Caml} pour donner l'intervalle de d�finition des entiers. La
fonction {\tt prefix} permet quant � elle d'obtenir la version pr�fixe
de n'importe quel op�rateur infixe de {\sc Caml}.
\section{Repr�sentation avec des pointeurs}

Les deux fonctions demand�es sont en fait des proc�dures : elles ne
renvoient pas de r�sultat (plus pr�cisement elles renvoient {\tt ()} de
type {\tt unit}). Elles agissent sur l'arbre qu'on leurs donne en
param�tre. 

\begin{verbatim}
let rec miroir = function
  Vide_p -> ()                             (* rien � faire *)
| Noeud n ->  begin 
                miroir n.gauche;       (* on modifie le fils gauche *)
                miroir n.droit;        (* on modifie le fils droit *)
                let temp=n.gauche in
                n.gauche <- n.droit;   (* on permute les deux fils *)
                n.droit <- temp
              end ;;
\end{verbatim}

Cette fonction d'insertion se limite � un arbre non vide car on ne
peut pas modifier directement l'arbre qui est pass� en param�tre : on
ne peut modifier que ses champs mutables. Il faut donc prendre garde �
ne pas l'appeler r�cursivement sur un fils vide du n\oe ud courant. Si
le fils convoit� est vide on remplace la valeur {\tt Vide\_p} du
champs mutable correspondant par le n\oe ud {\tt Noeud \{ val=x;
  gauche=Vide\_p; droit=Vide\_p \}}. Cette modification n'aurait pas
�t� possible si les champs {\tt gauche} et {\tt droit} n'�tait pas
mutables. 

\begin{verbatim}
let rec insere comp x = function
   Vide_p -> failwith "arbre_vide"
 | Noeud n -> 
     if n.val=x then ()
     else if comp x n.val then
        match n.gauche with
          Vide_p -> n.gauche <- Noeud { val=x; gauche=Vide_p; droit=Vide_p }
        | Noeud ng -> insere comp x n.gauche
     else match n.droit with
           Vide_p -> n.droit <- Noeud { val=x; gauche=Vide_p; droit=Vide_p }
        | Noeud nd -> insere comp x n.droit ;;

\end{verbatim}

\end{document}



