\documentclass[12pt,a4paper]{article}
\usepackage{fancyheadings}
\usepackage{pstricks,pst-node,pst-tree}
\usepackage{pstcol}
\usepackage{amsmath}
\usepackage[ansinew]{inputenc}
\usepackage[frenchb]{babel}
%\usepackage[francais]{layout}


\title{TP2 {\sc Caml} : Arbres rouges et noirs}
\author{}
\date{}

%MISE EN PAGE------------------------------------------------------
%\setlength{\hoffset}{-1.90cm}%2.54
%\setlength{\voffset}{-1.7cm} 
\setlength{\oddsidemargin}{0cm}
\addtolength{\textwidth}{70pt}
\setlength{\topmargin}{0cm}
% \setlength{\headsep}{0cm}
% \setlength{\headheight}{0cm}
\addtolength{\textheight}{3cm}
% \setlength{\columnseprule}{1pt}
%\addtolength{\parskip}{-0.1cm}

\pagestyle{fancy}
\lhead{Corrig� TP2 {\sc Caml}}
\rhead{Arbres blancs et noirs}

\newcounter{numquestion}
\setcounter{numquestion}{1}
\newenvironment{question}{\noindent{\bf \thenumquestion.}}%
{\stepcounter{numquestion}\medskip}

\newcommand{\noir}[1]{%
  \Tcircle*[fillstyle=solid,fillcolor=black]{\textcolor{white}{\tt #1}}}
\newcommand{\blanc}[1]{%
  \Tcircle{\tt #1}}
\newcommand{\fin}[1]{%
  \pstree{#1}{\noir{}\noir{}}}
\newcommand{\T}[1]{%
{\Tcircle[linecolor=white]{\small #1}}}

\setlength{\parindent}{0cm}

\renewcommand{\O}{\mathcal{O}}

\begin{document}
%\layout
%\maketitle
 
\begin{question}
\begin{verbatim}
let pere = function
  Vide -> failwith "pere inconnu"
 |Noeud n -> n.pere ;;

let fils_gauche = function
  Vide -> failwith "fils inconnu"
 |Noeud n -> n.gauche ;;

let fils_droit = function
  Vide -> failwith "fils inconnu"
 |Noeud n -> n.droit ;;

let valeur = function
  Vide -> failwith "arbre vide"
 |Noeud n -> n.val ;;

let couleur = function
  Vide -> Noir
 |Noeud n -> n.couleur ;;
\end{verbatim}
\end{question}

\begin{question}
\begin{verbatim}
let est_fils_gauche a = fils_gauche (pere a)=a ;;

let est_fils_droit a = fils_droit (pere a)=a ;;
\end{verbatim}
\end{question}

\begin{question}
\begin{verbatim}
let frere a = 
  if est_fils_gauche a then  fils_droit (pere a)
  else fils_gauche (pere a) ;;

let oncle a = (frere (pere a)) ;;
\end{verbatim}
\end{question}

\begin{question}
\begin{verbatim}
let rec transforme = function
   Vide -> V
 | Noeud n -> N (n.val,n.couleur,transforme n.gauche,transforme n.droit) ;;
\end{verbatim}
\end{question}

\begin{question}
\begin{verbatim}
let rec test_pere_aux p = function
   Vide -> true
 | Noeud n -> n.pere=p && test_pere_aux (Noeud n) n.gauche 
                       && test_pere_aux (Noeud n) n.droit;;

let test_pere = test_pere_aux Vide ;;
\end{verbatim}
\end{question}

\begin{question}
\begin{verbatim}
let rec test_prop4 = function
  Vide -> true
 |Noeud n -> match n.couleur with
      Blanc -> (couleur n.pere)=Noir && test_prop4 n.gauche 
                                     && test_prop4 n.droit
     |Noir  -> test_prop4 n.gauche && test_prop4 n.droit ;;
\end{verbatim}
\end{question}

\begin{question}
\begin{verbatim}
exception StopProp5;;

let rec nb_noirs = function
  Vide -> 1
 |Noeud n -> let g = nb_noirs n.gauche and d= nb_noirs n.droit in
             if g=d then match n.couleur with
                   Blanc -> g
                  |Noir  -> g+1
             else raise StopProp5 ;;
\end{verbatim}
Le d�clenchement d'une exception avec la commande {\tt raise} permet
d'arr�ter le calcul si � un n\oe ud donn�, la contrainte $5$ n'est pas
v�rifi�e. 
\end{question}

\begin{question}
La fonction {\tt test\_prop5} lance le calcul de {\tt nb\_noirs} et
regarde si aucune exception n'est d�clench�e. Si c'est le cas, la
contrainte est v�rifi�e : le r�sultat est {\tt true}. Dans le cas
contraire l'exception {\tt StopProp5} est d�tect�e est le r�sultat est
{\tt false}.
\begin{verbatim}
let test_prop5 a =
  try 
    nb_noirs a;
    true
  with
    StopProp5 -> false ;;
\end{verbatim}
Pour se passer des exceptions il faudrait modifier la fonction {\tt
  nb\_noirs} pour qu'elle renvoie un couple form� d'un bool�en qui
  indique si une violation de la contrainte $5$ a �t� d�tect�e et un
  entier qui renvoie la valeur calcul�e dans l'ancienne version.
\end{question}

\begin{question}
\begin{verbatim}
let testRN a =
  (couleur a)=Noir && test_prop4 a && test_prop5 a ;;
\end{verbatim}
\end{question}

\begin{question}
\begin{verbatim}
let inverse_couleur = function
  Vide -> ()
 |Noeud n -> match n.couleur with 
    Blanc -> n.couleur <- Noir
   |Noir  -> n.couleur <- Blanc ;;
\end{verbatim}
\end{question}

\begin{question}
\begin{verbatim}
let adopte_gauche = fun
   Vide Vide -> failwith "arbres vides"
 | Vide (Noeud p) -> p.gauche <- Vide
 | (Noeud f) Vide -> f.pere <- Vide
 | (Noeud f) (Noeud p) -> ( p.gauche <- Noeud f; 
                            f.pere <- Noeud p ) ;;

let adopte_droit f = fun
   Vide Vide -> failwith "arbres vides"
 | Vide (Noeud p) -> p.droit <- Vide
 | (Noeud f) Vide -> f.pere <- Vide
 | (Noeud f) (Noeud p) -> ( p.droit <- Noeud f; 
                            f.pere <- Noeud p ) ;;
\end{verbatim}
\end{question}

\begin{question}
\begin{verbatim}
let rotation_droite a =
  let b=fils_gauche a in
  let y=fils_droit b
  and p=pere a in
  if p=Vide || (est_fils_gauche a) then adopte_gauche b p
  else adopte_droit b p;
  adopte_gauche y a;
  adopte_droit a b ;;

let rotation_gauche b = 
  let a=fils_droit b in
  let y=fils_gauche a
  and p=pere b in
  if p=Vide || (est_fils_gauche b) then adopte_gauche a p
  else adopte_droit a p;
  adopte_droit y b;
  adopte_gauche b a ;;
\end{verbatim}
Ces deux fonctions sont beaucoups plus simple � �crire si on ne
se soucie plus du champ {\tt pere} des n\oe uds. C'est d'ailleurs le
choix qui avait �t� fait dans l'�preuve d'informatique $2000$ de
Central dont est inspir� ce TP.
\end{question}

\begin{question}
\begin{verbatim}
let rotation_gauche_droite a =
  let b=fils_gauche a in
  rotation_gauche b;
  rotation_droite a ;;

let rotation_droite_gauche b =
  let a=fils_droit b in
  rotation_droite a;
  rotation_gauche b ;;
\end{verbatim}
\end{question}

\begin{question}
\begin{verbatim}
exception FinInsertion ;;

let rec insere_blanc x = function
   Vide -> Noeud {val=x; couleur=Blanc; gauche=Vide; droit=Vide; pere=Vide}
 | Noeud n -> if n.val=x then raise FinInsertion
              else if x<n.val then match n.gauche with
                 Vide -> let new = Noeud {val=x; couleur=Blanc; gauche=Vide;
                                               droit=Vide; pere=Noeud n } in
                         (n.gauche <- new; new)
               | _ -> insere_blanc x n.gauche
              else match n.droit with
                 Vide -> let new = Noeud {val=x; couleur=Blanc; gauche=Vide;
                                               droit=Vide; pere=Noeud n } in
                         (n.droit <- new; new)
               | _ -> insere_blanc x n.droit
;;
\end{verbatim}
On peut l� encore se passer de l'utilisation des exceptions avec un
bool�en qui indique si la valeur {\tt x} est d�j� pr�sente dans
l'arbre.
\end{question}

\begin{question}
\begin{verbatim}
let rec mauvais_blanc x =
   let p=pere x in 
   match p with
     Vide     -> inverse_couleur x; (* x est la racine de l'arbre *)
   | Noeud np -> match np.couleur with
         Noir  -> ()
       | Blanc -> let o=frere p and g=pere p in
                  match couleur o with
                   Blanc -> ( inverse_couleur p;             (*  cas 1  *)
                              inverse_couleur o;
                              inverse_couleur g;
                              mauvais_blanc g  )
                 | Noir  -> if est_fils_gauche p then
                              if est_fils_gauche x then
                                ( rotation_droite g;         (* cas 2.1 *)
                                  inverse_couleur g;
                                  inverse_couleur p  )
                              else
                                ( rotation_gauche_droite g;  (* cas 2.2 *)
                                  inverse_couleur g;
                                  inverse_couleur x  )
                            else
                              if est_fils_gauche x then
                                ( rotation_droite_gauche g;  (* cas 2.3 *)
                                  inverse_couleur g;
                                  inverse_couleur x  )
                              else
                                ( rotation_gauche g;         (* cas 2.4 *)
                                  inverse_couleur g;
                                  inverse_couleur p ) ;;
\end{verbatim}
Les trois cas manquants sont~:
$$
\parbox[c]{6cm}{%
\pstree[levelsep=1cm]{\noir{g}}{
   \pstree{\blanc{p}}{\T{$f$}\pstree{\blanc{x}}{\T{$f_g$}\T{$f_d$}}}
   \pstree{\noir{o}}{\T{$c_g$}\T{$c_d$}}
  }
}
\underrightarrow{\quad\text{cas $2.2$}\quad}
\parbox[c]{6cm}{%
\pstree[levelsep=1cm]{\noir{x}}{
   \pstree{\blanc{p}}{\T{$f$}\T{$f_g$}}
   \pstree{\blanc{g}}{\T{$f_d$}\pstree{\noir{o}}{\T{$c_g$}\T{$c_d$}}}
  }
}
$$

$$
\parbox[c]{6cm}{%
\pstree[levelsep=1cm]{\noir{g}}{
   \pstree{\noir{o}}{\T{$c_g$}\T{$c_d$}}
   \pstree{\blanc{p}}{\pstree{\blanc{x}}{\T{$f_g$}\T{$f_d$}}\T{$f$}}
  }
}
\underrightarrow{\quad\text{cas $2.3$}\quad}
\parbox[c]{6cm}{%
\pstree[levelsep=1cm]{\noir{x}}{
   \pstree{\blanc{g}}{\pstree{\noir{o}}{\T{$c_g$}\T{$c_d$}}\T{$f_g$}}
   \pstree{\blanc{p}}{\T{$f_d$}\T{$f$}}
  }
}
$$

$$
\parbox[c]{6cm}{%
\pstree[levelsep=1cm]{\noir{g}}{
   \pstree{\noir{o}}{\T{$c_g$}\T{$c_d$}}
   \pstree{\blanc{p}}{\T{$f$}\pstree{\blanc{x}}{\T{$f_g$}\T{$f_d$}}}
  }
}
\underrightarrow{\quad\text{cas $2.4$}\quad}
\parbox[c]{6cm}{%
\pstree[levelsep=1cm]{\noir{p}}{
   \pstree{\blanc{g}}{\pstree{\noir{o}}{\T{$c_g$}\T{$c_d$}}\T{$f$}}
   \pstree{\blanc{x}}{\T{$f_g$}\T{$f_d$}}
  }
}
$$
\end{question}

\begin{question}
L'insertion d'une nouvelle valeur peut provoquer un changement de
racine dans l'arbre. On modifie donc les fonctions pr�c�dentes pour
envoyer la valeur de la racine~:
\begin{verbatim}
let est_racine a = (pere a)=Vide ;;

let rec mauvais_blanc val_racine x =
   let p=pere x in 
   match p with
     Vide     -> ( inverse_couleur x;  (* x est la racine de l'arbre *)
                   valeur x )
   | Noeud np -> match np.couleur with
        Noir  -> val_racine
      | Blanc -> let o=frere p and g=pere p in
                 match couleur o with
                  Blanc -> ( inverse_couleur p;            (*  cas 1  *) 
                             inverse_couleur o;
                             inverse_couleur g; 
                             mauvais_blanc val_racine g   )
                | Noir  -> if est_fils_gauche p then
                             if est_fils_gauche x then
                               ( rotation_droite g;         (* cas 2.1 *)
                                 inverse_couleur g;
                                 inverse_couleur p;
                                 if (est_racine p) then valeur p 
                                                   else val_racine )
                             else
                               ( rotation_gauche_droite g;  (* cas 2.2 *)
                                 inverse_couleur g;
                                 inverse_couleur x;
                                 if (est_racine x) then valeur x 
                                                   else val_racine )
                           else
                             if est_fils_gauche x then
                               ( rotation_droite_gauche g;  (* cas 2.3 *)
                                 inverse_couleur g;
                                 inverse_couleur x;
                                 if (est_racine x) then valeur x
                                                   else val_racine )
                             else
                               ( rotation_gauche g;         (* cas 2.4 *)
                                 inverse_couleur g;
                                 inverse_couleur p;
                                 if (est_racine p) then valeur p 
                                                   else val_racine ) ;;


let insere x a=
  try (mauvais_blanc (valeur a) (insere_blanc x a))
  with FinInsertion -> (valeur a) ;;
\end{verbatim}
\end{question}
\end{document}
