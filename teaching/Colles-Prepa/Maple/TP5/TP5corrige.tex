\documentclass[12pt,a4paper]{article}
\usepackage{fancyheadings}
\usepackage{pstricks,pst-node,pst-tree}
\usepackage{amsmath}
\usepackage[ansinew]{inputenc}
\usepackage{maple2e}
\usepackage[frenchb]{babel}
 


\title{corrig� TP5 {\sc Maple} : Interpolation}
\author{}
\date{}

\setlength{\oddsidemargin}{0cm}
\addtolength{\textwidth}{70pt}
\setlength{\topmargin}{0cm}
\addtolength{\textheight}{2cm}
\setlength{\parindent}{0cm}

\pagestyle{fancy}
\lhead{Corrig� TP5 {\sc Maple}}
\rhead{Interpolation}


\DefineParaStyle{Heading 1}
\DefineParaStyle{Maple Output}
\DefineParaStyle{Warning}
\DefineCharStyle{2D Comment}
\DefineCharStyle{2D Math}
\DefineCharStyle{2D Output}

\newcounter{numquestion}
\setcounter{numquestion}{1}
\newenvironment{question}{\noindent{\bf Exercice \thenumquestion.}}%
{\stepcounter{numquestion}\medskip}

\newcommand{\R}{\mathbf{R}}
\newcommand{\Z}{\mathbf{Z}}
\newcommand{\C}{\mathbf{C}}
\newcommand{\N}{\mathbf{N}}
\newcommand{\Q}{\mathbf{Q}}
\newcommand{\pgcd}{\operatorname{pgcd}}


\begin{document}
\section{Polyn�mes interpolateurs de Lagrange}

\begin{question}
Pour $i\in\{1,\ldots,n\}$
$$P_i(X)=\frac{\prod\limits_{1\leq j \leq n \atop j\not= i } 
     \left(X-x_j\right)}{\prod\limits_{1\leq j \leq n \atop j\not= i } 
     \left(x_i-x_j\right)}$$
\begin{verbatim}
list_pol:=proc(X)
  local L,t,n,i;
  n:=nops(X);
  t:=product(x-X[i],i=1..n);
  L:=[seq(t/(x-X[i]),i=1..n)];
  seq(L[i]/subs(x=X[i],L[i]),i=1..n);
end:
\end{verbatim}
\end{question}

\begin{question}
$$P(X)=\sum_{i=1}^n y_i P_i(X)$$
\begin{verbatim}
interpol:=proc(X,Y)
  local L,n;
  n:=nops(X);
  L:=list_pol(X);
  sum(Y[i]*L[i],i=1..n);
end:
\end{verbatim}
\end{question}

\begin{question}

\begin{maplegroup}
\begin{mapleinput}
\mapleinline{active}{1d}{X:=[1,2,4,5];}{%
}
\end{mapleinput}

\mapleresult
\begin{maplelatex}
\[
X := [1, \,2, \,4, \,5]
\]
\end{maplelatex}

\end{maplegroup}
\begin{maplegroup}
\begin{mapleinput}
\mapleinline{active}{1d}{Y:=[1,2,4,3];}{%
}
\end{mapleinput}

\mapleresult
\begin{maplelatex}
\[
Y := [1, \,2, \,4, \,3]
\]
\end{maplelatex}

\end{maplegroup}
\begin{maplegroup}
\begin{mapleinput}
\mapleinline{active}{1d}{Z:=expand(interpol(X,Y));}{%
}
\end{mapleinput}

\mapleresult
\begin{maplelatex}
\[
Z :=  - {\displaystyle \frac {1}{6}} \,x^{3} + {\displaystyle 
\frac {7}{6}} \,x^{2} - {\displaystyle \frac {4}{3}} \,x + 
{\displaystyle \frac {4}{3}} 
\]
\end{maplelatex}

\end{maplegroup}
\begin{maplegroup}
\begin{mapleinput}
\mapleinline{active}{1d}{L:=[seq([X[i],Y[i]],i=1..nops(X))];}{%
}
\end{mapleinput}

\mapleresult
\begin{maplelatex}
\[
L := [[1, \,1], \,[2, \,2], \,[4, \,4], \,[5, \,3]]
\]
\end{maplelatex}

\end{maplegroup}
\begin{maplegroup}
\begin{mapleinput}
\mapleinline{active}{1d}{P1:=plot(L,style=point,symbol=circle):}{%
}
\end{mapleinput}

\end{maplegroup}
\begin{maplegroup}
\begin{mapleinput}
\mapleinline{active}{1d}{P2:=plot(Z,x=-0..6):}{%
}
\end{mapleinput}

\end{maplegroup}
\begin{maplegroup}
\begin{mapleinput}
\mapleinline{active}{1d}{plots[display](P1,P2);}{%
}
\end{mapleinput}

\mapleresult
\begin{center}
\mapleplot{tp501.eps}
\end{center}

\end{maplegroup}
\end{question}

\begin{question}

\begin{maplegroup}
\begin{mapleinput}
\mapleinline{active}{1d}{f:=x->1/(1+8*x^2);}{%
}
\end{mapleinput}

\mapleresult
\begin{maplelatex}
\[
f := x\rightarrow {\displaystyle \frac {1}{1 + 8\,x^{2}}} 
\]
\end{maplelatex}

\end{maplegroup}

\begin{verbatim}
   >  runge:=proc(n)
        local X,Y,Z,i,P1,P2,P3;
        X:=[seq(-1+2*i/(n-1),i=0..n-1)];
        Y:=[seq(f(X[i]),i=1..n)];
        Z:=interpol(X,Y);
        P1:=plot(f(x),x=-1..1,color=yellow):
        P2:=plot([seq([X[i],Y[i]],i=1..n)],style=point,symbol=circle,color=black):
        P3:=plot(Z,x=-1..1,color=blue):
        plots[display](P1,P2,P3);
      end:
\end{verbatim}
\newpage
\begin{maplegroup}
\begin{mapleinput}
\mapleinline{active}{1d}{runge(8);}{%
}
\end{mapleinput}

\mapleresult
\begin{center}
\mapleplot{tp502.eps}
\end{center}

\end{maplegroup}

\end{question}

\begin{question} (Ph�nom�ne de Runge)\\

\begin{maplegroup}
\begin{mapleinput}
\mapleinline{active}{1d}{plots[display](seq(runge(j),j=3..20),insequen
ce=true);}{%
}
\end{mapleinput}
\end{maplegroup}

D'apr�s l'animation, il n'y a pas convergence uniforme sur $[-1,1]$, de
la suite des polyn�mes interpolateurs vers $f$ (revoir le TP sur les
suites de fonction).
\end{question}

\begin{question}
Si $t\in[-1,1]$,
\begin{eqnarray*}
  \cos(n\arccos(t))=0 & \iff & n\arccos(t)=\frac{\pi}{2} \mod \pi \\
  & \iff & \arccos(t)=\frac{\pi}{2n} \mod \frac{\pi}{n} \\
  & \iff & \arccos(t)=\frac{\pi}{2n}+k \frac{\pi}{n} \\
  \lefteqn{\quad\quad\quad\quad\quad 
       k=0,\ldots,n-1\quad\text{car } \arccos(t)\in [-\pi,\pi]} \\
  & \iff & t = \cos\left(\frac{\pi(1+2k)}{2n}  \right) \quad k=0,\ldots,n-1\quad
\end{eqnarray*}
\begin{verbatim}
   >  rungeTcheb:=proc(n)
        local X,Y,Z,i,P1,P2,P3;
        X:=[seq(cos((2*i+1)*Pi/(2*n)),i=0..n-1)];
        Y:=[seq(f(X[i]),i=1..n)];
        Z:=interpol(X,Y);
        P1:=plot(f(x),x=-1..1,color=yellow):
        P2:=plot([seq([X[i],Y[i]],i=1..n)],style=point,symbol=circle,color=black):
        P3:=plot(Z,x=-1..1,color=blue):
        plots[display](P1,P2,P3);
      end:
\end{verbatim}

\begin{maplegroup}
\begin{mapleinput}
\mapleinline{active}{1d}{rungeTcheb(8);}{%
}
\end{mapleinput}

\mapleresult
\begin{center}
\mapleplot{tp504.eps}
\end{center}

\end{maplegroup}
\begin{maplegroup}
\begin{mapleinput}
\mapleinline{active}{1d}{plots[display](seq(rungeTcheb(j),j=3..12),ins
equence=true);}{%
}
\end{mapleinput}
\end{maplegroup}
\end{question}

\section{Splines cubiques}

\begin{question} \\

\begin{maplegroup}
\begin{mapleinput}
\mapleinline{active}{1d}{X:=[0,2,3,4];}{%
}
\end{mapleinput}

\mapleresult
\begin{maplelatex}
\[
X := [0, \,2, \,3, \,4]
\]
\end{maplelatex}

\end{maplegroup}
\begin{maplegroup}
\begin{mapleinput}
\mapleinline{active}{1d}{Y:=[2,0,3,1];}{%
}
\end{mapleinput}

\mapleresult
\begin{maplelatex}
\[
Y := [2, \,0, \,3, \,1]
\]
\end{maplelatex}

\end{maplegroup}
\begin{maplegroup}
\begin{mapleinput}
\mapleinline{active}{1d}{n:=nops(X);}{%
}
\end{mapleinput}

\mapleresult
\begin{maplelatex}
\[
n := 4
\]
\end{maplelatex}

\end{maplegroup}

{\bf 7.1}
\begin{verbatim}
   >  for k from 1 to n-1 do
        P[k]:=a[k]*x^3+b[k]*x^2+c[k]*x+d[k];
        DP[k]:=diff(P[k],x);
        DDP[k]:=diff(DP[k],x);
      od;
\end{verbatim}

{\bf 7.2}

\begin{maplegroup}
\begin{mapleinput}
\mapleinline{active}{1d}{cond1:=seq(subs(x=X[k],P[k])=Y[k],k=1..n-1);}
{%
}
\end{mapleinput}
\end{maplegroup}
\begin{maplegroup}
\begin{mapleinput}
\mapleinline{active}{1d}{cond2:=seq(subs(x=X[k+1],P[k])=Y[k+1],k=1..n-
1);}{%
}
\end{mapleinput}
\end{maplegroup}
\begin{maplegroup}
\begin{mapleinput}
\mapleinline{active}{1d}{cond3:=seq(subs(x=X[k],DP[k-1]=DP[k]),k=2..n-
1);}{%
}
\end{mapleinput}
\end{maplegroup}
\begin{maplegroup}
\begin{mapleinput}
\mapleinline{active}{1d}{cond4:=seq(subs(x=X[k],DDP[k-1]=DDP[k]),k=2..
n-1);}{%
}
\end{mapleinput}
\end{maplegroup}
\begin{maplegroup}
\begin{mapleinput}
\mapleinline{active}{1d}{cond5:=subs(x=X[1],DP[1])=0,subs(x=X[n],DP[n-
1])=0;}{%
}
\end{mapleinput}
\end{maplegroup}

{\bf 7.3}

\begin{maplegroup}
\begin{mapleinput}
\mapleinline{active}{1d}{solve(\{cond1,cond2,cond3,cond4,cond5\});}{%
}
\end{mapleinput}
\end{maplegroup}

{\bf 7.4}

\begin{maplegroup}
\begin{mapleinput}
\mapleinline{active}{1d}{assign(");}{%
}
\end{mapleinput}
\end{maplegroup}

{\bf 7.5}

\begin{maplegroup}
\begin{mapleinput}
\mapleinline{active}{1d}{plot1:=plot([seq([X[i],Y[i]],i=1..n)],style=p
oint,symbol=circle,color=black):}{%
}
\end{mapleinput}

\end{maplegroup}
\begin{maplegroup}
\begin{mapleinput}
\mapleinline{active}{1d}{plots[display](plot1,seq(plot(P[k],x=X[k]..X[
k+1],color=blue),k=1..n-1));}{%
}
\end{mapleinput}

\mapleresult
\begin{center}
\mapleplot{tp506.eps}
\end{center}
\end{maplegroup}
\end{question}

\begin{question}

\begin{maplegroup}
\begin{mapleinput}
\mapleinline{active}{1d}{f:=x->sin(x^2);}{%
}
\end{mapleinput}

\mapleresult
\begin{maplelatex}
\[
f := x\rightarrow {\rm sin}(x^{2})
\]
\end{maplelatex}

\end{maplegroup}
\begin{verbatim}
   >  list_pol:=proc(X,Y)
        local L,t,n,i;
        n:=nops(X);
        t:=product(x-X[i],i=1..n);
        L:=[seq(t/(x-X[i]),i=1..n)];
      end:
\end{verbatim}

On reprend les fonctions d'interpolation de Lagrange en consid�rant cette fois
l'intervalle $[0,4]$~:
\begin{verbatim}
   >  interpol:=proc(X,Y)
        local L,n;
        n:=nops(X);
        L:=list_pol(X,Y);
        sum(Y[i]*L[i]/subs(x=X[i],L[i]),i=1..n);
      end:
\end{verbatim}

\begin{verbatim}
   >  plot_interpol:=proc(n)
        local X,Y,Z,i,P1,P2,P3;
        X:=[seq(4*i/(n-1),i=0..n-1)];
        Y:=[seq(f(X[i]),i=1..n)];
        Z:=interpol(X,Y);
        P1:=plot(f(x),x=0..4,color=yellow):
        P2:=plot([seq([X[i],Y[i]],i=1..n)],style=point,symbol=circle,color=black):
        P3:=plot(Z,x=0..4,color=blue):
        plots[display](P1,P2,P3);
      end:
\end{verbatim}

\begin{verbatim}
   >  spline:=proc(X,Y)
        local n,k,P,Df,DP,DDP,cond1,cond2,cond3,cond4,cond5;
        n:=nops(X);
        Df:=diff(f(x),x);
        unassign(a,b,c,d);
        for k from 1 to n-1 do
          P[k]:=a[k]*x^3+b[k]*x^2+c[k]*x+d[k];
          DP[k]:=diff(P[k],x);
          DDP[k]:=diff(DP[k],x);
        od;
        cond1:=seq(subs(x=X[k],P[k])=Y[k],k=1..n-1);
        cond2:=seq(subs(x=X[k+1],P[k])=Y[k+1],k=1..n-1);
        cond3:=seq(subs(x=X[k],DP[k-1]=DP[k]),k=2..n-1);
        cond4:=seq(subs(x=X[k],DDP[k-1]=DDP[k]),k=2..n-1);
        cond5:=subs(x=X[1],DP[1]=Df),subs(x=X[n],DP[n-1]=Df);
        solve(\{cond1,cond2,cond3,cond4,cond5\});
        assign(");
        P;
      end:
\end{verbatim}
\newpage
\begin{maplegroup}
\begin{mapleinput}
\mapleinline{active}{1d}{plot_spline(6);}{%
}
\end{mapleinput}

\mapleresult
\begin{center}
\mapleplot{tp507.eps}
\end{center}

\end{maplegroup}
\begin{maplegroup}
\begin{mapleinput}
\mapleinline{active}{1d}{plot_interpol(6);}{%
}
\end{mapleinput}

\mapleresult
\begin{center}
\mapleplot{tp508.eps}
\end{center}

\end{maplegroup}

Avec $6$ points, l'am�lioration n'est pas flagrante !

\newpage
\begin{maplegroup}
\begin{mapleinput}
\mapleinline{active}{1d}{plot_spline(12);}{%
}
\end{mapleinput}

\mapleresult
\begin{center}
\mapleplot{tp509.eps}
\end{center}

\end{maplegroup}
\begin{maplegroup}
\begin{mapleinput}
\mapleinline{active}{1d}{plot_interpol(12);}{%
}
\end{mapleinput}

\mapleresult
\begin{center}
\mapleplot{tp510.eps}
\end{center}

\end{maplegroup}

Avec $12$ points, la spline r�alise une bien meilleure interpolation
que le polyn�me de Lagrange.
\end{question}

\end{document}

