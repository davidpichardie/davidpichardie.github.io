\documentclass[12pt,a4paper]{article}
\usepackage{fancyheadings}
\usepackage{pstricks,pst-node,pst-tree}
\usepackage{amsmath}
\usepackage[ansinew]{inputenc}

\title{TP3 {\sc Maple} : Suites}
\author{}
\date{}


\setlength{\oddsidemargin}{0cm}
\addtolength{\textwidth}{70pt}
\setlength{\topmargin}{0cm}
\addtolength{\textheight}{2cm}
\setlength{\parindent}{0cm}


\pagestyle{fancy}
\lhead{TP3 {\sc Maple}}
\rhead{Suites}

\newcommand{\R}{\mathbf{R}}
\newcommand{\C}{\mathbf{C}}
\newcommand{\N}{\mathbf{N}}
 
\newcounter{numsubquestion}
\newcounter{numquestion}
\setcounter{numquestion}{1}
\newenvironment{question}{\setcounter{numsubquestion}{1}\noindent{\bf Exercice \thenumquestion.}}%
{\stepcounter{numquestion}\medskip}

\newenvironment{subquestion}{\smallskip\noindent{\bf \thenumquestion.\thenumsubquestion}}%
{\stepcounter{numsubquestion}\smallskip}


\begin{document}

\maketitle

\begin{question} {\bf Comparaison entre calcul formel et calcul num�rique}\\
\begin{subquestion}
Ecrire une proc�dure qui calcule le $n$-i�me terme de la suite suivante
$$\left\lbrace
\begin{array}{l}
A_0 = \frac{11}{2} \\
A_1 = \frac{61}{11} \\
A_n = 111-\frac{1130}{A_{n-1}}+\frac{3000}{A_{n-1}A_{n-2}}\quad\text{si }n>1
\end{array}\right.$$
\end{subquestion}

\vspace{-.5cm}

\begin{subquestion}
Ecrire une proc�dure qui calcule les $n$ premiers termes de la suite $(A_n)_{n\in\N}$
sous la forme $[[0,A_0],[1,A_1],\ldots,[n,A_n]]$. Repr�senter ainsi le comportement
de la suite avec un graphique.
\end{subquestion}

\begin{subquestion}
Recommencer le travail pr�c�dent en prenant des valeurs num�riques
({\tt evalf}) 
pour $A_0$ et $A_1$.
Comparer les deux calculs.
\end{subquestion}

\begin{subquestion}
Reprendre les questions pr�c�dentes pour la suite suivante
$$\left\lbrace
\begin{array}{l}
U_0 = 1 \\
U_1 = \frac{1}{3} \\
U_n = \frac{10}{3}U_{n-1}-U_{n-2}\quad\text{si }n>1
\end{array}\right.$$
\end{subquestion}

\vspace{-.5cm}

\begin{subquestion}
Donner l'expression de $U_n$ en fonction de $n$ ({\tt rsolve}) pour $U_1=1$ et
$U_1=1+\varepsilon$. Comparer les limites des suites obtenues.
\end{subquestion}
\end{question}

\begin{question} {\bf D�veloppement asymptotique}\\
On d�finit la suite $u_n$ comme �tant la solution unique de l'�quation
$\tan(u_n)=u_n$ dans l'intervalle
$](n-\frac{1}{2})\pi,(n+\frac{1}{2})\pi[$, pour tout $n\in\N$.
Le but de cet exercice est de trouver un d�veloppement asymptotique de
cette suite.

\begin{subquestion}
Calculer num�riquement les valeurs de $u_1$, $u_2$, $u_3$ et $u_{10}$
({\tt fsolve}).
\end{subquestion}

\begin{subquestion}
Montrer que $\lim\limits_{n\to+\infty}u_n=+\infty$.
\end{subquestion}

\begin{subquestion}
Pour $n\in\N$, on pose $v_n=n\pi+\frac{\pi}{2}-u_n$. Montrer que 
$\lim\limits_{n\to+\infty}v_n=0$. V�rifier ce r�sultat graphiquement.
\end{subquestion}

\begin{subquestion}
On suppose que $u_n=n\pi+\frac{\pi}{2}+\frac{a_1}{n}+\frac{a_2}{n^2}
+\cdots+\frac{a_p}{n^p}+o\left(\frac{1}{n^p}\right)$ ({\tt
  sum}). D�terminer $a_1$, $a_2$ et $a_3$ en identifiant $u_n$ avec le
d�veloppement asymptotique de $\tan(u_n-n\pi)$ ({\tt series}). Pour
manipuler les coefficients du d�veloppement, le convertir en polyn�me
({\tt convert}), puis utiliser {\tt coeffs}. V�rifier graphiquement ce
r�sultat. 
\end{subquestion}

\begin{subquestion}
D�terminer $a_1,\ldots,a_{12}$.
\end{subquestion}

\begin{subquestion}
Comparer la valeur num�rique de $u_{10}$ obtenu avec {\tt fsolve} et
avec le dernier d�veloppement asymptotique.
\end{subquestion}

\end{question}

\end{document}

