\documentclass[12pt,a4paper]{article}
\usepackage{fancyheadings}
\usepackage{pstricks,pst-node,pst-tree}
\usepackage{amsmath}
\usepackage[ansinew]{inputenc}
\usepackage{maple2e}
 


\title{corrig� TP3 {\sc Maple} : Suites}
\author{}
\date{}

\setlength{\oddsidemargin}{0cm}
\addtolength{\textwidth}{70pt}
\setlength{\topmargin}{0cm}
\addtolength{\textheight}{2cm}
\setlength{\parindent}{0cm}

\pagestyle{fancy}
\lhead{Corrig� TP3 {\sc Maple}}
\rhead{Suites}


\DefineParaStyle{Heading 1}
\DefineParaStyle{Maple Output}
\DefineParaStyle{Warning}
\DefineCharStyle{2D Comment}
\DefineCharStyle{2D Math}
\DefineCharStyle{2D Output}

\newcounter{numquestion}
\setcounter{numquestion}{1}
\newenvironment{question}{\noindent{\bf Exercice \thenumquestion.}}%
{\stepcounter{numquestion}\medskip}

\newcommand{\R}{\mathbf{R}}
\newcommand{\C}{\mathbf{C}}
\newcommand{\N}{\mathbf{N}}
\newcommand{\Q}{\mathbf{Q}}
\newcommand{\pgcd}{\operatorname{pgcd}}


\begin{document}

{\bf\Large Exercice 1}

Pour r�duire les temps de calcul, on utilise l'option {\tt remember}
qui permet � {\sc Maple} de se souvenir des calculs pr�c�dents et on
appelle qu'une seule fois l'�valuation de {\tt A(n-1)}~:

\begin{maplegroup}
\begin{mapleinput}
\mapleinline{active}{1d}{A:=proc(n)}{}
\vspace{-.5cm}
\begin{verbatim}
      local t;
      option remenber;
      if n=0 then 11/2
      elif n=1 then 61/11
      else t:=A(n-1);
         111-(1130/t)+(3000/(t*A(n-2)))
      fi;
      end:
\end{verbatim}
\end{mapleinput}
\end{maplegroup}

On peut encore acc�l�rer les calculs grace � une boucle {\tt for} et �
deux variables qui repr�sentent les deux derniers termes calcul�s. On
garde en m�moire les termes successivement calcul�s dans une liste de
fa�on � tracer la suite avec {\tt plot}.

\begin{maplegroup}
\begin{mapleinput}
\mapleinline{active}{1d}{Aliste1:=proc(n)}{}
\vspace{-.5cm}
\begin{verbatim}
      local i,a,b,c,L;
      a:=11/2;
      b:=61/11;
      L:=[0,a],[1,b];
      for i from 2 to n do
        c:=111-(1130/b)+(3000/(a*b));
        a:=b;
        b:=c;
        L:=L,[i,c];
      od;
      [L]; 
      end:
\end{verbatim}
\end{mapleinput}

\end{maplegroup}
\begin{maplegroup}
\begin{mapleinput}
\mapleinline{active}{1d}{plot(Aliste1(100),n=0..100);}{%
}
\end{mapleinput}

\mapleresult
\begin{center}
\mapleplot{TP301.eps}
\end{center}

\end{maplegroup}

On constate que la suite converge vers $6$. Il s'agit ici d'un calcul
formel~: {\sc Maple} ne manipule que des rationnels.

\begin{maplegroup}
\begin{mapleinput}
\mapleinline{active}{1d}{Aliste2:=proc(n)}{}
\vspace{-.5cm}
\begin{verbatim}
      local i,a,b,c,L;
      a:=evalf(11/2);
      b:=evalf(61/11);
      L:=[0,a],[1,b];
      for i from 2 to n do
        c:=111-(1130/b)+(3000/(a*b));
        a:=b;
        b:=c;
        L:=L,[i,c];
      od;
      [L];
      end:
\end{verbatim}
\end{mapleinput}
\end{maplegroup}
\begin{maplegroup}
\begin{mapleinput}
\mapleinline{active}{1d}{plot(Aliste2(100),n=0..100);}{%
}
\end{mapleinput}

\mapleresult
\begin{center}
\mapleplot{TP302.eps}
\end{center}

\end{maplegroup}

Si on force {\sc Maple} � faire du calcul num�rique (en donnant les
premi�res valeurs de la suite sous forme d�cimal), la suite calcul�e
semble converger vers $100$. Le calcul num�rique donne donc un
r�sultat faux.

\begin{maplegroup}
\begin{mapleinput}
\mapleinline{active}{1d}{Uliste1:=proc(n)}{}
\vspace{-.5cm}
\begin{verbatim}
      local i,a,b,c,L;
      a:=1;
      b:=1/3;
      L:=[0,a],[1,b];
      for i from 2 to n do
        c:=10/3*b-a;
        a:=b;
        b:=c;
        L:=L,[i,c];
      od;
      [L];
      end:
\end{verbatim}
\end{mapleinput}

\end{maplegroup}
\begin{maplegroup}
\begin{mapleinput}
\mapleinline{active}{1d}{plot(Uliste1(20),n=0..20);}{%
}
\end{mapleinput}

\mapleresult
\begin{center}
\mapleplot{TP303.eps}
\end{center}

\end{maplegroup}
\begin{maplegroup}
\begin{mapleinput}
\mapleinline{active}{1d}{Uliste2:=proc(n)}{}
\vspace{-.5cm}
\begin{verbatim}
      local i,a,b,c,L;
      a:=1;
      b:=evalf(1/3);
      L:=[0,a],[1,b];
      for i from 2 to n do
        c:=10/3*b-a;
        a:=b;
        b:=c;
        L:=L,[i,c];
      od;
      [L];
      end:
\end{verbatim}
\end{mapleinput}
\end{maplegroup}
\begin{maplegroup}
\begin{mapleinput}
\mapleinline{active}{1d}{plot(Uliste2(50),n=0..50);}{%
}
\end{mapleinput}

\mapleresult
\begin{center}
\mapleplot{TP304.eps}
\end{center}

\end{maplegroup}

Le calcul exacte sur $\Q$ montre que cette suite tend
vers $0$, alors que le calcul num�rique diverge.

\begin{maplegroup}
\begin{mapleinput}
\mapleinline{active}{1d}{rsolve(\{y(n)=10/3*y(n-1)-y(n-2),y(0)=1,y(1)=
1/3\},y);}{%
}
\end{mapleinput}

\mapleresult
\begin{maplelatex}
\[
({\displaystyle \frac {1}{3}} )^{n}
\]
\end{maplelatex}

\end{maplegroup}
\begin{maplegroup}
\begin{mapleinput}
\mapleinline{active}{1d}{rsolve(\{y(n)=10/3*y(n-1)-y(n-2),y(0)=1,y(1)=
1/3+epsilon\},y);}{%
}
\end{mapleinput}

\mapleresult
\begin{maplelatex}
\[
{\displaystyle \frac {3}{8}} \,\varepsilon \,3^{n} - 
{\displaystyle \frac {1}{3}} \,({\displaystyle \frac {9}{8}} \,
\varepsilon  - 3)\,({\displaystyle \frac {1}{3}} )^{n}
\]
\end{maplelatex}

\end{maplegroup}

Lorsque {\sc Maple} calcule {\tt evalf(1/3)}, il manipule en fait
$\frac{1}{3}+\varepsilon$ avec $\varepsilon\not=0$ et ce dernier
calcul explique alors pourquoi la suite diverge (sans compter les
autres erreurs d'arrondis au cours du calcul).

\medskip
{\bf\Large Exercice 2}
\medskip

Pour calculer les termes de la suite on utilise la fonction {\tt
fsolve} qui permet de r�soudre des �quations non-lin�aires de mani�re
approch�e. 

\begin{maplegroup}
\begin{mapleinput}
\mapleinline{active}{1d}{u:=n->fsolve(tan(x)=x,x,(n-1/2)*Pi..(n+1/2)*P
i);}{%
}
\end{mapleinput}

\mapleresult
\begin{maplelatex}
\[
u := n\rightarrow {\rm fsolve}({\rm tan}(x)=x, \,x, \,(n - 
{\displaystyle \frac {1}{2}} )\,\pi  .. (n + {\displaystyle 
\frac {1}{2}} )\,\pi )
\]
\end{maplelatex}

\end{maplegroup}
\begin{maplegroup}
\begin{mapleinput}
\mapleinline{active}{1d}{seq(u(i),i=1..6);}{%
}
\end{mapleinput}

\mapleresult
\begin{maplelatex}
\[
4.493409458, \,7.725251837, \,10.90412166, \,14.06619391, \,
17.22075527, \,20.37130296
\]
\end{maplelatex}

\end{maplegroup}

Pour tout $n\in\N$,
$$(n-\frac{1}{2})\pi\leq n\leq(n+\frac{1}{2})\pi$$
donc $\lim\limits_{n\to+\infty}u_n=+\infty$.

Pour $n\in\N^*$, $](n-\frac{1}{2})\pi,(n+\frac{1}{2})\pi[
\subset ]0,+\infty[$ donc $u_n>0$. On a ainsi $\tan(u_n)>0$ et
$u_n\in]n\pi,(n+\frac{1}{2})\pi[$. Ceci permet d'affirmer que
$v_n\in]0,\frac{\pi}{2}[$ pour $n\in\N^*$. On peut donc calculer
$\tan(v_n)$~:
\begin{eqnarray*}
 \tan(v_n) &=& \tan(n\pi+\frac{\pi}{2}-u_n) \\
           &=& \tan(\frac{\pi}{2}-u_n) \text{ par $\pi$-p�riodicit�} \\
           &=& \frac{1}{\tan(u_n)} 
\end{eqnarray*}

Or $\lim\limits_{n\to+\infty}u_n=+\infty$ donc
$\lim\limits_{n\to+\infty}\tan(v_n)=0$. Mais puisque
$v_n\in]0,\frac{\pi}{2}[$, on a $v_n=\operatorname{arctan}(\tan(v_n))$
et ainsi $\lim\limits_{n\to+\infty}v_n=0$.

\begin{maplegroup}
\begin{mapleinput}
\mapleinline{active}{1d}{v:=n->n*Pi+Pi/2-u(n);}{%
}
\end{mapleinput}

\mapleresult
\begin{maplelatex}
\[
v := n\rightarrow n\,\pi  + {\displaystyle \frac {1}{2}} \,\pi 
 - {\rm u}(n)
\]
\end{maplelatex}

\end{maplegroup}
\begin{maplegroup}
\begin{mapleinput}
\mapleinline{active}{1d}{plot([seq([i,v(i)],i=1..50)],i=1..50);}{%
}
\end{mapleinput}

\mapleresult
\begin{center}
\mapleplot{TP305.eps}
\end{center}

\end{maplegroup}

\begin{maplegroup}
\begin{mapleinput}
\mapleinline{active}{1d}{U:=n*Pi+Pi/2+sum(a[k]/n^k,k=1..p);}{%
}
\end{mapleinput}

\mapleresult
\begin{maplelatex}
\[
U := n\,\pi  + {\displaystyle \frac {1}{2}} \,\pi  +  \left(  \! 
{\displaystyle \sum _{k=1}^{p}} \,{\displaystyle \frac {{a_{k}}}{
n^{k}}}  \!  \right) 
\]
\end{maplelatex}

\end{maplegroup}
\begin{maplegroup}
\begin{mapleinput}
\mapleinline{active}{1d}{p:=3;}{%
}
\end{mapleinput}

\mapleresult
\begin{maplelatex}
\[
p := 3
\]
\end{maplelatex}

\end{maplegroup}
\begin{maplegroup}
\begin{mapleinput}
\mapleinline{active}{1d}{U;}{%
}
\end{mapleinput}

\mapleresult
\begin{maplelatex}
\[
n\,\pi  + {\displaystyle \frac {1}{2}} \,\pi  + {\displaystyle 
\frac {{a_{1}}}{n}}  + {\displaystyle \frac {{a_{2}}}{n^{2}}}  + 
{\displaystyle \frac {{a_{3}}}{n^{3}}} 
\]
\end{maplelatex}

\end{maplegroup}

Gr�ce � quelques essais, on constate qu'il faut effectuer un
d�veloppement � l'ordre $p-1$ pour avoir les $p$ premiers
coefficients.

\begin{maplegroup}
\begin{mapleinput}
\mapleinline{active}{1d}{series(tan(U-n*Pi)-U,n=infinity,p-1);}{%
}
\end{mapleinput}

\mapleresult
\begin{maplelatex}
\[
( - {\displaystyle \frac {1}{{a_{1}}}}  - \pi )\,n + 
{\displaystyle \frac {{a_{2}}}{{a_{1}}^{2}}}  - {\displaystyle 
\frac {1}{2}} \,\pi  + {\displaystyle \frac { - {\displaystyle 
\frac { - {\displaystyle \frac {{a_{3}}}{{a_{1}}}}  + 
{\displaystyle \frac {{a_{2}}^{2}}{{a_{1}}^{2}}} }{{a_{1}}}}  - 
{\displaystyle \frac {2}{3}} \,{a_{1}}}{n}}  + {\rm O}(
{\displaystyle \frac {1}{n^{2}}} )
\]
\end{maplelatex}

\end{maplegroup}
\begin{maplegroup}
\begin{mapleinput}
\mapleinline{active}{1d}{convert(",polynom):}{%
}
\end{mapleinput}

\end{maplegroup}
\begin{maplegroup}
\begin{mapleinput}
\mapleinline{active}{1d}{coeffs(",n):}{%
}
\end{mapleinput}

\end{maplegroup}
\begin{maplegroup}
\begin{mapleinput}
\mapleinline{active}{1d}{S:=\{"\}:}{%
}
\end{mapleinput}

\end{maplegroup}
\begin{maplegroup}
\begin{mapleinput}
\mapleinline{active}{1d}{solve(S);}{%
}
\end{mapleinput}

\mapleresult
\begin{maplelatex}
\[
\{{a_{2}}={\displaystyle \frac {1}{2}} \,{\displaystyle \frac {1
}{\pi }} , \,{a_{3}}= - {\displaystyle \frac {1}{12}} \,
{\displaystyle \frac {3\,\pi ^{2} + 8}{\pi ^{3}}} , \,{a_{1}}= - 
{\displaystyle \frac {1}{\pi }} \}
\]
\end{maplelatex}

\end{maplegroup}
\begin{maplegroup}
\begin{mapleinput}
\mapleinline{active}{1d}{assign(");}{%
}
\end{mapleinput}

\end{maplegroup}
\begin{maplegroup}
\begin{mapleinput}
\mapleinline{active}{1d}{U;}{%
}
\end{mapleinput}

\mapleresult
\begin{maplelatex}
\[
n\,\pi  + {\displaystyle \frac {1}{2}} \,\pi  - {\displaystyle 
\frac {1}{\pi \,n}}  + {\displaystyle \frac {1}{2}} \,
{\displaystyle \frac {1}{\pi \,n^{2}}}  - {\displaystyle \frac {1
}{12}} \,{\displaystyle \frac {3\,\pi ^{2} + 8}{\pi ^{3}\,n^{3}}
} 
\]
\end{maplelatex}

\end{maplegroup}

Pour v�rifier la valeur de $a_1$, on regarde
$\lim\limits_{n\to+\infty}\pi n \left(n\pi+\frac{\pi}{2}-u_n\right)$
qui doit valoir $1$~: 

\begin{maplegroup}
\begin{mapleinput}
\mapleinline{active}{1d}{w:=n->Pi*n*v(n);}{%
}
\end{mapleinput}

\mapleresult
\begin{maplelatex}
\[
w := n\rightarrow \pi \,n\,{\rm v}(n)
\]
\end{maplelatex}

\end{maplegroup}
\begin{maplegroup}
\begin{mapleinput}
\mapleinline{active}{1d}{plot([seq([i,w(i)],i=1..50)],i=1..50);}{%
}
\end{mapleinput}

\mapleresult
\begin{center}
\mapleplot{TP306.eps}
\end{center}

\end{maplegroup}
\begin{maplegroup}
\begin{mapleinput}
\mapleinline{active}{1d}{evalf(w(1000));}{%
}
\end{mapleinput}

\mapleresult
\begin{maplelatex}
\[
.9990264640
\]
\end{maplelatex}

\end{maplegroup}
 
On peut aussi v�rifier les autres coefficients de cette mani�re.

En poussant plus loin les calculs, on obtient~:

\begin{maplegroup}
\begin{mapleinput}
\mapleinline{active}{1d}{unassign('a[1]','a[2]','a[3]','p');}{%
}
\end{mapleinput}

\end{maplegroup}
\begin{maplegroup}
\begin{mapleinput}
\mapleinline{active}{1d}{U;}{%
}
\end{mapleinput}

\mapleresult
\begin{maplelatex}
\[
n\,\pi  + {\displaystyle \frac {1}{2}} \,\pi  + {\displaystyle 
\frac {{a_{1}}}{n}}  + {\displaystyle \frac {{a_{2}}}{n^{2}}}  + 
{\displaystyle \frac {{a_{3}}}{n^{3}}}  + {\displaystyle \frac {{
a_{4}}}{n^{4}}}  + {\displaystyle \frac {{a_{5}}}{n^{5}}}  + 
{\displaystyle \frac {{a_{6}}}{n^{6}}}  + {\displaystyle \frac {{
a_{7}}}{n^{7}}}  + {\displaystyle \frac {{a_{8}}}{n^{8}}}  + 
{\displaystyle \frac {{a_{9}}}{n^{9}}}  + {\displaystyle \frac {{
a_{10}}}{n^{10}}}  + {\displaystyle \frac {{a_{11}}}{n^{11}}}  + 
{\displaystyle \frac {{a_{12}}}{n^{12}}} 
\]
\end{maplelatex}

\end{maplegroup}
\begin{maplegroup}
\begin{mapleinput}
\mapleinline{active}{1d}{p:=12:}{%
}
\end{mapleinput}

\end{maplegroup}
\begin{maplegroup}
\begin{mapleinput}
\mapleinline{active}{1d}{series(tan(U-n*Pi)-U,n=infinity,p-1):}{%
}
\end{mapleinput}

\end{maplegroup}
\begin{maplegroup}
\begin{mapleinput}
\mapleinline{active}{1d}{convert(",polynom):}{%
}
\end{mapleinput}

\end{maplegroup}
\begin{maplegroup}
\begin{mapleinput}
\mapleinline{active}{1d}{S:=\{coeffs(",n)\}:}{%
}
\end{mapleinput}

\end{maplegroup}
\begin{maplegroup}
\begin{mapleinput}
\mapleinline{active}{1d}{solve(S):}{%
}
\end{mapleinput}

\end{maplegroup}
\begin{maplegroup}
\begin{mapleinput}
\mapleinline{active}{1d}{assign(");}{%
}
\end{mapleinput}

\end{maplegroup}
\begin{maplegroup}
\begin{mapleinput}
\mapleinline{active}{1d}{U;}{%
}
\end{mapleinput}

\mapleresult
\begin{maplelatex}
\begin{eqnarray*}
\lefteqn{n\,\pi  + {\displaystyle \frac {1}{2}} \,\pi  - 
{\displaystyle \frac {1}{\pi \,n}}  + {\displaystyle \frac {1}{2}
} \,{\displaystyle \frac {1}{\pi \,n^{2}}}  - {\displaystyle 
\frac {1}{12}} \,{\displaystyle \frac {3\,\pi ^{2} + 8}{\pi ^{3}
\,n^{3}}}  + {\displaystyle \frac {1}{8}} \,{\displaystyle 
\frac {8 + \pi ^{2}}{\pi ^{3}\,n^{4}}}  - {\displaystyle \frac {1
}{240}} \,{\displaystyle \frac {240\,\pi ^{2} + 15\,\pi ^{4} + 
208}{\pi ^{5}\,n^{5}}} } \\
 & & \mbox{} + {\displaystyle \frac {1}{96}} \,{\displaystyle 
\frac {208 + 80\,\pi ^{2} + 3\,\pi ^{4}}{\pi ^{5}\,n^{6}}}  - 
{\displaystyle \frac {1}{6720}} \,{\displaystyle \frac {21840\,
\pi ^{2} + 9344 + 4200\,\pi ^{4} + 105\,\pi ^{6}}{\pi ^{7}\,n^{7}
}}  \\
 & & \mbox{} + {\displaystyle \frac {1}{1920}} \,{\displaystyle 
\frac {9344 + 7280\,\pi ^{2} + 840\,\pi ^{4} + 15\,\pi ^{6}}{\pi 
^{7}\,n^{8}}}  \\
 & & \mbox{} - {\displaystyle \frac {1}{80640}} \,{\displaystyle 
\frac {23520\,\pi ^{6} + 784896\,\pi ^{2} + 199936 + 315\,\pi ^{8
} + 305760\,\pi ^{4}}{\pi ^{9}\,n^{9}}}  \\
 & & \mbox{} + {\displaystyle \frac {1}{17920}} \,{\displaystyle 
\frac {199936 + 261632\,\pi ^{2} + 61152\,\pi ^{4} + 3360\,\pi ^{
6} + 35\,\pi ^{8}}{\pi ^{9}\,n^{10}}}  - {\displaystyle \frac {1
}{3548160}}  \\
 & & {\displaystyle \frac {98968320\,\pi ^{2} + 16719872 + 
64753920\,\pi ^{4} + 10090080\,\pi ^{6} + 415800\,\pi ^{8} + 3465
\,\pi ^{10}}{\pi ^{11}\,n^{11}}}  \\
 & & \mbox{} + {\displaystyle \frac {1}{645120}} \,
{\displaystyle \frac {32989440\,\pi ^{2} + 16719872 + 12950784\,
\pi ^{4} + 1441440\,\pi ^{6} + 46200\,\pi ^{8} + 315\,\pi ^{10}}{
\pi ^{11}\,n^{12}}} 
\end{eqnarray*}
\end{maplelatex}

\end{maplegroup}

On compare les deux calculs de $u_{10}$~:

\begin{maplegroup}
\begin{mapleinput}
\mapleinline{active}{1d}{evalf(subs(\{n=10\},U));}{%
}
\end{mapleinput}

\mapleresult
\begin{maplelatex}
\[
32.95638904
\]
\end{maplelatex}

\end{maplegroup}
\begin{maplegroup}
\begin{mapleinput}
\mapleinline{active}{1d}{u(10);}{%
}
\end{mapleinput}

\mapleresult
\begin{maplelatex}
\[
32.95638904
\]
\end{maplelatex}

\end{maplegroup}
\begin{maplegroup}
\begin{mapleinput}
\end{mapleinput}

\end{maplegroup}


\end{document}