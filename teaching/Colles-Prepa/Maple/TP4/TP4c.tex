%% Created by Maple V Release 4 (IBM INTEL NT)
%% Source Worksheet: cor_fourier.mws
%% Generated: Mon Nov 26 20:59:46 2001
\documentclass{article}
\usepackage{maple2e}
\DefineParaStyle{Bullet Item}
\DefineParaStyle{Error}
\DefineParaStyle{Heading 1}
\DefineParaStyle{Maple Output}
\DefineParaStyle{Maple Plot}
\DefineParaStyle{Warning}
\DefineCharStyle{2D Comment}
\DefineCharStyle{2D Math}
\DefineCharStyle{2D Output}
\begin{document}

\section{Exercice 1}

\begin{maplegroup}
\begin{center}
\textbf{Premi�re suite de fonctions}
\end{center}

\end{maplegroup}
\begin{maplegroup}
\begin{mapleinput}
\mapleinline{active}{1d}{restart;}{%
}
\end{mapleinput}

\end{maplegroup}
\begin{maplegroup}
\begin{mapleinput}
\mapleinline{active}{1d}{norminf:=proc(f,n,a,b)
  maximize(abs(f(x,n)),x,\{x=a..b\})
end:}{%
}
\end{mapleinput}

\end{maplegroup}
\begin{maplegroup}
\begin{mapleinput}
\mapleinline{active}{1d}{f := (x,n)->x^n;}{%
}
\end{mapleinput}

\mapleresult
\begin{maplelatex}
\[
f := (x, \,n)\rightarrow x^{n}
\]
\end{maplelatex}

\end{maplegroup}
\begin{maplegroup}
\begin{mapleinput}
\mapleinline{active}{1d}{norminf(f,5,0,0.9);}{%
}
\end{mapleinput}

\mapleresult
\begin{maplelatex}
\[
.59049
\]
\end{maplelatex}

\end{maplegroup}
\begin{maplegroup}
\begin{mapleinput}
\mapleinline{active}{1d}{plot([seq([n,norminf(f,n,0,1)],n=1..20)],
x=0..20);}{%
}
\end{mapleinput}

\mapleresult
\begin{center}
\mapleplot{TP401.eps}
\end{center}

\end{maplegroup}
\begin{maplegroup}
\begin{mapleinput}
\mapleinline{active}{1d}{plot([seq(f(x,n),n=1..20)], x=0..1,
color=black);}{%
}
\end{mapleinput}

\mapleresult
\begin{center}
\mapleplot{TP402.eps}
\end{center}

\end{maplegroup}
\begin{maplegroup}
\begin{mapleinput}
\mapleinline{active}{1d}{with(plots):}{%
}
\end{mapleinput}

\end{maplegroup}
\begin{maplegroup}
\begin{mapleinput}
\mapleinline{active}{1d}{animate(f(x,n), x=0..1, n=1..40,
frames=40);}{%
}
\end{mapleinput}

\mapleresult
\begin{center}
\end{center}

\end{maplegroup}
\begin{maplegroup}
\begin{Bullet Item}
Il y a convergence simple sur [0,1] vers la fonction f(x)=0 si
x\TEXTsymbol{<}\TEXTsymbol{>}1 et f(1)=1.
\end{Bullet Item}

\begin{Bullet Item}
La fonction limite f est discontinue et il n'y a pas convergence
uniforme sur [0,1].
\end{Bullet Item}

\begin{Bullet Item}
Il y a convergence uniforme sur les compacts de [0,1[.
\end{Bullet Item}

\end{maplegroup}
\begin{maplegroup}
\begin{mapleinput}
\mapleinline{active}{1d}{maximize(f(x,5),x,\{x=0..0.9\});}{%
}
\end{mapleinput}

\mapleresult
\begin{maplelatex}
\[
.59049
\]
\end{maplelatex}

\end{maplegroup}
\begin{maplegroup}
\begin{center}
\textbf{Deuxi�me suite de fonctions}
\end{center}

\end{maplegroup}
\begin{maplegroup}
\begin{mapleinput}
\mapleinline{active}{1d}{restart;}{%
}
\end{mapleinput}

\end{maplegroup}
\begin{maplegroup}
\begin{mapleinput}
\mapleinline{active}{1d}{with(plots):}{%
}
\end{mapleinput}

\end{maplegroup}
\begin{maplegroup}
\begin{mapleinput}
\mapleinline{active}{1d}{g := (x,n)->n*x*exp(-n*x);}{%
}
\end{mapleinput}

\mapleresult
\begin{maplelatex}
\[
g := (x, \,n)\rightarrow n\,x\,e^{( - n\,x)}
\]
\end{maplelatex}

\end{maplegroup}
\begin{maplegroup}
\begin{mapleinput}
\mapleinline{active}{1d}{plot([seq(g(x,n),n=1..20)], x=0..1,
color=black);}{%
}
\end{mapleinput}

\mapleresult
\begin{center}
\mapleplot{TP404.eps}
\end{center}

\end{maplegroup}
\begin{maplegroup}
\begin{mapleinput}
\mapleinline{active}{1d}{animate(g(x,n), x=0..1, n=1..40, color=black,
frames=40, numpoints=100);}{%
}
\end{mapleinput}

\mapleresult
\begin{center}
\end{center}

\end{maplegroup}
\begin{maplegroup}
C'est une "bosse glissante".

\begin{Bullet Item}
Il y a convergence simple sur [0,1] vers la fonction f(x)=0.
\end{Bullet Item}

\begin{Bullet Item}
La fonction limite f est continue mais il n'y a pas convergence
uniforme sur [0,1].
\end{Bullet Item}

\begin{Bullet Item}
Il y a convergence uniforme sur les compacts de ]0,1].
\end{Bullet Item}

\end{maplegroup}
\begin{maplegroup}
\begin{mapleinput}
\end{mapleinput}

\end{maplegroup}
\begin{maplegroup}
\begin{center}
\textbf{Troisi�me suite de fonctions}
\end{center}

\end{maplegroup}
\begin{maplegroup}
\begin{mapleinput}
\mapleinline{active}{1d}{restart;}{%
}
\end{mapleinput}

\end{maplegroup}
\begin{maplegroup}
\begin{mapleinput}
\mapleinline{active}{1d}{with(plots):}{%
}
\end{mapleinput}

\end{maplegroup}
\begin{maplegroup}
\begin{mapleinput}
\mapleinline{active}{1d}{h := (x,n)->(1+x/n)^n;}{%
}
\end{mapleinput}

\mapleresult
\begin{maplelatex}
\[
h := (x, \,n)\rightarrow (1 + {\displaystyle \frac {x}{n}} )^{n}
\]
\end{maplelatex}

\end{maplegroup}
\begin{maplegroup}
\begin{mapleinput}
\mapleinline{active}{1d}{plot([seq(h(x,n),n=1..20)], x=-5..5,
color=black);}{%
}
\end{mapleinput}

\mapleresult
\begin{center}
\mapleplot{TP406.eps}
\end{center}

\end{maplegroup}
\begin{maplegroup}
\begin{mapleinput}
\mapleinline{active}{1d}{animate(exp(x)-h(x,n), x=-5..5, n=1..40,
color=black, frames=40);}{%
}
\end{mapleinput}

\mapleresult
\begin{center}
\end{center}

\end{maplegroup}
\begin{maplegroup}
\begin{Bullet Item}
Il y a convergence simple sur \textbf{R} vers la fonction
exponentielle.
\end{Bullet Item}

\begin{Bullet Item}
La fonction limite f est continue mais il n'y a pas convergence
uniforme sur \textbf{R} (on consid�re la suite de points 
\mapleinline{inert}{2d}{x[n]=n}{%
${x_{n}}=n$%
}).
\end{Bullet Item}

\begin{Bullet Item}
Il y a convergence uniforme sur les compacts de \textbf{R}.
\end{Bullet Item}

\end{maplegroup}
\begin{maplegroup}
\begin{mapleinput}
\end{mapleinput}

\end{maplegroup}

\section{Exercice 2}

Quelques exemples de trac�s de fonctions p�riodiques.

\begin{maplegroup}
\begin{mapleinput}
\mapleinline{active}{1d}{restart;}{%
}
\end{mapleinput}

\end{maplegroup}
\begin{maplegroup}
\begin{mapleinput}
\mapleinline{active}{1d}{f := x->(-1)^floor(x);}{%
}
\end{mapleinput}

\mapleresult
\begin{maplelatex}
\[
f := x\rightarrow (-1)^{{\rm floor}(x)}
\]
\end{maplelatex}

\end{maplegroup}
\begin{maplegroup}
\begin{mapleinput}
\mapleinline{active}{1d}{plot(f(x),x=-4..4, discont=true,
color=black);}{%
}
\end{mapleinput}

\mapleresult
\begin{center}
\mapleplot{TP408.eps}
\end{center}

\end{maplegroup}
\begin{maplegroup}
Pour une p�riode de 4 :

\end{maplegroup}
\begin{maplegroup}
\begin{mapleinput}
\mapleinline{active}{1d}{plot(f(x/2),x=-4..4, discont=true,
color=black);}{%
}
\end{mapleinput}

\mapleresult
\begin{center}
\mapleplot{TP409.eps}
\end{center}

\end{maplegroup}
\begin{maplegroup}
\begin{mapleinput}
\mapleinline{active}{1d}{g := x->arctan(tan(x));}{%
}
\end{mapleinput}

\mapleresult
\begin{maplelatex}
\[
g := x\rightarrow {\rm arctan}({\rm tan}(x))
\]
\end{maplelatex}

\end{maplegroup}
\begin{maplegroup}
\begin{mapleinput}
\mapleinline{active}{1d}{plot(g(x),x=-2*Pi..2*Pi, discont=true,
color=black);}{%
}
\end{mapleinput}

\mapleresult
\begin{center}
\mapleplot{TP410.eps}
\end{center}

\end{maplegroup}
\begin{maplegroup}
\begin{mapleinput}
\mapleinline{active}{1d}{h := x->arccos(cos(x));}{%
}
\end{mapleinput}

\mapleresult
\begin{maplelatex}
\[
h := x\rightarrow {\rm arccos}({\rm cos}(x))
\]
\end{maplelatex}

\end{maplegroup}
\begin{maplegroup}
\begin{mapleinput}
\mapleinline{active}{1d}{plot(h(x),x=-4*Pi..4*Pi, discont=true,
color=black);}{%
}
\end{mapleinput}

\mapleresult
\begin{center}
\mapleplot{TP411.eps}
\end{center}

\end{maplegroup}
\begin{maplegroup}
\begin{mapleinput}
\mapleinline{active}{1d}{i := x->arcsin(sin(x));}{%
}
\end{mapleinput}

\mapleresult
\begin{maplelatex}
\[
i := x\rightarrow {\rm arcsin}({\rm sin}(x))
\]
\end{maplelatex}

\end{maplegroup}
\begin{maplegroup}
\begin{mapleinput}
\mapleinline{active}{1d}{plot(i(x),x=-4*Pi..4*Pi, discont=true,
color=black);}{%
}
\end{mapleinput}

\mapleresult
\begin{center}
\mapleplot{TP412.eps}
\end{center}

\end{maplegroup}
\begin{maplegroup}
\begin{mapleinput}
\end{mapleinput}

\end{maplegroup}

\section{Exercice 3}

Coefficients et s�ries de Fourier (en r�els et complexes).

\begin{maplegroup}
\begin{mapleinput}
\mapleinline{active}{1d}{restart;}{%
}
\end{mapleinput}

\end{maplegroup}
\begin{maplegroup}
\begin{Heading 1}
En complexes
\end{Heading 1}

\end{maplegroup}
\begin{maplegroup}
\begin{mapleinput}
\mapleinline{active}{1d}{c := (f,n,a,b) -> 1/(b-a) * int(f(t) *
exp(-I*n*t*2*Pi/(b-a)), t=a..b);}{%
}
\end{mapleinput}

\mapleresult
\begin{maplelatex}
\[
c := (f, \,n, \,a, \,b)\rightarrow {\displaystyle \frac {
{\displaystyle \int _{a}^{b}} f(t)\,e^{( - 2\,\frac {I\,n\,t\,\pi
 }{b - a})}\,dt}{b - a}} 
\]
\end{maplelatex}

\end{maplegroup}
\begin{maplegroup}
\begin{mapleinput}
\mapleinline{active}{1d}{e:=x->exp(I*x);}{%
}
\end{mapleinput}

\mapleresult
\begin{maplelatex}
\[
e := x\rightarrow e^{(I\,x)}
\]
\end{maplelatex}

\end{maplegroup}
\begin{maplegroup}
\begin{mapleinput}
\mapleinline{active}{1d}{c(e,0,0,2*Pi);}{%
}
\end{mapleinput}

\mapleresult
\begin{maplelatex}
\[
0
\]
\end{maplelatex}

\end{maplegroup}
\begin{maplegroup}
\begin{mapleinput}
\mapleinline{active}{1d}{c(e,1,0,2*Pi);}{%
}
\end{mapleinput}

\mapleresult
\begin{maplelatex}
\[
1
\]
\end{maplelatex}

\end{maplegroup}
\begin{maplegroup}
\begin{mapleinput}
\mapleinline{active}{1d}{c(e,2,0,2*Pi);}{%
}
\end{mapleinput}

\mapleresult
\begin{maplelatex}
\[
0
\]
\end{maplelatex}

\end{maplegroup}
\begin{maplegroup}
ok

\end{maplegroup}
\begin{maplegroup}
\begin{mapleinput}
\mapleinline{active}{1d}{fourierC := (f,n,a,b,x) -> sum(c(f,k,a,b) *
exp(I*k*x*2*Pi/(b-a)), k=-n..n);}{%
}
\end{mapleinput}

\mapleresult
\begin{maplelatex}
\[
{\it fourierC} := (f, \,n, \,a, \,b, \,x)\rightarrow 
{\displaystyle \sum _{k= - n}^{n}} \,{\rm c}(f, \,k, \,a, \,b)\,e
^{(2\,\frac {I\,k\,x\,\pi }{b - a})}
\]
\end{maplelatex}

\end{maplegroup}
\begin{maplegroup}
\begin{mapleinput}
\mapleinline{active}{1d}{fourierC(e,5,0,2*Pi,x);}{%
}
\end{mapleinput}

\mapleresult
\begin{maplettyout}
Error, (in sum) division by zero
\end{maplettyout}

\end{maplegroup}
\begin{maplegroup}
\begin{mapleinput}
\mapleinline{active}{1d}{normal(c(e,k,0,2*Pi));}{%
}
\end{mapleinput}

\mapleresult
\begin{maplelatex}
\[
{\displaystyle \frac {1}{2}} \,{\displaystyle \frac {I\,(e^{( - 2
\,I\,\pi \,k)} - 1)}{\pi \,( - 1 + k)}} 
\]
\end{maplelatex}

\end{maplegroup}
\begin{maplegroup}
Maple donne une formule g�n�rale pour c(f,k,0,2*Pi) qui est fausse
pour k=1 (on ne pouvait pas prendre une primitive...).

Voici comment rem�dier � ce probl�me :

\end{maplegroup}
\begin{maplegroup}
\begin{mapleinput}
\mapleinline{active}{1d}{fourierC2 := proc(f,n,a,b,x)
local S,k;
S := 0;
for k from -n to n do
    S := S + c(f,k,a,b) * exp(I*k*x*2*Pi/(b-a));
od;
S;
end:}{%
}
\end{mapleinput}

\end{maplegroup}
\begin{maplegroup}
\begin{mapleinput}
\mapleinline{active}{1d}{fourierC2(e,5,0,2*Pi,x);}{%
}
\end{mapleinput}

\mapleresult
\begin{maplelatex}
\[
e^{(I\,x)}
\]
\end{maplelatex}

\end{maplegroup}
\begin{maplegroup}
\begin{mapleinput}
\mapleinline{active}{1d}{e(x);}{%
}
\end{mapleinput}

\mapleresult
\begin{maplelatex}
\[
e^{(I\,x)}
\]
\end{maplelatex}

\end{maplegroup}
\begin{maplegroup}
ok

\end{maplegroup}
\begin{maplegroup}
\begin{Heading 1}
En r�els
\end{Heading 1}

\end{maplegroup}
\begin{maplegroup}
\begin{mapleinput}
\mapleinline{active}{1d}{A := (f,n,a,b) -> 1/(b-a) * int(f(t) *
cos(n*t*2*Pi/(b-a)), t=a..b);}{%
}
\end{mapleinput}

\mapleresult
\begin{maplelatex}
\[
A := (f, \,n, \,a, \,b)\rightarrow {\displaystyle \frac {
{\displaystyle \int _{a}^{b}} f(t)\,{\rm cos}(2\,{\displaystyle 
\frac {n\,t\,\pi }{b - a}} )\,dt}{b - a}} 
\]
\end{maplelatex}

\end{maplegroup}
\begin{maplegroup}
\begin{mapleinput}
\mapleinline{active}{1d}{B := (f,n,a,b) -> 1/(b-a) * int(f(t) *
sin(n*t*2*Pi/(b-a)), t=a..b);}{%
}
\end{mapleinput}

\mapleresult
\begin{maplelatex}
\[
B := (f, \,n, \,a, \,b)\rightarrow {\displaystyle \frac {
{\displaystyle \int _{a}^{b}} f(t)\,{\rm sin}(2\,{\displaystyle 
\frac {n\,t\,\pi }{b - a}} )\,dt}{b - a}} 
\]
\end{maplelatex}

\end{maplegroup}
\begin{maplegroup}
\begin{mapleinput}
\mapleinline{active}{1d}{A(cos,0,0,2*Pi);}{%
}
\end{mapleinput}

\mapleresult
\begin{maplelatex}
\[
0
\]
\end{maplelatex}

\end{maplegroup}
\begin{maplegroup}
\begin{mapleinput}
\mapleinline{active}{1d}{A(cos,1,0,2*Pi);}{%
}
\end{mapleinput}

\mapleresult
\begin{maplelatex}
\[
{\displaystyle \frac {1}{2}} 
\]
\end{maplelatex}

\end{maplegroup}
\begin{maplegroup}
\begin{mapleinput}
\mapleinline{active}{1d}{A(cos,2,0,2*Pi);}{%
}
\end{mapleinput}

\mapleresult
\begin{maplelatex}
\[
0
\]
\end{maplelatex}

\end{maplegroup}
\begin{maplegroup}
\begin{mapleinput}
\mapleinline{active}{1d}{B(cos,0,0,2*Pi);}{%
}
\end{mapleinput}

\mapleresult
\begin{maplelatex}
\[
0
\]
\end{maplelatex}

\end{maplegroup}
\begin{maplegroup}
ok

\end{maplegroup}
\begin{maplegroup}
\begin{mapleinput}
\mapleinline{active}{1d}{fourierR := (f,n,a,b,x) -> A(f,0,a,b) +
2*sum(A(f,k,a,b) * cos(k*x*2*Pi/(b-a)) + B(f,k,a,b) *
sin(k*x*2*Pi/(b-a)), k=1..n);}{%
}
\end{mapleinput}

\mapleresult
\begin{maplelatex}
\begin{eqnarray*}
\lefteqn{{\it fourierR} := (f, \,n, \,a, \,b, \,x)\rightarrow }
 \\
 & & {\rm A}(f, \,0, \,a, \,b) + 2\, \left(  \! {\displaystyle 
\sum _{k=1}^{n}} \,({\rm A}(f, \,k, \,a, \,b)\,{\rm cos}(2\,
{\displaystyle \frac {k\,x\,\pi }{b - a}} ) + {\rm B}(f, \,k, \,a
, \,b)\,{\rm sin}(2\,{\displaystyle \frac {k\,x\,\pi }{b - a}} ))
 \!  \right) 
\end{eqnarray*}
\end{maplelatex}

\end{maplegroup}
\begin{maplegroup}
\begin{mapleinput}
\mapleinline{active}{1d}{fourierR(cos,5,0,2*Pi,x);}{%
}
\end{mapleinput}

\mapleresult
\begin{maplettyout}
Error, (in sum) division by zero
\end{maplettyout}

\end{maplegroup}
\begin{maplegroup}
m�me probl�me qu'en complexes :

\end{maplegroup}
\begin{maplegroup}
\begin{mapleinput}
\mapleinline{active}{1d}{normal(A(cos,k,0,2*Pi));}{%
}
\end{mapleinput}

\mapleresult
\begin{maplelatex}
\[
{\displaystyle \frac {k\,{\rm sin}(\pi \,k)\,{\rm cos}(\pi \,k)}{
\pi \,( - 1 + k)\,(1 + k)}} 
\]
\end{maplelatex}

\end{maplegroup}
\begin{maplegroup}
\begin{mapleinput}
\mapleinline{active}{1d}{fourierR2 := proc(f,n,a,b,x)
local S,k;
S := A(f,0,a,b);
for k from 1 to n do
    S := S + 2*(A(f,k,a,b) * cos(k*x*2*Pi/(b-a)) + B(f,k,a,b) *
sin(k*x*2*Pi/(b-a)));
od;
S;
end:}{%
}
\end{mapleinput}

\end{maplegroup}
\begin{maplegroup}
\begin{mapleinput}
\mapleinline{active}{1d}{fourierR2(cos,5,0,2*Pi,x);}{%
}
\end{mapleinput}

\mapleresult
\begin{maplelatex}
\[
{\rm cos}(x)
\]
\end{maplelatex}

\end{maplegroup}
\begin{maplegroup}
\begin{mapleinput}
\mapleinline{active}{1d}{fourierR2(sin,5,0,2*Pi,x);}{%
}
\end{mapleinput}

\mapleresult
\begin{maplelatex}
\[
{\rm sin}(x)
\]
\end{maplelatex}

\end{maplegroup}
\begin{maplegroup}
\begin{mapleinput}
\mapleinline{active}{1d}{fourierR2(x->cos(2*x),5,0,2*Pi,x);}{%
}
\end{mapleinput}

\mapleresult
\begin{maplelatex}
\[
{\rm cos}(2\,x)
\]
\end{maplelatex}

\end{maplegroup}
\begin{maplegroup}
\begin{mapleinput}
\mapleinline{active}{1d}{fourierR2(x->cos(x)^2,5,0,2*Pi,x);}{%
}
\end{mapleinput}

\mapleresult
\begin{maplelatex}
\[
{\displaystyle \frac {1}{2}}  + {\displaystyle \frac {1}{2}} \,
{\rm cos}(2\,x)
\]
\end{maplelatex}

\end{maplegroup}
\begin{maplegroup}
\begin{mapleinput}
\mapleinline{active}{1d}{fourierR2(x->cos(x)^5,5,0,2*Pi,x);}{%
}
\end{mapleinput}

\mapleresult
\begin{maplelatex}
\[
{\displaystyle \frac {5}{8}} \,{\rm cos}(x) + {\displaystyle 
\frac {5}{16}} \,{\rm cos}(3\,x) + {\displaystyle \frac {1}{16}} 
\,{\rm cos}(5\,x)
\]
\end{maplelatex}

\end{maplegroup}
\begin{maplegroup}
\begin{mapleinput}
\mapleinline{active}{1d}{fourierR2(x->cos(7*x),5,0,2*Pi,x);}{%
}
\end{mapleinput}

\mapleresult
\begin{maplelatex}
\[
0
\]
\end{maplelatex}

\end{maplegroup}
\begin{maplegroup}
H� oui !

\end{maplegroup}
\begin{maplegroup}
\begin{mapleinput}
\end{mapleinput}

\end{maplegroup}

\section{Exercice 4}

Calculs et trac�s de s�ries de Fourier de fonction p�riodiques

\begin{maplegroup}
\begin{Heading 1}
Fonction f
\end{Heading 1}

\end{maplegroup}
\begin{maplegroup}
\begin{mapleinput}
\mapleinline{active}{1d}{f := x->(-1)^floor(x);}{%
}
\end{mapleinput}

\mapleresult
\begin{maplelatex}
\[
f := x\rightarrow (-1)^{{\rm floor}(x)}
\]
\end{maplelatex}

\end{maplegroup}
\begin{maplegroup}
\begin{mapleinput}
\mapleinline{active}{1d}{fourierC(f,1,0,2,x);}{%
}
\end{mapleinput}

\mapleresult
\begin{maplettyout}
Error, (in sum) division by zero
\end{maplettyout}

\end{maplegroup}
\begin{maplegroup}
\begin{mapleinput}
\mapleinline{active}{1d}{fourierC2(f,1,0,2,x);}{%
}
\end{mapleinput}

\mapleresult
\begin{maplelatex}
\[
2\,{\displaystyle \frac {I\,e^{( - I\,x\,\pi )}}{\pi }}  - 2\,
{\displaystyle \frac {I\,e^{(I\,x\,\pi )}}{\pi }} 
\]
\end{maplelatex}

\end{maplegroup}
\begin{maplegroup}
\begin{mapleinput}
\mapleinline{active}{1d}{f1C := evalc(");}{%
}
\end{mapleinput}

\mapleresult
\begin{maplelatex}
\[
{\it f1C} := 4\,{\displaystyle \frac {{\rm sin}(x\,\pi )}{\pi }} 
\]
\end{maplelatex}

\end{maplegroup}
\begin{maplegroup}
\begin{mapleinput}
\mapleinline{active}{1d}{plot([f(x),f1C], x=-2..2, color=black,
discont=true);}{%
}
\end{mapleinput}

\mapleresult
\begin{center}
\mapleplot{TP413.eps}
\end{center}

\end{maplegroup}
\begin{maplegroup}
\begin{mapleinput}
\mapleinline{active}{1d}{f3C := evalc(fourierC2(f,3,0,2,x));}{%
}
\end{mapleinput}

\mapleresult
\begin{maplelatex}
\[
{\it f3C} := {\displaystyle \frac {4}{3}} \,{\displaystyle 
\frac {{\rm sin}(3\,x\,\pi )}{\pi }}  + 4\,{\displaystyle \frac {
{\rm sin}(x\,\pi )}{\pi }} 
\]
\end{maplelatex}

\end{maplegroup}
\begin{maplegroup}
\begin{mapleinput}
\mapleinline{active}{1d}{plot([f(x),f3C], x=-2..2, color=black,
discont=true);}{%
}
\end{mapleinput}

\mapleresult
\begin{center}
\mapleplot{TP414.eps}
\end{center}

\end{maplegroup}
\begin{maplegroup}
\begin{mapleinput}
\mapleinline{active}{1d}{f5C := evalc(fourierC2(f,5,0,2,x));}{%
}
\end{mapleinput}

\mapleresult
\begin{maplelatex}
\[
{\it f5C} := {\displaystyle \frac {4}{5}} \,{\displaystyle 
\frac {{\rm sin}(5\,x\,\pi )}{\pi }}  + {\displaystyle \frac {4}{
3}} \,{\displaystyle \frac {{\rm sin}(3\,x\,\pi )}{\pi }}  + 4\,
{\displaystyle \frac {{\rm sin}(x\,\pi )}{\pi }} 
\]
\end{maplelatex}

\end{maplegroup}
\begin{maplegroup}
\begin{mapleinput}
\mapleinline{active}{1d}{plot([f(x),f5C], x=-2..2, color=black,
discont=true);}{%
}
\end{mapleinput}

\mapleresult
\begin{center}
\mapleplot{TP415.eps}
\end{center}

\end{maplegroup}
\begin{maplegroup}
\begin{mapleinput}
\mapleinline{active}{1d}{f7C := evalc(fourierC2(f,7,0,2,x));}{%
}
\end{mapleinput}

\mapleresult
\begin{maplelatex}
\[
{\it f7C} := {\displaystyle \frac {4}{7}} \,{\displaystyle 
\frac {{\rm sin}(7\,x\,\pi )}{\pi }}  + {\displaystyle \frac {4}{
5}} \,{\displaystyle \frac {{\rm sin}(5\,x\,\pi )}{\pi }}  + 
{\displaystyle \frac {4}{3}} \,{\displaystyle \frac {{\rm sin}(3
\,x\,\pi )}{\pi }}  + 4\,{\displaystyle \frac {{\rm sin}(x\,\pi )
}{\pi }} 
\]
\end{maplelatex}

\end{maplegroup}
\begin{maplegroup}
\begin{mapleinput}
\mapleinline{active}{1d}{plot([f(x),f7C], x=-2..2, color=black,
discont=true);}{%
}
\end{mapleinput}

\mapleresult
\begin{center}
\mapleplot{TP416.eps}
\end{center}

\end{maplegroup}
\begin{maplegroup}
\begin{mapleinput}
\mapleinline{active}{1d}{f13C := evalc(fourierC2(f,13,0,2,x));}{%
}
\end{mapleinput}

\mapleresult
\begin{maplelatex}
\begin{eqnarray*}
\lefteqn{{\it f13C} := {\displaystyle \frac {4}{13}} \,
{\displaystyle \frac {{\rm sin}(13\,x\,\pi )}{\pi }}  + 
{\displaystyle \frac {4}{11}} \,{\displaystyle \frac {{\rm sin}(
11\,x\,\pi )}{\pi }}  + {\displaystyle \frac {4}{9}} \,
{\displaystyle \frac {{\rm sin}(9\,x\,\pi )}{\pi }}  + 
{\displaystyle \frac {4}{7}} \,{\displaystyle \frac {{\rm sin}(7
\,x\,\pi )}{\pi }}  + {\displaystyle \frac {4}{5}} \,
{\displaystyle \frac {{\rm sin}(5\,x\,\pi )}{\pi }} } \\
 & & \mbox{} + {\displaystyle \frac {4}{3}} \,{\displaystyle 
\frac {{\rm sin}(3\,x\,\pi )}{\pi }}  + 4\,{\displaystyle \frac {
{\rm sin}(x\,\pi )}{\pi }} \mbox{\hspace{271pt}}
\end{eqnarray*}
\end{maplelatex}

\end{maplegroup}
\begin{maplegroup}
\begin{mapleinput}
\mapleinline{active}{1d}{plot([f(x),f13C], x=-2..2, color=black,
discont=true);}{%
}
\end{mapleinput}

\mapleresult
\begin{center}
\mapleplot{TP417.eps}
\end{center}

\end{maplegroup}
\begin{maplegroup}
\begin{mapleinput}
\mapleinline{active}{1d}{f21C := evalc(fourierC2(f,21,0,2,x));}{%
}
\end{mapleinput}

\mapleresult
\begin{maplelatex}
\begin{eqnarray*}
\lefteqn{{\it f21C} := {\displaystyle \frac {4}{21}} \,
{\displaystyle \frac {{\rm sin}(21\,x\,\pi )}{\pi }}  + 
{\displaystyle \frac {4}{19}} \,{\displaystyle \frac {{\rm sin}(
19\,x\,\pi )}{\pi }}  + {\displaystyle \frac {4}{17}} \,
{\displaystyle \frac {{\rm sin}(17\,x\,\pi )}{\pi }}  + 
{\displaystyle \frac {4}{15}} \,{\displaystyle \frac {{\rm sin}(
15\,x\,\pi )}{\pi }} } \\
 & & \mbox{} + {\displaystyle \frac {4}{13}} \,{\displaystyle 
\frac {{\rm sin}(13\,x\,\pi )}{\pi }}  + {\displaystyle \frac {4
}{11}} \,{\displaystyle \frac {{\rm sin}(11\,x\,\pi )}{\pi }}  + 
{\displaystyle \frac {4}{9}} \,{\displaystyle \frac {{\rm sin}(9
\,x\,\pi )}{\pi }}  + {\displaystyle \frac {4}{7}} \,
{\displaystyle \frac {{\rm sin}(7\,x\,\pi )}{\pi }}  + 
{\displaystyle \frac {4}{5}} \,{\displaystyle \frac {{\rm sin}(5
\,x\,\pi )}{\pi }}  \\
 & & \mbox{} + {\displaystyle \frac {4}{3}} \,{\displaystyle 
\frac {{\rm sin}(3\,x\,\pi )}{\pi }}  + 4\,{\displaystyle \frac {
{\rm sin}(x\,\pi )}{\pi }} 
\end{eqnarray*}
\end{maplelatex}

\end{maplegroup}
\begin{maplegroup}
\begin{mapleinput}
\mapleinline{active}{1d}{plot([f(x),f21C], x=-2..2, color=black,
discont=true, numpoints=300);}{%
}
\end{mapleinput}

\mapleresult
\begin{center}
\mapleplot{TP418.eps}
\end{center}

\end{maplegroup}
\begin{maplegroup}
\begin{center}
\textbf{On observe le c�l�bre ph�nom�ne de Gibbs.}
\end{center}

Il n'y a pas convergence uniforme sur \textbf{R}.

\end{maplegroup}
\begin{maplegroup}
\begin{mapleinput}
\end{mapleinput}

\end{maplegroup}
\begin{maplegroup}
\begin{Heading 1}
Fonction g
\end{Heading 1}

\end{maplegroup}
\begin{maplegroup}
\begin{mapleinput}
\mapleinline{active}{1d}{g := x->arctan(tan(x));}{%
}
\end{mapleinput}

\mapleresult
\begin{maplelatex}
\[
g := x\rightarrow {\rm arctan}({\rm tan}(x))
\]
\end{maplelatex}

\end{maplegroup}
\begin{maplegroup}
g co�ncide avec l'identit� sur  
\mapleinline{inert}{2d}{[-Pi/2,Pi/2]}{%
$[ - \frac {\pi }{2}, \,\frac {\pi }{2}]$%
} :

\end{maplegroup}
\begin{maplegroup}
\begin{mapleinput}
\mapleinline{active}{1d}{i := x->x;}{%
}
\end{mapleinput}

\mapleresult
\begin{maplelatex}
\[
i := x\rightarrow x
\]
\end{maplelatex}

\end{maplegroup}
\begin{maplegroup}
\begin{mapleinput}
\mapleinline{active}{1d}{g1C := evalc(fourierC2(i,1,-Pi/2,Pi/2,x));}{%
}
\end{mapleinput}

\mapleresult
\begin{maplelatex}
\[
{\it g1C} := {\rm sin}(2\,x)
\]
\end{maplelatex}

\end{maplegroup}
\begin{maplegroup}
\begin{mapleinput}
\mapleinline{active}{1d}{plot([g(x),g1C], x=-Pi..Pi, color=black,
discont=true);}{%
}
\end{mapleinput}

\mapleresult
\begin{center}
\mapleplot{TP419.eps}
\end{center}

\end{maplegroup}
\begin{maplegroup}
\begin{mapleinput}
\mapleinline{active}{1d}{g2C := evalc(fourierC2(i,2,-Pi/2,Pi/2,x));}{%
}
\end{mapleinput}

\mapleresult
\begin{maplelatex}
\[
{\it g2C} :=  - {\displaystyle \frac {1}{2}} \,{\rm sin}(4\,x) + 
{\rm sin}(2\,x)
\]
\end{maplelatex}

\end{maplegroup}
\begin{maplegroup}
\begin{mapleinput}
\mapleinline{active}{1d}{plot([g(x),g2C], x=-Pi..Pi, color=black,
discont=true);}{%
}
\end{mapleinput}

\mapleresult
\begin{center}
\mapleplot{TP420.eps}
\end{center}

\end{maplegroup}
\begin{maplegroup}
\begin{mapleinput}
\mapleinline{active}{1d}{g3C := evalc(fourierC2(i,3,-Pi/2,Pi/2,x));}{%
}
\end{mapleinput}

\mapleresult
\begin{maplelatex}
\[
{\it g3C} := {\displaystyle \frac {1}{3}} \,{\rm sin}(6\,x) - 
{\displaystyle \frac {1}{2}} \,{\rm sin}(4\,x) + {\rm sin}(2\,x)
\]
\end{maplelatex}

\end{maplegroup}
\begin{maplegroup}
\begin{mapleinput}
\mapleinline{active}{1d}{plot([g(x),g3C], x=-Pi..Pi, color=black,
discont=true);}{%
}
\end{mapleinput}

\mapleresult
\begin{center}
\mapleplot{TP421.eps}
\end{center}

\end{maplegroup}
\begin{maplegroup}
\begin{mapleinput}
\mapleinline{active}{1d}{g4C := evalc(fourierC2(i,4,-Pi/2,Pi/2,x));}{%
}
\end{mapleinput}

\mapleresult
\begin{maplelatex}
\[
{\it g4C} :=  - {\displaystyle \frac {1}{4}} \,{\rm sin}(8\,x) + 
{\displaystyle \frac {1}{3}} \,{\rm sin}(6\,x) - {\displaystyle 
\frac {1}{2}} \,{\rm sin}(4\,x) + {\rm sin}(2\,x)
\]
\end{maplelatex}

\end{maplegroup}
\begin{maplegroup}
\begin{mapleinput}
\mapleinline{active}{1d}{plot([g(x),g4C], x=-Pi..Pi, color=black,
discont=true);}{%
}
\end{mapleinput}

\mapleresult
\begin{center}
\mapleplot{TP422.eps}
\end{center}

\end{maplegroup}
\begin{maplegroup}
\begin{mapleinput}
\mapleinline{active}{1d}{g10C :=
evalc(fourierC2(i,10,-Pi/2,Pi/2,x));}{%
}
\end{mapleinput}

\mapleresult
\begin{maplelatex}
\begin{eqnarray*}
\lefteqn{{\it g10C} :=  - {\displaystyle \frac {1}{10}} \,{\rm 
sin}(20\,x) + {\displaystyle \frac {1}{9}} \,{\rm sin}(18\,x) - 
{\displaystyle \frac {1}{8}} \,{\rm sin}(16\,x) + {\displaystyle 
\frac {1}{7}} \,{\rm sin}(14\,x) - {\displaystyle \frac {1}{6}} 
\,{\rm sin}(12\,x) + {\displaystyle \frac {1}{5}} \,{\rm sin}(10
\,x)} \\
 & & \mbox{} - {\displaystyle \frac {1}{4}} \,{\rm sin}(8\,x) + 
{\displaystyle \frac {1}{3}} \,{\rm sin}(6\,x) - {\displaystyle 
\frac {1}{2}} \,{\rm sin}(4\,x) + {\rm sin}(2\,x)
\mbox{\hspace{200pt}}
\end{eqnarray*}
\end{maplelatex}

\end{maplegroup}
\begin{maplegroup}
\begin{mapleinput}
\mapleinline{active}{1d}{plot([g(x),g10C], x=-Pi..Pi, color=black,
discont=true);}{%
}
\end{mapleinput}

\mapleresult
\begin{center}
\mapleplot{TP423.eps}
\end{center}

\end{maplegroup}
\begin{maplegroup}
\begin{mapleinput}
\mapleinline{active}{1d}{g20C :=
evalc(fourierC2(i,20,-Pi/2,Pi/2,x));}{%
}
\end{mapleinput}

\mapleresult
\begin{maplelatex}
\begin{eqnarray*}
\lefteqn{{\it g20C} :=  - {\displaystyle \frac {1}{16}} \,{\rm 
sin}(32\,x) - {\displaystyle \frac {1}{12}} \,{\rm sin}(24\,x) + 
{\displaystyle \frac {1}{3}} \,{\rm sin}(6\,x) - {\displaystyle 
\frac {1}{4}} \,{\rm sin}(8\,x) - {\displaystyle \frac {1}{2}} \,
{\rm sin}(4\,x) - {\displaystyle \frac {1}{10}} \,{\rm sin}(20\,x
)} \\
 & & \mbox{} + {\displaystyle \frac {1}{17}} \,{\rm sin}(34\,x)
 + {\rm sin}(2\,x) + {\displaystyle \frac {1}{19}} \,{\rm sin}(38
\,x) + {\displaystyle \frac {1}{11}} \,{\rm sin}(22\,x) - 
{\displaystyle \frac {1}{18}} \,{\rm sin}(36\,x) - 
{\displaystyle \frac {1}{20}} \,{\rm sin}(40\,x) \\
 & & \mbox{} - {\displaystyle \frac {1}{14}} \,{\rm sin}(28\,x)
 + {\displaystyle \frac {1}{15}} \,{\rm sin}(30\,x) + 
{\displaystyle \frac {1}{5}} \,{\rm sin}(10\,x) + {\displaystyle 
\frac {1}{13}} \,{\rm sin}(26\,x) - {\displaystyle \frac {1}{8}} 
\,{\rm sin}(16\,x) + {\displaystyle \frac {1}{7}} \,{\rm sin}(14
\,x) \\
 & & \mbox{} - {\displaystyle \frac {1}{6}} \,{\rm sin}(12\,x) + 
{\displaystyle \frac {1}{9}} \,{\rm sin}(18\,x)
\mbox{\hspace{289pt}}
\end{eqnarray*}
\end{maplelatex}

\end{maplegroup}
\begin{maplegroup}
\begin{mapleinput}
\mapleinline{active}{1d}{plot([g(x),g20C], x=-Pi..Pi, color=black,
discont=true);}{%
}
\end{mapleinput}

\mapleresult
\begin{center}
\mapleplot{TP424.eps}
\end{center}

\end{maplegroup}
\begin{maplegroup}
\begin{center}
\textbf{Ici aussi, on observe le ph�nom�ne de Gibbs.}
\end{center}

Il n'y a pas convergence uniforme sur \textbf{R}.

\end{maplegroup}
\begin{maplegroup}
\begin{mapleinput}
\end{mapleinput}

\end{maplegroup}
\begin{maplegroup}
\begin{Heading 1}
Fonction h
\end{Heading 1}

\end{maplegroup}
\begin{maplegroup}
\begin{mapleinput}
\mapleinline{active}{1d}{h := x->arccos(cos(x));}{%
}
\end{mapleinput}

\mapleresult
\begin{maplelatex}
\[
h := x\rightarrow {\rm arccos}({\rm cos}(x))
\]
\end{maplelatex}

\end{maplegroup}
\begin{maplegroup}
\begin{mapleinput}
\mapleinline{active}{1d}{fourierC2(h,0,0,2*Pi,x);}{%
}
\end{mapleinput}

\begin{Maple Output}
\end{Maple Output}

\mapleresult
\begin{maplelatex}
\[
{\displaystyle \frac {1}{2}} \,{\displaystyle \frac {
{\displaystyle \int _{0}^{2\,\pi }} {\rm arccos}({\rm cos}(t))\,d
t}{\pi }} 
\]
\end{maplelatex}

\end{maplegroup}
\begin{maplegroup}
\begin{mapleinput}
\mapleinline{active}{1d}{evalc(");}{%
}
\end{mapleinput}

\mapleresult
\begin{maplettyout}
Error, (in evalc/int) Unable to handle definite integral
\end{maplettyout}

\end{maplegroup}
\begin{maplegroup}
Il faut utiliser les s�ries de Fourier en r�els, puis prendre des
valeurs approch�es avec evalf :

\end{maplegroup}
\begin{maplegroup}
\begin{mapleinput}
\mapleinline{active}{1d}{h10R := evalf(fourierR2(h,10,0,2*Pi,x));}{%
}
\end{mapleinput}

\mapleresult
\begin{maplelatex}
\begin{eqnarray*}
\lefteqn{{\it h10R} := 1.570796327 - 1.273239544\,{\rm cos}(x) - 
.6579099401\,10^{-16}\,{\rm cos}(2.\,x)} \\
 & & \mbox{} - .1414710605\,{\rm cos}(3.\,x) - .7237009346\,10^{
-15}\,{\rm cos}(4.\,x) - .05092958178\,{\rm cos}(5.\,x) \\
 & & \mbox{} - .7565964313\,10^{-15}\,{\rm cos}(6.\,x) - 
.02598448050\,{\rm cos}(7.\,x) + .3618504673\,10^{-15}\,{\rm cos}
(8.\,x) \\
 & & \mbox{} - .01571900672\,{\rm cos}(9.\,x) - .7565964313\,10^{
-15}\,{\rm cos}(10.\,x)
\end{eqnarray*}
\end{maplelatex}

\end{maplegroup}
\begin{maplegroup}
\begin{mapleinput}
\mapleinline{active}{1d}{h0R := op(1,h10R):}{%
}
\end{mapleinput}

\end{maplegroup}
\begin{maplegroup}
\begin{mapleinput}
\mapleinline{active}{1d}{h1R := op(1,h10R) + op(2,h10R):}{%
}
\end{mapleinput}

\end{maplegroup}
\begin{maplegroup}
\begin{mapleinput}
\mapleinline{active}{1d}{h4R := op(1,h10R) + op(2,h10R) + op(3,h10R) +
op(4,h10R) + op(5,h10R):}{%
}
\end{mapleinput}

\end{maplegroup}
\begin{maplegroup}
\begin{mapleinput}
\mapleinline{active}{1d}{plot([h(x),h0R], x=-4*Pi..4*Pi,
color=black);}{%
}
\end{mapleinput}

\mapleresult
\begin{center}
\mapleplot{TP425.eps}
\end{center}

\end{maplegroup}
\begin{maplegroup}
\begin{mapleinput}
\mapleinline{active}{1d}{plot([h(x),h1R], x=-4*Pi..4*Pi,
color=black);}{%
}
\end{mapleinput}

\mapleresult
\begin{center}
\mapleplot{TP426.eps}
\end{center}

\end{maplegroup}
\begin{maplegroup}
\begin{mapleinput}
\mapleinline{active}{1d}{plot([h(x),h4R], x=-4*Pi..4*Pi,
color=black);}{%
}
\end{mapleinput}

\mapleresult
\begin{center}
\mapleplot{TP427.eps}
\end{center}

\end{maplegroup}
\begin{maplegroup}
\begin{mapleinput}
\mapleinline{active}{1d}{plot([h(x),h10R], x=-4*Pi..4*Pi,
color=black);}{%
}
\end{mapleinput}

\mapleresult
\begin{center}
\mapleplot{TP428.eps}
\end{center}

\end{maplegroup}
\begin{maplegroup}
Ici, il y a convergence uniforme sur R.

\end{maplegroup}
\begin{maplegroup}
\begin{mapleinput}
\mapleinline{active}{1d}{sup}{%
}
\end{mapleinput}

\end{maplegroup}

\section{Exercice 5}

Sommes remarquables et �galit� de Parseval.

\begin{maplegroup}
\begin{mapleinput}
\mapleinline{active}{1d}{j := x->x*(Pi-x);}{%
}
\end{mapleinput}

\mapleresult
\begin{maplelatex}
\[
j := x\rightarrow x\,(\pi  - x)
\]
\end{maplelatex}

\end{maplegroup}
\begin{maplegroup}
\begin{mapleinput}
\mapleinline{active}{1d}{plot(j(x), x=0..Pi, color=black);}{%
}
\end{mapleinput}

\mapleresult
\begin{center}
\mapleplot{TP429.eps}
\end{center}

\end{maplegroup}
\begin{maplegroup}
\begin{mapleinput}
\mapleinline{active}{1d}{j7R := fourierR2(j,7,0,Pi,x);}{%
}
\end{mapleinput}

\mapleresult
\begin{maplelatex}
\begin{eqnarray*}
\lefteqn{{\it j7R} := {\displaystyle \frac {1}{6}} \,\pi ^{2} - 
{\rm cos}(2\,x) - {\displaystyle \frac {1}{4}} \,{\rm cos}(4\,x)
 - {\displaystyle \frac {1}{9}} \,{\rm cos}(6\,x) - 
{\displaystyle \frac {1}{16}} \,{\rm cos}(8\,x) - {\displaystyle 
\frac {1}{25}} \,{\rm cos}(10\,x)} \\
 & & \mbox{} - {\displaystyle \frac {1}{36}} \,{\rm cos}(12\,x)
 - {\displaystyle \frac {1}{49}} \,{\rm cos}(14\,x)
\mbox{\hspace{216pt}}
\end{eqnarray*}
\end{maplelatex}

\end{maplegroup}
\begin{maplegroup}
\begin{mapleinput}
\mapleinline{active}{1d}{plot([j(x),j7R], x=-2*Pi..2*Pi, y=-3..3,
color=black);}{%
}
\end{mapleinput}

\mapleresult
\begin{center}
\mapleplot{TP430.eps}
\end{center}

\end{maplegroup}
\begin{maplegroup}
\begin{mapleinput}
\mapleinline{active}{1d}{A(j,k,0,Pi);}{%
}
\end{mapleinput}

\mapleresult
\begin{maplelatex}
\[
{\displaystyle \frac { - {\displaystyle \frac {1}{4}} \,
{\displaystyle \frac {2\,\pi \,k\,{\rm cos}(\pi \,k)^{2} - \pi \,
k - 2\,{\rm sin}(\pi \,k)\,{\rm cos}(\pi \,k)}{k^{3}}}  - 
{\displaystyle \frac {1}{4}} \,{\displaystyle \frac {\pi }{k^{2}}
} }{\pi }} 
\]
\end{maplelatex}

\end{maplegroup}
\begin{maplegroup}
\begin{mapleinput}
\mapleinline{active}{1d}{assume(k, integer, k>0);}{%
}
\end{mapleinput}

\end{maplegroup}
\begin{maplegroup}
\begin{mapleinput}
\mapleinline{active}{1d}{A(j,k,0,Pi);}{%
}
\end{mapleinput}

\mapleresult
\begin{maplelatex}
\[
 - {\displaystyle \frac {1}{2}} \,{\displaystyle \frac {1}{{\it k
\symbol{126}}^{2}}} 
\]
\end{maplelatex}

\end{maplegroup}
\begin{maplegroup}
\begin{mapleinput}
\mapleinline{active}{1d}{A(j,0,0,Pi);}{%
}
\end{mapleinput}

\mapleresult
\begin{maplelatex}
\[
{\displaystyle \frac {1}{6}} \,\pi ^{2}
\]
\end{maplelatex}

\end{maplegroup}
\begin{maplegroup}
\begin{mapleinput}
\mapleinline{active}{1d}{B(j,k,0,Pi);}{%
}
\end{mapleinput}

\mapleresult
\begin{maplelatex}
\[
0
\]
\end{maplelatex}

\end{maplegroup}
\begin{maplegroup}
Donc, la s�rie de Fourier de j est :

\end{maplegroup}
\begin{maplegroup}
\begin{mapleinput}
\mapleinline{active}{1d}{SF_j := fourierR(j,infinity,0,Pi,x);}{%
}
\end{mapleinput}

\mapleresult
\begin{maplelatex}
\[
{\it SF\_j} := {\displaystyle \frac {1}{6}} \,\pi ^{2} + 2\,
 \left(  \! {\displaystyle \sum _{{\it k\symbol{126}}=1}^{\infty 
}} \,( - {\displaystyle \frac {1}{2}} \,{\displaystyle \frac {
{\rm cos}(2\,{\it k\symbol{126}}\,x)}{{\it k\symbol{126}}^{2}}} )
 \!  \right) 
\]
\end{maplelatex}

\end{maplegroup}
\begin{maplegroup}
En x=0, cela donne :

\end{maplegroup}
\begin{maplegroup}
\begin{mapleinput}
\mapleinline{active}{1d}{subs(x=0,SF_j);}{%
}
\end{mapleinput}

\mapleresult
\begin{maplelatex}
\[
{\displaystyle \frac {1}{6}} \,\pi ^{2} + 2\, \left(  \! 
{\displaystyle \sum _{{\it k\symbol{126}}=1}^{\infty }} \,( - 
{\displaystyle \frac {1}{2}} \,{\displaystyle \frac {{\rm cos}(0)
}{{\it k\symbol{126}}^{2}}} ) \!  \right) 
\]
\end{maplelatex}

\end{maplegroup}
\begin{maplegroup}
\begin{mapleinput}
\mapleinline{active}{1d}{value(");}{%
}
\end{mapleinput}

\mapleresult
\begin{maplelatex}
\[
0
\]
\end{maplelatex}

\end{maplegroup}
\begin{maplegroup}
or, 

\end{maplegroup}
\begin{maplegroup}
\begin{mapleinput}
\mapleinline{active}{1d}{j(0);}{%
}
\end{mapleinput}

\mapleresult
\begin{maplelatex}
\[
0
\]
\end{maplelatex}

\end{maplegroup}
\begin{maplegroup}
ok. Mais surtout, on retrouve la somme remarquable  
\mapleinline{inert}{2d}{Pi^2/6 = sum(1/k^2,k=1..infinity)}{%
$\frac {\pi ^{2}}{6}=\sum _{k=1}^{\infty }\,\frac {1}{k^{2}}$%
} .

\end{maplegroup}
\begin{maplegroup}
\begin{mapleinput}
\mapleinline{active}{1d}{sum(1/k^2, k=1..infinity);}{%
}
\end{mapleinput}

\mapleresult
\begin{maplelatex}
\[
{\displaystyle \frac {1}{6}} \,\pi ^{2}
\]
\end{maplelatex}

\end{maplegroup}
\begin{maplegroup}
\begin{mapleinput}
\end{mapleinput}

\end{maplegroup}

\section{Exercice 6}

\begin{maplegroup}
\begin{mapleinput}
\mapleinline{active}{1d}{k:='k';}{%
}
\end{mapleinput}

\mapleresult
\begin{maplelatex}
\[
k := k
\]
\end{maplelatex}

\end{maplegroup}
\begin{maplegroup}
\begin{mapleinput}
\mapleinline{active}{1d}{norme2carreR := (f,a,b) -> 1/(b-a) *
int(f(t)^2, t=a..b);}{%
}
\end{mapleinput}

\mapleresult
\begin{maplelatex}
\[
{\it norme2carreR} := (f, \,a, \,b)\rightarrow {\displaystyle 
\frac {{\displaystyle \int _{a}^{b}} f(t)^{2}\,dt}{b - a}} 
\]
\end{maplelatex}

\end{maplegroup}
\begin{maplegroup}
\begin{mapleinput}
\mapleinline{active}{1d}{parsevalR := (f,a,b) -> A(f,0,a,b)^2   +   2
* sum(A(f,k,a,b)^2 + B(f,k,a,b)^2, k=1..infinity);}{%
}
\end{mapleinput}

\mapleresult
\begin{maplelatex}
\[
{\it parsevalR} := (f, \,a, \,b)\rightarrow {\rm A}(f, \,0, \,a, 
\,b)^{2} + 2\,({\displaystyle \sum _{k=1}^{\infty }} \,({\rm A}(f
, \,k, \,a, \,b)^{2} + {\rm B}(f, \,k, \,a, \,b)^{2}))
\]
\end{maplelatex}

\end{maplegroup}
\begin{maplegroup}
\begin{mapleinput}
\mapleinline{active}{1d}{norme2carreR(cos,0,2*Pi);}{%
}
\end{mapleinput}

\mapleresult
\begin{maplelatex}
\[
{\displaystyle \frac {1}{2}} 
\]
\end{maplelatex}

\end{maplegroup}
\begin{maplegroup}
\begin{mapleinput}
\mapleinline{active}{1d}{parsevalR(cos,0,2*Pi);}{%
}
\end{mapleinput}

\mapleresult
\begin{maplelatex}
\[
{\displaystyle \frac {1}{2}} 
\]
\end{maplelatex}

\end{maplegroup}
\begin{maplegroup}
ok.

\end{maplegroup}
\begin{maplegroup}
\begin{mapleinput}
\end{mapleinput}

\end{maplegroup}
\begin{maplegroup}
\begin{mapleinput}
\mapleinline{active}{1d}{f := x->(-1)^floor(x);}{%
}
\end{mapleinput}

\end{maplegroup}
\begin{maplegroup}
\begin{mapleinput}
\mapleinline{active}{1d}{norme2carreR(f,0,2);}{%
}
\end{mapleinput}

\end{maplegroup}
\begin{maplegroup}
\begin{mapleinput}
\mapleinline{active}{1d}{pf := parsevalR(f,0,2);}{%
}
\end{mapleinput}

\end{maplegroup}
\begin{maplegroup}
\begin{mapleinput}
\mapleinline{active}{1d}{A(f,k,0,2);}{%
}
\end{mapleinput}

\end{maplegroup}
\begin{maplegroup}
\begin{mapleinput}
\mapleinline{active}{1d}{assume(k, integer, k>0);}{%
}
\end{mapleinput}

\end{maplegroup}
\begin{maplegroup}
\begin{mapleinput}
\mapleinline{active}{1d}{A(f,k,0,2);}{%
}
\end{mapleinput}

\end{maplegroup}
\begin{maplegroup}
\begin{mapleinput}
\mapleinline{active}{1d}{B(f,k,0,2);}{%
}
\end{mapleinput}

\end{maplegroup}
\begin{maplegroup}
\begin{mapleinput}
\mapleinline{active}{1d}{k:=2*l+1;}{%
}
\end{mapleinput}

\end{maplegroup}
\begin{maplegroup}
\begin{mapleinput}
\mapleinline{active}{1d}{assume(l, integer, l>=0);}{%
}
\end{mapleinput}

\end{maplegroup}
\begin{maplegroup}
\begin{mapleinput}
\mapleinline{active}{1d}{B(f,k,0,2);}{%
}
\end{mapleinput}

\end{maplegroup}
\begin{maplegroup}
\begin{mapleinput}
\mapleinline{active}{1d}{parseval_f := 2 * sum(B(f,k,0,2)^2,
l=0..infinity);}{%
}
\end{mapleinput}

\end{maplegroup}
\begin{maplegroup}
ok !

\end{maplegroup}
\begin{maplegroup}
\begin{mapleinput}
\end{mapleinput}

\end{maplegroup}
\begin{maplegroup}
\begin{mapleinput}
\mapleinline{active}{1d}{k:='k':}{%
}
\end{mapleinput}

\end{maplegroup}
\begin{maplegroup}
\begin{mapleinput}
\mapleinline{active}{1d}{l:='l':}{%
}
\end{mapleinput}

\end{maplegroup}
\begin{maplegroup}
\begin{mapleinput}
\mapleinline{active}{1d}{parsevalR(j,0,Pi);}{%
}
\end{mapleinput}

\end{maplegroup}
\begin{maplegroup}
\begin{mapleinput}
\mapleinline{active}{1d}{?hypergeom}{%
}
\end{mapleinput}

\end{maplegroup}
\begin{maplegroup}
\begin{mapleinput}
\mapleinline{active}{1d}{A(j,0,0,Pi);}{%
}
\end{mapleinput}

\end{maplegroup}
\begin{maplegroup}
\begin{mapleinput}
\mapleinline{active}{1d}{A(j,k,0,Pi);}{%
}
\end{mapleinput}

\end{maplegroup}
\begin{maplegroup}
\begin{mapleinput}
\mapleinline{active}{1d}{assume(k, integer, k>0);}{%
}
\end{mapleinput}

\end{maplegroup}
\begin{maplegroup}
\begin{mapleinput}
\mapleinline{active}{1d}{A(j,k,0,Pi);}{%
}
\end{mapleinput}

\end{maplegroup}
\begin{maplegroup}
\begin{mapleinput}
\mapleinline{active}{1d}{B(j,k,0,Pi);}{%
}
\end{mapleinput}

\end{maplegroup}
\begin{maplegroup}
Donc, la "somme de Parseval" est :

\end{maplegroup}
\begin{maplegroup}
\begin{mapleinput}
\mapleinline{active}{1d}{A(j,0,0,Pi)^2   +   2 * Sum(A(j,k,0,Pi)^2,
k=1..infinity);}{%
}
\end{mapleinput}

\end{maplegroup}
\begin{maplegroup}
or,

\end{maplegroup}
\begin{maplegroup}
\begin{mapleinput}
\mapleinline{active}{1d}{norme2carreR(j,0,Pi);}{%
}
\end{mapleinput}

\end{maplegroup}
\begin{maplegroup}
On retrouve donc que la valeur de  
\mapleinline{inert}{2d}{sum(1/k^4,k=1..infinity)}{%
$\sum _{k=1}^{\infty }\,\frac {1}{k^{4}}$%
}  est :

\end{maplegroup}
\begin{maplegroup}
\begin{mapleinput}
\mapleinline{active}{1d}{2* (norme2carreR(j,0,Pi) - A(j,0,0,Pi)^2);}{%
}
\end{mapleinput}

\end{maplegroup}
\begin{maplegroup}
\begin{mapleinput}
\end{mapleinput}

\end{maplegroup}
\end{document}
%% End of Maple V Output
