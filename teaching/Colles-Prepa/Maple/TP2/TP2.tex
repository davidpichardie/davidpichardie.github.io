\documentclass[12pt,a4paper]{article}
\usepackage{fancyheadings}
\usepackage{pstricks,pst-node,pst-tree}
\usepackage{amsmath}
\usepackage[ansinew]{inputenc}

\title{TP2 {\sc Maple} : Polyn�mes}
\author{}
\date{}


\setlength{\oddsidemargin}{0cm}
\addtolength{\textwidth}{70pt}
\setlength{\topmargin}{0cm}
\addtolength{\textheight}{2cm}
\setlength{\parindent}{0cm}


\pagestyle{fancy}
\lhead{TP2 {\sc Maple}}
\rhead{Polyn�mes}

\newcommand{\R}{\mathbf{R}}
\newcommand{\C}{\mathbf{C}}
\newcommand{\N}{\mathbf{N}}
 
\newcounter{numsubquestion}
\newcounter{numquestion}
\setcounter{numquestion}{1}
\newenvironment{question}{\setcounter{numsubquestion}{1}\noindent{\bf Exercice \thenumquestion.}}%
{\stepcounter{numquestion}\medskip}

\newenvironment{subquestion}{\smallskip\noindent{\bf \thenumquestion.\thenumsubquestion}}%
{\stepcounter{numsubquestion}\smallskip}


\begin{document}

\maketitle

\begin{question}

\begin{subquestion}
Factoriser le polyn�me suivant sur $\R$ puis sur $\C$.
$$P=x^5-x^4-4x^3+4x^2-5x+5$$    
\end{subquestion}
\vspace{-5ex}

\begin{subquestion}
D�composer la fraction rationelle suivante en �l�ment simple dans $\R$
puis dans $\C$.
$$F=\frac{1}{X^7 + 27 X^4 - X^3 - 27}$$
\end{subquestion}
Commandes {\sc Maple} : {\tt parfrac, factor, solve, denom}
\end{question}

\begin{question}
Quelle est la multiplicit� de $1$ dans le polyn�me suivant, $n\in\N$ ?
$$P_n=X^{2n} - n^2 X^{n+1} + 2 (n^2-1) X^n - n^2 X^{n-1} + 1$$

Commandes {\sc Maple} : {\tt subs, simplify, diff}
\end{question}

\begin{question}
Soient $E$ l'espace vectoriel des polyn�mes de $\R [X]$ de degr�
inf�rieur ou �gale � $3$, $\varphi_1$, $\varphi_2$, $\varphi_3$,
$\varphi_4$ les �l�ments de $E^*$ d�finis par~: 
$$\forall P\in E\quad
   \varphi_1(P)=P(1),\ 
   \varphi_2(P)=P(2),\ 
   \varphi_3(P)=P'(1),\ 
   \varphi_4(P)=\int_0^1 P(x)\,dx
$$
Montrer que $\varphi_1$, $\varphi_2$, $\varphi_3$,
$\varphi_4$ forment une base de $E^*$ et en donner la base duale. 
V�rifier vos r�sultats.

Indication : Ecrire les $\varphi_i$ dans la base duale usuelle de $E$.

Commandes {\sc Maple} : {\tt matrix(4,4,(i,j)->...), linsolve}
\end{question}

\begin{question} {\bf (Suite de Sturm)}

\begin{subquestion}
Ecrire une fonction qui renvoie la liste des restes obtenus lors du
calcul du pgcd de deux polyn�mes par l'algorithme d'Euclide.

Test : {\tt euclide}$(x^5+1,x^3+3)=[x^5+1,x^3+3,1-3x^2,3+\frac{1}{3}x,-242]$

Commandes {\sc Maple} : {\tt rem, degree}
\end{subquestion}

\begin{subquestion}
Ecrire une fonction qui calcule le nombre de changements de signes
dans une liste, sans tenir compte des valeurs nulles.

Indication : Pour enlever les z�ros d'une liste {\tt L}, utiliser la commande
{\tt remove(x->(x=0),L)}.

Test : {\tt chg\_signe}$([-2,0,3,0,-4,-2,1,0])=3$
\end{subquestion}

\begin{subquestion}
Ecrire une fonction qui calcule pour un polyn�me $P$ donn�, les $s$
premiers termes de la suite 
$P_0,P_1,\ldots,P_{s+1}$ d�finie par~:
$$P_0=P,\ P_1=-P',\ P_2=-P_0 \mod P_1,\ldots,
            P_{i+1}=-P_{i-1} \mod P_i,\ldots,P_{s+1}=0$$

C'est une {\it suite de Sturm} du polyn�me $P$.

Test : {\tt sturm}$(x^4-20x^2+9x-8)=
[x^3-2x^2-5x+6, -3x^2+4x+5, -\frac{44}{9}+\frac{38}{9}x, -\frac{2025}{361}]$
\end{subquestion}

\begin{subquestion}
Ecrire une fonction qui donne le nombre de changements de signes dans
une suite de Sturm �valu�e en un point $x$ donn�. 

Test : {\tt eval\_sturm}$(x^3-2x^2-5x+6,0)$=1
\end{subquestion}

\begin{subquestion}
Gr�ce � un repr�sentation graphique, proposer une interpr�tation de
l'entier
$$\text{\tt eval\_sturm}(P,b)-\text{\tt eval\_sturm}(P,a)$$
o� $P$ est le polyn�me $x^3-2x^2-5x+6$ et $a<b$. 

Cette propri�t� se g�n�ralise au polyn�mes de $\R[X]$ � z�ros
simples. Tester-le avec d'autres exemples.
\end{subquestion}

\begin{subquestion}
Ecrire une fonction qui pour un polyn�me $P$ de $\R[X]$, renvoie un polyn�me 
poss�dant les m�me z�ros, mais tous simples.

Commandes {\sc Maple} : {\tt quo, gcd, diff}
\end{subquestion}
\end{question}

\end{document}

