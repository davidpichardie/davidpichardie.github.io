\documentclass[12pt,a4paper]{article}
\usepackage{fancyheadings}
\usepackage{pstricks,pst-node,pst-tree}
\usepackage{amsmath}
\usepackage[ansinew]{inputenc}
\usepackage{maple2e}
 


\title{TP2 {\sc Maple} : Polyn�mes}
\author{}
\date{}

\setlength{\oddsidemargin}{0cm}
\addtolength{\textwidth}{70pt}
\setlength{\topmargin}{0cm}
\addtolength{\textheight}{2cm}
\setlength{\parindent}{0cm}

\pagestyle{fancy}
\lhead{Corrig� TP2 {\sc Maple}}
\rhead{Polyn�mes}


\DefineParaStyle{Heading 1}
\DefineParaStyle{Maple Output}
\DefineParaStyle{Warning}
\DefineCharStyle{2D Comment}
\DefineCharStyle{2D Math}
\DefineCharStyle{2D Output}

\newcounter{numquestion}
\setcounter{numquestion}{1}
\newenvironment{question}{\noindent{\bf Exercice \thenumquestion.}}%
{\stepcounter{numquestion}\medskip}

\newcommand{\R}{\mathbf{R}}
\newcommand{\C}{\mathbf{C}}
\newcommand{\N}{\mathbf{N}}
\newcommand{\Q}{\mathbf{Q}}
\newcommand{\pgcd}{\operatorname{pgcd}}


\begin{document}

{\bf\Large Exercice 1}

\begin{maplegroup}
\begin{mapleinput}
\mapleinline{active}{1d}{P:=x^5-x^4-4*x^3+4*x^2-5*x+5;}{%
}
\end{mapleinput}

\mapleresult
\begin{maplelatex}
\[
P := x^{5} - x^{4} - 4\,x^{3} + 4\,x^{2} - 5\,x + 5
\]
\end{maplelatex}

\end{maplegroup}
\begin{maplegroup}
\begin{mapleinput}
\mapleinline{active}{1d}{factor(P);}{%
}
\end{mapleinput}

\mapleresult
\begin{maplelatex}
\[
(x - 1)\,(x^{2} - 5)\,(x^{2} + 1)
\]
\end{maplelatex}

\end{maplegroup}

Par d�faut, {\sc Maple} factorise dans $\Q$. Pour pousser plus loin la
factorisation, il faut indiquer dans quelle {\it extension} de $\Q$
{\sc Maple} doit travailler. Concr�tement, on fournit � {\sc Maple}
une liste $\{a_1,\ldots,a_n\}$ de nombres complexes tels que
les z�ros du polyn�me � factoriser s'�crivent comme des produits et
des sommes des $a_i$ et de rationelles. On travaille ainsi dans 
l'extension $\Q(a_1,\ldots,a_n)$ {\it i.e.} le plus petit corps
contenant $\Q$ et les $a_i$.

\begin{maplegroup}
\begin{mapleinput}
\mapleinline{active}{1d}{solve(P);}{%
}
\end{mapleinput}

\mapleresult
\begin{maplelatex}
\[
1, \,\sqrt{5}, \, - \sqrt{5}, \,I, \, - I
\]
\end{maplelatex}

\end{maplegroup}
\begin{maplegroup}
\begin{mapleinput}
\mapleinline{active}{1d}{factor(P,sqrt(5));}{%
}
\end{mapleinput}

\mapleresult
\begin{maplelatex}
\[
(x^{2} + 1)\,(x - \sqrt{5})\,(x + \sqrt{5})\,(x - 1)
\]
\end{maplelatex}

\end{maplegroup}
\begin{maplegroup}
\begin{mapleinput}
\mapleinline{active}{1d}{factor(P,\{sqrt(5),I\});}{%
}
\end{mapleinput}

\mapleresult
\begin{maplelatex}
\[
(x - I)\,(x + I)\,(x - \sqrt{5})\,(x + \sqrt{5})\,(x - 1)
\]
\end{maplelatex}

\end{maplegroup}
\begin{maplegroup}
\begin{mapleinput}
\mapleinline{active}{1d}{F := 1/(X^7 + 27*X^4 - X^3 - 27);}{%
}
\end{mapleinput}

\mapleresult
\begin{maplelatex}
\[
F := {\displaystyle \frac {1}{X^{7} + 27\,X^{4} - X^{3} - 27}} 
\]
\end{maplelatex}

\end{maplegroup}
\begin{maplegroup}
\begin{mapleinput}
\mapleinline{active}{1d}{convert(F,parfrac,X);}{%
}
\end{mapleinput}

\mapleresult
\begin{maplelatex}
\[
{\displaystyle \frac {1}{112}} \,{\displaystyle \frac {1}{X - 1}
}  + {\displaystyle \frac {1}{2160}} \,{\displaystyle \frac {1}{X
 + 3}}  - {\displaystyle \frac {1}{104}} \,{\displaystyle \frac {
1}{X + 1}}  - {\displaystyle \frac {1}{1460}} \,{\displaystyle 
\frac {27 + X}{X^{2} + 1}}  + {\displaystyle \frac {1}{179361}} 
\,{\displaystyle \frac { - 249 + 163\,X}{X^{2} - 3\,X + 9}} 
\]
\end{maplelatex}

\end{maplegroup}
\begin{maplegroup}
\begin{mapleinput}
\mapleinline{active}{1d}{solve(denom(F));}{%
}
\end{mapleinput}

\mapleresult
\begin{maplelatex}
\[
1, \,-3, \,-1, \,I, \, - I, \,{\displaystyle \frac {3}{2}}  - 
{\displaystyle \frac {3}{2}} \,I\,\sqrt{3}, \,{\displaystyle 
\frac {3}{2}}  + {\displaystyle \frac {3}{2}} \,I\,\sqrt{3}
\]
\end{maplelatex}

\end{maplegroup}
\begin{maplegroup}
\begin{mapleinput}
\mapleinline{active}{1d}{convert(F,parfrac,X,\{I,sqrt(3)\});}{%
}
\end{mapleinput}

\mapleresult
\begin{maplelatex}
\begin{eqnarray*}
\lefteqn{{\displaystyle \frac {1}{179361}} \,{\displaystyle 
\frac {163 + I\,\sqrt{3}}{2\,X - 3 - 3\,I\,\sqrt{3}}}  - 
{\displaystyle \frac {1}{179361}} \,{\displaystyle \frac {I\,
\sqrt{3} - 163}{2\,X - 3 + 3\,I\,\sqrt{3}}}  + {\displaystyle 
\frac { - {\displaystyle \frac {1}{2920}}  - {\displaystyle 
\frac {27}{2920}} \,I}{X + I}}  + {\displaystyle \frac { - 
{\displaystyle \frac {1}{2920}}  + {\displaystyle \frac {27}{2920
}} \,I}{X - I}} } \\
 & & \mbox{} - {\displaystyle \frac {1}{104}} \,{\displaystyle 
\frac {1}{X + 1}}  + {\displaystyle \frac {1}{112}} \,
{\displaystyle \frac {1}{X - 1}}  + {\displaystyle \frac {1}{2160
}} \,{\displaystyle \frac {1}{X + 3}} \mbox{\hspace{237pt}}
\end{eqnarray*}
\end{maplelatex}

\end{maplegroup}

\medskip{\bf\Large Exercice 2}

\begin{maplegroup}
\begin{mapleinput}
\mapleinline{active}{1d}{P := X^(2*n) - n^2*X^(n+1) + 2*(n^2-1)*X^n -
n^2*X^(n-1) + 1;}{%
}
\end{mapleinput}

\mapleresult
\begin{maplelatex}
\[
P := X^{(2\,n)} - n^{2}\,X^{(n + 1)} + 2\,(n^{2} - 1)\,X^{n} - n
^{2}\,X^{(n - 1)} + 1
\]
\end{maplelatex}

\end{maplegroup}
\begin{maplegroup}
\begin{mapleinput}
\mapleinline{active}{1d}{subs(X=1,A);}{%
}
\end{mapleinput}

\mapleresult
\begin{maplelatex}
\[
0
\]
\end{maplelatex}

\end{maplegroup}
\begin{maplegroup}
\begin{mapleinput}
\mapleinline{active}{1d}{P1:=diff(A,X):}{%
}
\end{mapleinput}

\end{maplegroup}
\begin{maplegroup}
\begin{mapleinput}
\mapleinline{active}{1d}{subs(X=1,P1);}{%
}
\end{mapleinput}

\mapleresult
\begin{maplelatex}
\[
2\,n - n^{2}\,(n + 1) + 2\,(n^{2} - 1)\,n - n^{2}\,(n - 1)
\]
\end{maplelatex}

\end{maplegroup}
\begin{maplegroup}
\begin{mapleinput}
\mapleinline{active}{1d}{simplify(");}{%
}
\end{mapleinput}

\mapleresult
\begin{maplelatex}
\[
0
\]
\end{maplelatex}

\end{maplegroup}
\begin{maplegroup}
\begin{mapleinput}
\mapleinline{active}{1d}{P2:=diff(P1,X):simplify(subs(X=1,P2));
}{%
}
\end{mapleinput}

\mapleresult
\begin{maplelatex}
\[
0
\]
\end{maplelatex}

\end{maplegroup}
\begin{maplegroup}
\begin{mapleinput}
\mapleinline{active}{1d}{P3:=diff(P2,X):simplify(subs(X=1,P3));}{%
}
\end{mapleinput}

\mapleresult
\begin{maplelatex}
\[
0
\]
\end{maplelatex}

\end{maplegroup}
\begin{maplegroup}
\begin{mapleinput}
\mapleinline{active}{1d}{P4:=diff(P3,X):simplify(subs(X=1,P4));}{%
}
\end{mapleinput}

\mapleresult
\begin{maplelatex}
\[
 - 2\,n^{2} + 2\,n^{4}
\]
\end{maplelatex}

\end{maplegroup}
\begin{maplegroup}
\begin{mapleinput}
\mapleinline{active}{1d}{solve(");}{%
}
\end{mapleinput}

\mapleresult
\begin{maplelatex}
\[
1, \,-1, \,0, \,0
\]
\end{maplelatex}

\end{maplegroup}

Donc $1$ est un z�ro d'ordre $4$ pour $P_n$ si $n\geq 2$. On traite
les cas $n=0$ et $n=1$ � part~:
 
\begin{maplegroup}
\begin{mapleinput}
\mapleinline{active}{1d}{n:=0: P;}{%
}
\end{mapleinput}

\mapleresult
\begin{maplelatex}
\[
0
\]
\end{maplelatex}

\end{maplegroup}
\begin{maplegroup}
\begin{mapleinput}
\mapleinline{active}{1d}{n:=1: P;}{%
}
\end{mapleinput}

\mapleresult
\begin{maplelatex}
\[
0
\]
\end{maplelatex}

\end{maplegroup}

{\bf\Large Exercice 3}

\begin{maplegroup}
\begin{mapleinput}
\mapleinline{active}{1d}{restart;}{%
}
\end{mapleinput}

\end{maplegroup}
\begin{maplegroup}
\begin{mapleinput}
\mapleinline{active}{1d}{with(linalg):}{%
}
\end{mapleinput}
\end{maplegroup}

\begin{maplegroup}
\begin{mapleinput}
\mapleinline{active}{1d}{f[1]:=P->subs(\{x=1\},P);}{%
}
\end{mapleinput}

\mapleresult
\begin{maplelatex}
\[
{f_{1}} := P\rightarrow {\rm subs}(\{x=1\}, \,P)
\]
\end{maplelatex}

\end{maplegroup}
\begin{maplegroup}
\begin{mapleinput}
\mapleinline{active}{1d}{f[2]:=P->subs(\{x=2\},P);}{%
}
\end{mapleinput}

\mapleresult
\begin{maplelatex}
\[
{f_{2}} := P\rightarrow {\rm subs}(\{x=2\}, \,P)
\]
\end{maplelatex}

\end{maplegroup}
\begin{maplegroup}
\begin{mapleinput}
\mapleinline{active}{1d}{f[3]:=P -> subs(\{x=1\},diff(P,x));}{%
}
\end{mapleinput}

\mapleresult
\begin{maplelatex}
\[
{f_{3}} := P\rightarrow {\rm subs}(\{x=1\}, \,{\frac {\partial }{
\partial x}}\,P)
\]
\end{maplelatex}

\end{maplegroup}
\begin{maplegroup}
\begin{mapleinput}
\mapleinline{active}{1d}{f[4]:=P -> int(P,x=0..1);}{%
}
\end{mapleinput}

\mapleresult
\begin{maplelatex}
\[
{f_{4}} := P\rightarrow {\displaystyle \int _{0}^{1}} P\,dx
\]
\end{maplelatex}

\end{maplegroup}

Si on note $(e_1,e_2,e_3,e_4)=(1,X,X^2,X^3)$ la base usuelle de $E$ et  
$(e^*_1,e^*_2,e^*_3,e^*_4)$ sa base duale, chaque $\varphi_i$ peut
s'�crire dans cette base~:
$$\varphi_i = a_{i1}e^*_1+a_{i2}e^*_2+a_{i3}e^*_3+a_{i4}e^*_4$$

On calcule les $a_{ij}$ gr�ce � la formule
$$a_{ij}=\varphi_i(e_j)$$

On place ces coefficients dans une matrice o� chaque ligne repr�sente
une des forme $\varphi_i$.

\begin{maplegroup}
\begin{mapleinput}
\mapleinline{active}{1d}{A:=matrix(4,4,(i,j)->f[i](x^(j-1)));}{%
}
\end{mapleinput}

\mapleresult
\begin{maplelatex}
\[
A :=  \left[ 
{\begin{array}{rccc}
1 & 1 & 1 & 1 \\
1 & 2 & 4 & 8 \\
0 & 1 & 2 & 3 \\
1 & {\displaystyle \frac {1}{2}}  & {\displaystyle \frac {1}{3}} 
 & {\displaystyle \frac {1}{4}} 
\end{array}}
 \right] 
\]
\end{maplelatex}

\end{maplegroup}
\begin{maplegroup}
\begin{mapleinput}
\mapleinline{active}{1d}{det(A);}{%
}
\end{mapleinput}

\mapleresult
\begin{maplelatex}
\[
{\displaystyle \frac {7}{12}} 
\]
\end{maplelatex}

\end{maplegroup}

La famille $(\varphi_1,\varphi_2,\varphi_3,\varphi_4)$ forme donc une
base de $E^*$. 

\begin{maplegroup}
\begin{mapleinput}
\mapleinline{active}{1d}{X:=vector([1,x,x^2,x^3]);}{%
}
\end{mapleinput}

\mapleresult
\begin{maplelatex}
\[
X := [1, \,x, \,x^{2}, \,x^{3}]
\]
\end{maplelatex}

\end{maplegroup}

Si $P\in E$ admet comme d�composition $P=ae_1+be_2+ce_3+de_4$, alors
$$A\cdot \left(\begin{array}{c} a \\ b \\ c \\ d \end{array}\right)
  = \left(\begin{array}{c} \varphi_1(P) \\ 
                           \varphi_2(P) \\ 
                           \varphi_3(P) \\ 
                           \varphi_4(P) \end{array}\right)$$
On a donc $4$ syst�mes lin�aires � r�soudre pour trouver la base
duale~:

\begin{maplegroup}
\begin{mapleinput}
\mapleinline{active}{1d}{linsolve(A,[1,0,0,0]);P[1]:=evalm(" &* X);}{%
}
\end{mapleinput}

\mapleresult
\begin{maplelatex}
\[
 \left[  \! {\displaystyle \frac {-16}{7}} , \,{\displaystyle 
\frac {54}{7}} , \,{\displaystyle \frac {-39}{7}} , \,
{\displaystyle \frac {8}{7}}  \!  \right] 
\]
\end{maplelatex}

\begin{maplelatex}
\[
{P_{1}} :=  - {\displaystyle \frac {16}{7}}  + {\displaystyle 
\frac {54}{7}} \,x - {\displaystyle \frac {39}{7}} \,x^{2} + 
{\displaystyle \frac {8}{7}} \,x^{3}
\]
\end{maplelatex}

\end{maplegroup}
\begin{maplegroup}
\begin{mapleinput}
\mapleinline{active}{1d}{linsolve(A,[0,1,0,0]);P[2]:=evalm(" &* X);}{%
}
\end{mapleinput}

\mapleresult
\begin{maplelatex}
\[
 \left[  \! {\displaystyle \frac {-1}{7}} , \,{\displaystyle 
\frac {6}{7}} , \,{\displaystyle \frac {-9}{7}} , \,
{\displaystyle \frac {4}{7}}  \!  \right] 
\]
\end{maplelatex}

\begin{maplelatex}
\[
{P_{2}} :=  - {\displaystyle \frac {1}{7}}  + {\displaystyle 
\frac {6}{7}} \,x - {\displaystyle \frac {9}{7}} \,x^{2} + 
{\displaystyle \frac {4}{7}} \,x^{3}
\]
\end{maplelatex}

\end{maplegroup}
\begin{maplegroup}
\begin{mapleinput}
\mapleinline{active}{1d}{linsolve(A,[0,0,1,0]);P[3]:=evalm(" &* X);}{%
}
\end{mapleinput}

\mapleresult
\begin{maplelatex}
\[
 \left[  \! {\displaystyle \frac {6}{7}} , \,{\displaystyle 
\frac {-29}{7}} , \,{\displaystyle \frac {33}{7}} , \,
{\displaystyle \frac {-10}{7}}  \!  \right] 
\]
\end{maplelatex}

\begin{maplelatex}
\[
{P_{3}} := {\displaystyle \frac {6}{7}}  - {\displaystyle \frac {
29}{7}} \,x + {\displaystyle \frac {33}{7}} \,x^{2} - 
{\displaystyle \frac {10}{7}} \,x^{3}
\]
\end{maplelatex}

\end{maplegroup}
\begin{maplegroup}
\begin{mapleinput}
\mapleinline{active}{1d}{linsolve(A,[0,0,0,1]);P[4]:=evalm(" &* X);}{%
}
\end{mapleinput}

\mapleresult
\begin{maplelatex}
\[
 \left[  \! {\displaystyle \frac {24}{7}} , \,{\displaystyle 
\frac {-60}{7}} , \,{\displaystyle \frac {48}{7}} , \,
{\displaystyle \frac {-12}{7}}  \!  \right] 
\]
\end{maplelatex}

\begin{maplelatex}
\[
{P_{4}} := {\displaystyle \frac {24}{7}}  - {\displaystyle 
\frac {60}{7}} \,x + {\displaystyle \frac {48}{7}} \,x^{2} - 
{\displaystyle \frac {12}{7}} \,x^{3}
\]
\end{maplelatex}

\end{maplegroup}

V�rification~:

\begin{maplegroup}
\begin{mapleinput}
\mapleinline{active}{1d}{matrix(4,4,(i,j)->f[j](P[i]));}{%
}
\end{mapleinput}

\mapleresult
\begin{maplelatex}
\[
 \left[ 
{\begin{array}{rrrr}
1 & 0 & 0 & 0 \\
0 & 1 & 0 & 0 \\
0 & 0 & 1 & 0 \\
0 & 0 & 0 & 1
\end{array}}
 \right] 
\]
\end{maplelatex}

\end{maplegroup}

{\bf\Large Exercice 4}

\begin{maplegroup}
\begin{mapleinput}
\mapleinline{active}{1d}{euclide:=proc(P,Q)}{}
\vspace{-.3cm}
\begin{verbatim}
       local i,A,B,C,R;
       A:=P;
       B:=Q;
       R:=A,B;
       while degree(B)<>0 do
         C:=rem(A,B,x);
         R:=R,C;
         A:=B;     
         B:=C;
       od;
       [R];
      end:
\end{verbatim}
\end{mapleinput}
\end{maplegroup}

\begin{maplegroup}
\begin{mapleinput}
\mapleinline{active}{1d}{chg_signe:=proc(L)}{}
\vspace{-.3cm}
\begin{verbatim}
       local i,y,R,nb;
       R:=remove(x->(x=0),L);
       nb:=0;
       for i from 2 to nops(R) do
         if R[i-1]*R[i]<0 then nb:=nb+1 fi;
       od;
       nb;
      end:
\end{verbatim}
\end{mapleinput}
\end{maplegroup}

\newpage
\begin{maplegroup}
\begin{mapleinput}
\mapleinline{active}{1d}{sturm:=proc(P)}{}
\vspace{-.3cm}
\begin{verbatim}
       local i,A,B,C,R;
       A:=P;
       B:=-diff(P,x);
       R:=A,B;
       while degree(B)<>0 do
         C:=-rem(A,B,x);
         R:=R,C;
         A:=B;     
         B:=C;
       od;
       [R];
      end:
\end{verbatim}
\end{mapleinput}
\end{maplegroup}

\begin{maplegroup}
\begin{mapleinput}
\mapleinline{active}{1d}{eval_sturm:=proc(P,a)}{}
\vspace{-.38cm}
\begin{verbatim}
       chg_signe(subs(\{x=a\},sturm(P)));
      end:
\end{verbatim}
\end{mapleinput}
\end{maplegroup}

\begin{maplegroup}
\begin{mapleinput}
\mapleinline{active}{1d}{P:=expand((x-1)*(x+2)*(x-3));}{%
}
\end{mapleinput}

\mapleresult
\begin{maplelatex}
\[
P := x^{3} - 2\,x^{2} - 5\,x + 6
\]
\end{maplelatex}

\end{maplegroup}
\begin{maplegroup}
\mapleresult
\begin{maplettyout}
\end{maplettyout}

\end{maplegroup}
\begin{maplegroup}
\begin{mapleinput}
\mapleinline{active}{1d}{plot(v->eval_sturm(P,v),-4..4);}{%
}
\end{mapleinput}

\mapleresult
\begin{center}
\mapleplot{TP201.eps}
\end{center}

\end{maplegroup}

On constate que {\tt eval\_sturm} augmente de $1$ � chaque z�ro de $P$.

\begin{maplegroup}
\begin{mapleinput}
\mapleinline{active}{1d}{eval_sturm(P,4)-eval_sturm(P,-4);}{%
}
\end{mapleinput}

\mapleresult
\begin{maplelatex}
\[
3
\]
\end{maplelatex}
\end{maplegroup}

$\text{\tt eval\_sturm}(P,b)-\text{\tt eval\_sturm}(P,a)$ repr�sente
le nombre de z�ros du polyn�me $P$ dans l'intervalle $[a,b[$.

\begin{maplegroup}
\begin{mapleinput}
\mapleinline{active}{1d}{zero_simple:=proc(P)}{}
\vspace{-.26cm}
\begin{verbatim}
        quo(P,gcd(P,diff(P,x)),x) 
       end:
\end{verbatim}
\end{mapleinput}
\end{maplegroup}

Il faut en effet calculer $$\frac{P}{\pgcd(P,P')}$$
Pour le v�rifier on �crie $P$ sous la forme
$$P(X)=\alpha\prod_{i=1}^n \left(X-a_i\right)^{r_i}\quad a_i\in\C$$
On sait alors que chaque $a_i$ est un z�ro d'ordre $r_i-1$ pour
$P'$. Par cons�quent, en notant
$$\Pi(X)=\prod_{i=1}^n \left(X-a_i\right)^{r_i-1}$$
on a
$$\Pi | P'\text{ et donc }
\Pi | \pgcd(P,P')$$
Si $\deg(\Pi)<\deg(\pgcd(P,P'))$, il existe $a\in\C-\{a_i | r_i>1\}$
tel que $X-a | \pgcd(P,P')$ et par suite $P(a)=P'(a)=0$. Or ceci est
impossible si $a\not\in\{a_i | r_i>1\}$ donc $\Pi=\pgcd(P,P')$. On a
ainsi
$$\frac{P}{\pgcd(P,P')}=\alpha\prod_{i=1}^n \left(X-a_i\right)$$
 

\begin{maplegroup}
\begin{mapleinput}
\mapleinline{active}{1d}{P:=(x-1)^3*(x-2)^2*(x-3);}{%
}
\end{mapleinput}

\mapleresult
\begin{maplelatex}
\[
P := (x - 1)^{3}\,(x - 2)^{2}\,(x - 3)
\]
\end{maplelatex}

\end{maplegroup}
\begin{maplegroup}
\begin{mapleinput}
\mapleinline{active}{1d}{factor(zero_simple(P));}{%
}
\end{mapleinput}

\mapleresult
\begin{maplelatex}
\[
(x - 1)\,(x - 2)\,(x - 3)
\]
\end{maplelatex}

\end{maplegroup}
\end{document}