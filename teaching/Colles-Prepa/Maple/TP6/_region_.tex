\message{ !name(TP6corrige.tex)}{\makeatletter\gdef\AucTeX@cite#1[#2]#3{[#3#1#2]}\gdef\cite{\@ifnextchar[{\AucTeX@cite{, }}{\AucTeX@cite{}[]}}}

\message{ !name(TP6corrige.tex) !offset(-3) }
\documentclass[12pt,a4paper]{article}
\usepackage{fancyheadings}
\usepackage{pstricks,pst-node,pst-tree}
\usepackage{amsmath}
\usepackage[ansinew]{inputenc}
\usepackage{maple2e}
\usepackage[frenchb]{babel}
 


\title{corrig� TP6 {\sc Maple} : Introduction aux �quations diff�rentielles}
\author{}
\date{}

\setlength{\oddsidemargin}{0cm}
\addtolength{\textwidth}{70pt}
\setlength{\topmargin}{0cm}
\addtolength{\textheight}{2cm}
\setlength{\parindent}{0cm}

\pagestyle{fancy}
\lhead{Corrig� TP6 {\sc Maple}}
\rhead{Introduction aux �quations diff�rentielles}


\DefineParaStyle{Heading 1}
\DefineParaStyle{Maple Output}
\DefineParaStyle{Warning}
\DefineCharStyle{2D Comment}
\DefineCharStyle{2D Math}
\DefineCharStyle{2D Output}

\newcounter{numquestion}
\setcounter{numquestion}{1}
\newenvironment{question}{\noindent{\bf Exercice \thenumquestion.}}%
{\stepcounter{numquestion}\medskip}

\newcommand{\R}{\mathbf{R}}
\newcommand{\Z}{\mathbf{Z}}
\newcommand{\C}{\mathbf{C}}
\newcommand{\N}{\mathbf{N}}
\newcommand{\Q}{\mathbf{Q}}
\newcommand{\pgcd}{\operatorname{pgcd}}


\begin{document}

\section{Etude qualitative}

\begin{question}

\begin{maplegroup}
\begin{mapleinput}
\mapleinline{active}{1d}{with(DEtools);}{%
}
\end{mapleinput}
\end{maplegroup}
\begin{maplegroup}
\begin{mapleinput}
\mapleinline{active}{1d}{E1:=diff(y(x),x)=y(x)^2;}{%
}
\end{mapleinput}

\mapleresult
\begin{maplelatex}
\[
{\it E1} := {\frac {\partial }{\partial x}}\,{\rm y}(x)={\rm y}(x
)^{2}
\]
\end{maplelatex}

\end{maplegroup}
\begin{maplegroup}
\begin{mapleinput}
\mapleinline{active}{1d}{dsolve(E1,y(x));}{%
}
\end{mapleinput}

\mapleresult
\begin{maplelatex}
\[
{\displaystyle \frac {1}{{\rm y}(x)}} = - x + {\it \_C1}
\]
\end{maplelatex}

\end{maplegroup}

{\sc Maple} a oubli� la solution nulle !

\end{question}

\begin{question}
  
Le champ des tangente est donn� par $\phi$.
\begin{maplegroup}
\begin{mapleinput}
\mapleinline{active}{1d}{dfieldplot(E1,y(x),x=-5..5,y=-2..2,colour=blue);}{%
}
\end{mapleinput}

\mapleresult
\begin{center}
\mapleplot{tp601.eps}
\end{center}

\end{maplegroup}
\begin{maplegroup}
\begin{mapleinput}
\mapleinline{active}{1d}{DEplot(E1,y(x),x=-5..5,[[y(0)=0],[y(0)=1],[y(
0)=-1]],y=-2..2);}{%
}
\end{mapleinput}

\mapleresult
\begin{center}
\mapleplot{tp603.eps}
\end{center}

\end{maplegroup}
Pour construire une solution \guillemotleft\ � la main
\guillemotright\, , il faut trouver un chemin qui est tangent � toutes
les fl�ches qu'il rencontre.

\end{question}

\begin{question}
Si on se restreint au solution qui ne s'annulent jamai,s $(1)$
est �quivalent � 
$$\frac{y'}{y^2}=1$$
ce qui s'int�gre pour donner
$$y(x)=\frac{-1}{x-C}\quad x\in]-\infty,C[\cup ]C,+\infty[$$

La fonction nulle est une solution maximale su $\R$ de l'�quation
$(1)$. Une autre solution maximale ne peut donc jamais s'annuler ({\it
  cf} introduction du TP), elle
dont forc�ment, d'apr�s  ce qui pr�c�de de la forme $x\mapsto
\frac{-1}{x-C}$ sur $]-\infty,C[$ ou $x\mapsto
\frac{-1}{x-C}$ sur $]C,+\infty[$. De telles solutions sont maximales
car elles ne peuvent pas �tre prolong�es en $C$.
\end{question}

\begin{question}
  
\begin{maplegroup}
\begin{mapleinput}
\mapleinline{active}{1d}{E2:=diff(y(x),x)=y(x)^2-1;}{%
}
\end{mapleinput}

\mapleresult
\begin{maplelatex}
\[
{\it E2} := {\frac {\partial }{\partial x}}\,{\rm y}(x)={\rm y}(x
)^{2} - 1
\]
\end{maplelatex}

\end{maplegroup}
\begin{maplegroup}
\begin{mapleinput}
\mapleinline{active}{1d}{dsolve(E2,y(x));}{%
}
\end{mapleinput}

\mapleresult
\begin{maplelatex}
\[
{\rm arctanh}({\rm y}(x)) + x={\it \_C1}
\]
\end{maplelatex}

\end{maplegroup}

{\sc Maple} donne le r�sultat sous forme implicite.

\begin{maplegroup}
\begin{mapleinput}
\mapleinline{active}{1d}{solve(",y(x));}{%
}
\end{mapleinput}

\mapleresult
\begin{maplelatex}
\[
 - {\rm tanh}(x - {\it \_C1})
\]
\end{maplelatex}

\end{maplegroup}
\begin{maplegroup}
\begin{mapleinput}
\mapleinline{active}{1d}{dfieldplot(E2,y(x),x=-5..5,y=-3..3,colour=blue);}{%
}
\end{mapleinput}

\mapleresult
\begin{center}
\mapleplot{tp604.eps}
\end{center}

\end{maplegroup}
\begin{maplegroup}
\mapleresult
\begin{maplettyout}
\end{maplettyout}

\end{maplegroup}
\begin{maplegroup}
\begin{mapleinput}
\mapleinline{active}{1d}{DEplot(E2,y(x),x=-5..5,[[y(0)=0],[y(0)=1],[y(0)=-1],[y(0)=1.5]],y=-2..2);}{%
}
\end{mapleinput}

\mapleresult
\begin{center}
\mapleplot{tp605.eps}
\end{center}

\end{maplegroup}

On distingue deux solutions constantes~: la solution constante � $1$
et la solution constante � $-1$. Elles sont toutes les deux maximales
sur $\R$. Soit $y$ une autre solution maximale d�finie sur un
intervalle $I$. $y$ ne prend jamais les valeurs $1$ et $-1$ donc
l'�quation $(2)$ est �quivalente �
$\displaystyle\frac{y'}{y^2+1}=1$, soit $\displaystyle\fra

et est
continue donc d'apr�s la th�or�me des valeurs interm�diaires, on est
dans l'un des trois cas suivants~:
\begin{itemize}
\item $y(x)<-1\ \forall x\in I$~:

\item $|y(x)| < 1\ \forall x\in I$~:

\item $y(x)>1\ \forall x\in I$~:

\end{itemize}
\end{question}

\begin{maplegroup}
\begin{mapleinput}
\mapleinline{active}{1d}{E4:=diff(y(x),x)=abs(y(x))^(2/3);}{%
}
\end{mapleinput}

\mapleresult
\begin{maplelatex}
\[
{\it E4} := {\frac {\partial }{\partial x}}\,{\rm y}(x)= \left| 
 \! \,{\rm y}(x)\, \!  \right| ^{2/3}
\]
\end{maplelatex}

\end{maplegroup}
\begin{maplegroup}
\begin{mapleinput}
\mapleinline{active}{1d}{dsolve(E4,y(x));}{%
}
\end{mapleinput}

\mapleresult
\begin{maplelatex}
\[
 - 3\,{\displaystyle \frac {{\rm y}(x)}{ \left|  \! \,{\rm y}(x)
\, \!  \right| ^{2/3}}}  + x={\it \_C1}
\]
\end{maplelatex}

\end{maplegroup}
\begin{maplegroup}
\begin{mapleinput}
\mapleinline{active}{1d}{dfieldplot(E4,y(x),x=-5..5,y=-3..3,colour=blu
e);}{%
}
\end{mapleinput}

\mapleresult
\begin{center}
\mapleplot{tp606.eps}
\end{center}

\end{maplegroup}
\begin{maplegroup}
\mapleresult
\begin{maplettyout}
\end{maplettyout}

\end{maplegroup}
\begin{maplegroup}
\begin{mapleinput}
\mapleinline{active}{1d}{DEplot(E4,y(x),x=-5..5,[[y(0)=0],[y(0)=1],[y(
0)=-1]],y=-2..2);}{%
}
\end{mapleinput}

\mapleresult
\begin{center}
\mapleplot{tp607.eps}
\end{center}

\end{maplegroup}
\begin{maplegroup}
\begin{mapleinput}
\mapleinline{active}{1d}{Euler:=proc(phi,n,x0,y0,xn)
local Y,i,h,x,y;
h:=(xn-x0)/n;
x:=x0;
y:=y0;
Y:=[x,y];
for i from 1 to n do
  y:=y+h*phi(x,y);
  x:=x+h;
  Y:=Y,[x,y];
od;
[Y];
end:}{%
}
\end{mapleinput}

\end{maplegroup}
\begin{maplegroup}
\begin{mapleinput}
\mapleinline{active}{1d}{Euler((x,y)->x+y,6,0,1,3);}{%
}
\end{mapleinput}

\mapleresult
\begin{maplelatex}
\[
[[0, \,1], \,[{\displaystyle \frac {1}{2}} , \,{\displaystyle 
\frac {3}{2}} ], \,[1, \,{\displaystyle \frac {5}{2}} ], \,[
{\displaystyle \frac {3}{2}} , \,{\displaystyle \frac {17}{4}} ]
, \,[2, \,{\displaystyle \frac {57}{8}} ], \,[{\displaystyle 
\frac {5}{2}} , \,{\displaystyle \frac {187}{16}} ], \,[3, \,
{\displaystyle \frac {601}{32}} ]]
\]
\end{maplelatex}

\end{maplegroup}
\begin{maplegroup}
\begin{mapleinput}
\mapleinline{active}{1d}{dsolve(\{diff(y(x),x)=y(x)+x,y(0)=1\},y(x));}
{%
}
\end{mapleinput}

\mapleresult
\begin{maplelatex}
\[
{\rm y}(x)= - x - 1 + 2\,e^{x}
\]
\end{maplelatex}

\end{maplegroup}
\begin{maplegroup}
\begin{mapleinput}
\mapleinline{active}{1d}{e:=rhs(");}{%
}
\end{mapleinput}

\mapleresult
\begin{maplelatex}
\[
e :=  - x - 1 + 2\,e^{x}
\]
\end{maplelatex}

\end{maplegroup}
\begin{maplegroup}
\begin{mapleinput}
\mapleinline{active}{1d}{plots[display](plot(e,x=0..3,color=blue),
               plot(Euler((x,y)->x+y,50,0,1,3)));}{%
}
\end{mapleinput}

\mapleresult
\begin{center}
\mapleplot{tp608.eps}
\end{center}

\end{maplegroup}
\begin{maplegroup}
\begin{mapleinput}
\mapleinline{active}{1d}{dsolve(\{E1,y(0)=1\},y(x));}{%
}
\end{mapleinput}

\mapleresult
\begin{maplelatex}
\[
{\rm y}(x)= - {\displaystyle \frac {1}{x - 1}} 
\]
\end{maplelatex}

\end{maplegroup}
\begin{maplegroup}
\begin{mapleinput}
\mapleinline{active}{1d}{e:=rhs(");}{%
}
\end{mapleinput}

\mapleresult
\begin{maplelatex}
\[
e :=  - {\displaystyle \frac {1}{x - 1}} 
\]
\end{maplelatex}

\end{maplegroup}
\begin{maplegroup}
\begin{mapleinput}
\mapleinline{active}{1d}{plots[display](plot(e,x=0..2,y=-5..10,color=b
lue),
               plot(Euler((x,y)->y^2,10,0,1,2),x=0..2));}{%
}
\end{mapleinput}

\mapleresult
\begin{center}
\mapleplot{tp609.eps}
\end{center}

\end{maplegroup}
\begin{maplegroup}
\begin{mapleinput}
\end{mapleinput}

\end{maplegroup}

\end{document}


\message{ !name(TP6corrige.tex) !offset(-437) }
