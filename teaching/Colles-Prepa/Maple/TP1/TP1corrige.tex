\documentclass[12pt,a4paper]{article}
\usepackage{fancyheadings}
\usepackage{pstricks,pst-node,pst-tree}
\usepackage{amsmath}
\usepackage[ansinew]{inputenc}
\usepackage{maple2e}

\title{TP1 {\sc Maple} : Alg�bre lin�aire}
\author{}
\date{}

\setlength{\oddsidemargin}{0cm}
\addtolength{\textwidth}{70pt}
\setlength{\topmargin}{0cm}
\addtolength{\textheight}{2cm}
\setlength{\parindent}{0cm}

\pagestyle{fancy}
\lhead{Corrig� TP1 {\sc Maple}}
\rhead{Alg�bre lin�aire}


\DefineParaStyle{Heading 1}
\DefineParaStyle{Maple Output}
\DefineParaStyle{Warning}
\DefineCharStyle{2D Comment}
\DefineCharStyle{2D Math}
\DefineCharStyle{2D Output}

\newcounter{numquestion}
\setcounter{numquestion}{1}
\newenvironment{question}{\noindent{\bf Exercice \thenumquestion.}}%
{\stepcounter{numquestion}\medskip}


\begin{document}

{\bf\Large Exercice 1}

\begin{maplegroup}
\begin{mapleinput}
\mapleinline{active}{1d}{restart:}{}
\mapleinline{active}{1d}{with(linalg):}{}
\end{mapleinput}
\end{maplegroup}

\begin{maplegroup}
\begin{mapleinput}
\mapleinline{active}{1d}{M:=matrix([[0,1, 1, 1],
           [1,0,-1,-1],
           [1,-1,0,-1],
           [1,-1,-1,0]]);}{%
}
\end{mapleinput}

\mapleresult
\begin{maplelatex}
\[
M :=  \left[ 
{\begin{array}{rrrr}
0 & 1 & 1 & 1 \\
1 & 0 & -1 & -1 \\
1 & -1 & 0 & -1 \\
1 & -1 & -1 & 0
\end{array}}
 \right] 
\]
\end{maplelatex}

\end{maplegroup}
\begin{maplegroup}
\begin{mapleinput}
\mapleinline{active}{1d}{P:=factor(charpoly(M,X));}{%
}
\end{mapleinput}

\mapleresult
\begin{maplelatex}
\[
(X + 3)\,(X - 1)^{3}
\]
\end{maplelatex}

\end{maplegroup}
\begin{maplegroup}
\begin{mapleinput}
\mapleinline{active}{1d}{Id:=diag(1,1,1,1);}{%
}
\end{mapleinput}

\mapleresult
\begin{maplelatex}
\[
{\it Id} :=  \left[ 
{\begin{array}{rrrr}
1 & 0 & 0 & 0 \\
0 & 1 & 0 & 0 \\
0 & 0 & 1 & 0 \\
0 & 0 & 0 & 1
\end{array}}
 \right] 
\]
\end{maplelatex}

\end{maplegroup}
\begin{maplegroup}
\begin{mapleinput}
\mapleinline{active}{1d}{k1:=kernel(M-Id,'n');}{%
}
\end{mapleinput}

\mapleresult
\begin{maplelatex}
\[
{\it k1} := \{[1, \,0, \,1, \,0], \,[1, \,0, \,0, \,1], \,[1, \,1
, \,0, \,0]\}
\]
\end{maplelatex}

\end{maplegroup}
\begin{maplegroup}
Autre m�thode pour calculer la dimension du sous-espace propre:

\end{maplegroup}
\begin{maplegroup}
\begin{mapleinput}
\mapleinline{active}{1d}{4-rank(M-Id);}{%
}
\end{mapleinput}

\mapleresult
\begin{maplelatex}
\[
3
\]
\end{maplelatex}

\end{maplegroup}
\begin{maplegroup}
\begin{mapleinput}
\mapleinline{active}{1d}{k2:=kernel(M+3*Id,'m');}{%
}
\end{mapleinput}

\mapleresult
\begin{maplelatex}
\[
{\it k2} := \{[-1, \,1, \,1, \,1]\}
\]
\end{maplelatex}

\end{maplegroup}
\begin{maplegroup}
\begin{mapleinput}
\mapleinline{active}{1d}{n+m;}{%
}
\end{mapleinput}

\mapleresult
\begin{maplelatex}
\[
4
\]
\end{maplelatex}

\end{maplegroup}
\begin{maplegroup}
Donc M est diagonalisable : la somme des dimension de ses sous-espaces
propres est �gale � 4.

\end{maplegroup}
\begin{maplegroup}
\begin{mapleinput}
\mapleinline{active}{1d}{k:=k1 union k2;}{%
}
\end{mapleinput}

\mapleresult
\begin{maplelatex}
\[
k := \{[1, \,0, \,1, \,0], \,[1, \,0, \,0, \,1], \,[-1, \,1, \,1
, \,1], \,[1, \,1, \,0, \,0]\}
\]
\end{maplelatex}

\end{maplegroup}
\begin{maplegroup}
\begin{mapleinput}
\mapleinline{active}{1d}{P:=transpose(matrix([op(k)]));}{%
}
\end{mapleinput}

\mapleresult
\begin{maplelatex}
\[
P :=  \left[ 
{\begin{array}{rrrr}
-1 & 1 & 1 & 1 \\
1 & 0 & 0 & 1 \\
1 & 1 & 0 & 0 \\
1 & 0 & 1 & 0
\end{array}}
 \right] 
\]
\end{maplelatex}

\end{maplegroup}
\begin{maplegroup}
\begin{mapleinput}
\mapleinline{active}{1d}{evalm(P &* diag(-3,1,1,1) &* P^(-1));}{%
}
\end{mapleinput}

\mapleresult
\begin{maplelatex}
\[
 \left[ 
{\begin{array}{rrrr}
0 & 1 & 1 & 1 \\
1 & 0 & -1 & -1 \\
1 & -1 & 0 & -1 \\
1 & -1 & -1 & 0
\end{array}}
 \right] 
\]
\end{maplelatex}

\end{maplegroup}
\begin{maplegroup}
\begin{mapleinput}
\mapleinline{active}{1d}{K:=jordan(M,'Q');}{%
}
\end{mapleinput}

\mapleresult
\begin{maplelatex}
\[
K :=  \left[ 
{\begin{array}{rrrr}
1 & 0 & 0 & 0 \\
0 & -3 & 0 & 0 \\
0 & 0 & 1 & 0 \\
0 & 0 & 0 & 1
\end{array}}
 \right] 
\]
\end{maplelatex}

\end{maplegroup}
\begin{maplegroup}
\begin{mapleinput}
\mapleinline{active}{1d}{evalm(Q);}{%
}
\end{mapleinput}

\mapleresult
\begin{maplelatex}
\[
 \left[ 
{\begin{array}{ccrr}
{\displaystyle \frac {11}{4}}  & {\displaystyle \frac {1}{4}}  & 
2 & 1 \\ [2ex]
{\displaystyle \frac {1}{4}}  & {\displaystyle \frac {-1}{4}}  & 
0 & 0 \\ [2ex]
{\displaystyle \frac {5}{4}}  & {\displaystyle \frac {-1}{4}}  & 
1 & 0 \\ [2ex]
{\displaystyle \frac {5}{4}}  & {\displaystyle \frac {-1}{4}}  & 
1 & 1
\end{array}}
 \right] 
\]
\end{maplelatex}

\end{maplegroup}
\begin{maplegroup}
\begin{mapleinput}
\mapleinline{active}{1d}{evalm(Q &* K &* Q^(-1));}{%
}
\end{mapleinput}

\mapleresult
\begin{maplelatex}
\[
 \left[ 
{\begin{array}{rrrr}
0 & 1 & 1 & 1 \\
1 & 0 & -1 & -1 \\
1 & -1 & 0 & -1 \\
1 & -1 & -1 & 0
\end{array}}
 \right] 
\]
\end{maplelatex}

\end{maplegroup}

{\bf\Large Exercice 2}\\*

\begin{maplegroup}
\begin{mapleinput}
\mapleinline{active}{1d}{A:=matrix(3,3,[a-b-c,2*a,2*a,2*b,b-a-c,2*b,2*
c,2*c,c-a-b]);}{%
}
\end{mapleinput}

\mapleresult
\begin{maplelatex}
\[
A :=  \left[ 
{\begin{array}{ccc}
a - b - c & 2\,a & 2\,a \\
2\,b & b - a - c & 2\,b \\
2\,c & 2\,c & c - a - b
\end{array}}
 \right] 
\]
\end{maplelatex}

\end{maplegroup}
\begin{maplegroup}
\begin{mapleinput}
\mapleinline{active}{1d}{factor(det(A));}{%
}
\end{mapleinput}

\mapleresult
\begin{maplelatex}
\[
(a + b + c)^{3}
\]
\end{maplelatex}

\end{maplegroup}

Donc $A$ est diagonalisable si et seulement si $a+b+c\not= 0$.

\begin{maplegroup}
\begin{mapleinput}
\mapleinline{active}{1d}{factor(charpoly(A,X));}{%
}
\end{mapleinput}

\mapleresult
\begin{maplelatex}
\[
 - (a - X + c + b)\,(a + X + b + c)^{2}
\]
\end{maplelatex}

\end{maplegroup}
\begin{maplegroup}
\begin{mapleinput}
\mapleinline{active}{1d}{eigenvects(A);}{%
}
\end{mapleinput}

\mapleresult
\begin{maplelatex}
\[
[ - a - b - c, \,2, \,\{[-1, \,0, \,1], \,[-1, \,1, \,0]\}], \,[a
 + b + c, \,1, \,\{ \left[  \! {\displaystyle \frac {a}{b}} , \,1
, \,{\displaystyle \frac {c}{b}}  \!  \right] \}]
\]
\end{maplelatex}

\end{maplegroup}

Si $a+b+c\not=0$, $A$ admet deux valeurs propres distinctes~: $-(a+b+c)$
et $a+b+c$. Le sous-espace propre associ� � $-(a+b+c)$ est de
dimension $2$ et celui de $a+b+c$ de dimension $1$, donc $A$ est
diagonalisable. 

Si $a+b+c=0$, $A$ a une valeur propre de degr� $3$~: elle est
diagonalisable si et seulement si elle est diagonale. Or
$(a,b,c)\not=0$ donc $A$ n'est pas diagonalisable.

On peut donc en conclure que 
$$ A\text{ inversible }\Leftrightarrow\ A\text{ diagonalisable}$$

{\sc Maple} suppose toujours que les expressions avec des inconnus
sont non nulles~: 

\begin{maplegroup}
\begin{mapleinput}
\mapleinline{active}{1d}{Ad:=jordan(A,'P');}{%
}
\end{mapleinput}

\mapleresult
\begin{maplelatex}
\[
{\it Ad} :=  \left[ 
{\begin{array}{ccc}
 - a - b - c & 0 & 0 \\
0 & a + b + c & 0 \\
0 & 0 &  - a - b - c
\end{array}}
 \right] 
\]
\end{maplelatex}

\end{maplegroup}
\begin{maplegroup}
\begin{mapleinput}
\mapleinline{active}{1d}{evalm(P);}{%
}
\end{mapleinput}

\mapleresult
\begin{maplelatex}
\[
 \left[ 
{\begin{array}{ccr}
 - {\displaystyle \frac {a}{a + b + c}}  & {\displaystyle \frac {
a}{a + b + c}}  & -1 \\ [2ex]
 - {\displaystyle \frac {b}{a + b + c}}  & {\displaystyle \frac {
b}{a + b + c}}  & 0 \\ [2ex]
{\displaystyle \frac {a + b}{a + b + c}}  & {\displaystyle 
\frac {c}{a + b + c}}  & 1
\end{array}}
 \right] 
\]
\end{maplelatex}

\end{maplegroup}
\begin{maplegroup}
\begin{mapleinput}
\mapleinline{active}{1d}{map(simplify,evalm(P &* Ad &* P^(-1)));}{%
}
\end{mapleinput}

\mapleresult
\begin{maplelatex}
\[
 \left[ 
{\begin{array}{ccc}
a - b - c & 2\,a & 2\,a \\
2\,b & b - a - c & 2\,b \\
2\,c & 2\,c & c - a - b
\end{array}}
 \right] 
\]
\end{maplelatex}

\end{maplegroup}

{\bf\Large Exercice 3}

\begin{maplegroup}
\begin{mapleinput}
\mapleinline{active}{1d}{A:=matrix([[58, 52, 36], [-29, 187, 9],
[-145, 65, 219]]);}{%
}
\end{mapleinput}

\mapleresult
\begin{maplelatex}
\[
A :=  \left[ 
{\begin{array}{rrr}
58 & 52 & 36 \\
-29 & 187 & 9 \\
-145 & 65 & 219
\end{array}}
 \right] 
\]
\end{maplelatex}

\end{maplegroup}

{\bf M�thode 1 : en diagonalisant $A$}

\begin{maplegroup}
\begin{mapleinput}
\mapleinline{active}{1d}{eig:=eigenvals(A);}{%
}
\end{mapleinput}

\mapleresult
\begin{maplelatex}
\[
{\it eig} := 116, \,174, \,174
\]
\end{maplelatex}

$A$ a trois valeurs propres distinctes et est donc diagonalisable.

\end{maplegroup}
\begin{maplegroup}
\begin{mapleinput}
\mapleinline{active}{1d}{J:=jordan(A,'P');}{%
}
\end{mapleinput}

\mapleresult
\begin{maplelatex}
\[
J :=  \left[ 
{\begin{array}{rrr}
174 & 0 & 0 \\
0 & 116 & 0 \\
0 & 0 & 174
\end{array}}
 \right] 
\]
\end{maplelatex}

\end{maplegroup}

$A=P\cdot J\cdot P^{-1}$, donc $A^n=P\cdot J^n\cdot P^{-1}$.

\begin{maplegroup}

\begin{mapleinput}
\mapleinline{active}{1d}{puiss:=x->x^n;}{%
}
\end{mapleinput}

\mapleresult
\begin{maplelatex}
\[
{\it puiss} := x\rightarrow x^{n}
\]
\end{maplelatex}

\end{maplegroup}
\begin{maplegroup}
\begin{mapleinput}
\mapleinline{active}{1d}{Jn:=map(puiss,J);}{%
}
\end{mapleinput}

\mapleresult
\begin{maplelatex}
\[
{\it Jn} :=  \left[ 
{\begin{array}{ccc}
174^{n} & 0 & 0 \\
0 & 116^{n} & 0 \\
0 & 0 & 174^{n}
\end{array}}
 \right] 
\]
\end{maplelatex}

\end{maplegroup}
\begin{maplegroup}
\begin{mapleinput}
\mapleinline{active}{1d}{An1:=evalm(P &* Jn &* P^(-1));}{%
}
\end{mapleinput}

\mapleresult
\begin{maplelatex}
\[
{\it An1} :=  \left[ 
{\begin{array}{ccc}
 - 174^{n} + 2\,116^{n} & {\displaystyle \frac {26}{29}} \,174^{n
} - {\displaystyle \frac {26}{29}} \,116^{n} & {\displaystyle 
\frac {18}{29}} \,174^{n} - {\displaystyle \frac {18}{29}} \,116
^{n} \\ [2ex]
 - {\displaystyle \frac {1}{2}} \,174^{n} + {\displaystyle 
\frac {1}{2}} \,116^{n} & {\displaystyle \frac {71}{58}} \,174^{n
} - {\displaystyle \frac {13}{58}} \,116^{n} & {\displaystyle 
\frac {9}{58}} \,174^{n} - {\displaystyle \frac {9}{58}} \,116^{n
} \\ [2ex]
 - {\displaystyle \frac {5}{2}} \,174^{n} + {\displaystyle 
\frac {5}{2}} \,116^{n} & {\displaystyle \frac {65}{58}} \,174^{n
} - {\displaystyle \frac {65}{58}} \,116^{n} & {\displaystyle 
\frac {103}{58}} \,174^{n} - {\displaystyle \frac {45}{58}} \,116
^{n}
\end{array}}
 \right] 
\]
\end{maplelatex}

\end{maplegroup}

{\bf M�thode 2 : gr�ce au polyn�me minimal de $A$}

\begin{maplegroup}
\begin{mapleinput}
\mapleinline{active}{1d}{P:=minpoly(A,X);}{%
}
\end{mapleinput}

\mapleresult
\begin{maplelatex}
\[
P := 20184 - 290\,X + X^{2}
\]
\end{maplelatex}

\end{maplegroup}
\begin{maplegroup}
\begin{mapleinput}
\mapleinline{active}{1d}{Rem(X^n,P,X);}{%
}
\end{mapleinput}

\mapleresult
\begin{maplelatex}
\[
{\rm Rem}(X^{n}, \,20184 - 290\,X + X^{2}, \,X)
\]
\end{maplelatex}

\end{maplegroup}

{\sc Maple} ne sait pas r�pondre � cette question, donc il faut
l'aider...

\begin{maplegroup}
\begin{mapleinput}
\mapleinline{active}{1d}{eq:=X^n=Q[n](X)*P+a[n]*X+b[n];}{%
}
\end{mapleinput}

\mapleresult
\begin{maplelatex}
\[
{\it eq} := X^{n}={Q_{n}}(X)\,(20184 - 290\,X + X^{2}) + {a_{n}}
\,X + {b_{n}}
\]
\end{maplelatex}

\end{maplegroup}
\begin{maplegroup}
\begin{mapleinput}
\mapleinline{active}{1d}{l1:=subs(\{X=116\},eq);}{%
}
\end{mapleinput}

\mapleresult
\begin{maplelatex}
\[
{\it l1} := 116^{n}=116\,{a_{n}} + {b_{n}}
\]
\end{maplelatex}

\end{maplegroup}
\begin{maplegroup}
\begin{mapleinput}
\mapleinline{active}{1d}{l2:=subs(\{X=174\},eq);}{%
}
\end{mapleinput}

\mapleresult
\begin{maplelatex}
\[
{\it l2} := 174^{n}=174\,{a_{n}} + {b_{n}}
\]
\end{maplelatex}

\end{maplegroup}
\begin{maplegroup}
\begin{mapleinput}
\mapleinline{active}{1d}{solve(\{l1,l2\},\{a[n],b[n]\});}{%
}
\end{mapleinput}

\mapleresult
\begin{maplelatex}
\[
\{{b_{n}}=3\,116^{n} - 2\,174^{n}, \,{a_{n}}={\displaystyle 
\frac {1}{58}} \,174^{n} - {\displaystyle \frac {1}{58}} \,116^{n
}\}
\]
\end{maplelatex}

\end{maplegroup}
\begin{maplegroup}
\begin{mapleinput}
\mapleinline{active}{1d}{assign(");}{%
}
\end{mapleinput}

\end{maplegroup}
\begin{maplegroup}
\begin{mapleinput}
\mapleinline{active}{1d}{An2:=evalm(a[n]*A+b[n]);}{%
}
\end{mapleinput}

\mapleresult
\begin{maplelatex}
\[
{\it An2} :=  \left[ 
{\begin{array}{ccc}
 - 174^{n} + 2\,116^{n} & {\displaystyle \frac {26}{29}} \,174^{n
} - {\displaystyle \frac {26}{29}} \,116^{n} & {\displaystyle 
\frac {18}{29}} \,174^{n} - {\displaystyle \frac {18}{29}} \,116
^{n} \\ [2ex]
 - {\displaystyle \frac {1}{2}} \,174^{n} + {\displaystyle 
\frac {1}{2}} \,116^{n} & {\displaystyle \frac {71}{58}} \,174^{n
} - {\displaystyle \frac {13}{58}} \,116^{n} & {\displaystyle 
\frac {9}{58}} \,174^{n} - {\displaystyle \frac {9}{58}} \,116^{n
} \\ [2ex]
 - {\displaystyle \frac {5}{2}} \,174^{n} + {\displaystyle 
\frac {5}{2}} \,116^{n} & {\displaystyle \frac {65}{58}} \,174^{n
} - {\displaystyle \frac {65}{58}} \,116^{n} & {\displaystyle 
\frac {103}{58}} \,174^{n} - {\displaystyle \frac {45}{58}} \,116
^{n}
\end{array}}
 \right] 
\]
\end{maplelatex}

\end{maplegroup}
\begin{maplegroup}
\begin{mapleinput}
\mapleinline{active}{1d}{iszero(An-An2);}{%
}
\end{mapleinput}

\mapleresult
\begin{maplelatex}
\[
{\it true}
\]
\end{maplelatex}
\end{maplegroup}

{\bf\Large Exercice 4}

\begin{maplegroup}
\begin{mapleinput}
\mapleinline{active}{1d}{A:=matrix([[a,0,0,0,b],
           [0,a,0,b,0],
           [0,1,2,1,0],
           [0,b,0,a,0],
           [b,0,0,0,a]]);}{%
}
\end{mapleinput}

\mapleresult
\begin{maplelatex}
\[
A :=  \left[ 
{\begin{array}{ccrcc}
a & 0 & 0 & 0 & b \\
0 & a & 0 & b & 0 \\
0 & 1 & 2 & 1 & 0 \\
0 & b & 0 & a & 0 \\
b & 0 & 0 & 0 & a
\end{array}}
 \right] 
\]
\end{maplelatex}

\end{maplegroup}
\begin{maplegroup}
\begin{mapleinput}
\mapleinline{active}{1d}{P:=factor(charpoly(A,X));}{%
}
\end{mapleinput}

\mapleresult
\begin{maplelatex}
\[
P := (X - 2)\,( - b + a - X)^{2}\,(b + a - X)^{2}
\]
\end{maplelatex}

\end{maplegroup}
\begin{maplegroup}
\begin{mapleinput}
\mapleinline{active}{1d}{eig:=eigenvects(A);}{%
}
\end{mapleinput}

\mapleresult
\begin{maplelatex}
\begin{eqnarray*}
\lefteqn{{\it eig} := [ - b + a, \,2, \,\{[-1, \,0, \,0, \,0, \,1
], \,[0, \,-1, \,0, \,1, \,0]\}], \,[2, \,1, \,\{[0, \,0, \,1, \,
0, \,0]\}], } \\
 & & [b + a, \,2, \,\{ \left[  \! 0, \,{\displaystyle \frac {1}{2
}} \,b + {\displaystyle \frac {1}{2}} \,a - 1, \,1, \,
{\displaystyle \frac {1}{2}} \,b + {\displaystyle \frac {1}{2}} 
\,a - 1, \,0 \!  \right] , \,[1, \,0, \,0, \,0, \,1]\}]
\mbox{\hspace{57pt}}
\end{eqnarray*}
\end{maplelatex}

\end{maplegroup}
\begin{maplegroup}
\begin{mapleinput}
\mapleinline{active}{1d}{vp:=eig[3];}{%
}
\end{mapleinput}

\mapleresult
\begin{maplelatex}
\[
{\it vp} := [b + a, \,2, \,\{ \left[  \! 0, \,{\displaystyle 
\frac {1}{2}} \,b + {\displaystyle \frac {1}{2}} \,a - 1, \,1, \,
{\displaystyle \frac {1}{2}} \,b + {\displaystyle \frac {1}{2}} 
\,a - 1, \,0 \!  \right] , \,[1, \,0, \,0, \,0, \,1]\}]
\]
\end{maplelatex}

\end{maplegroup}
\begin{maplegroup}
\begin{mapleinput}
\mapleinline{active}{1d}{vects:=vp[3];}{%
}
\end{mapleinput}

\mapleresult
\begin{maplelatex}
\[
{\it vects} := \{ \left[  \! 0, \,{\displaystyle \frac {1}{2}} \,
b + {\displaystyle \frac {1}{2}} \,a - 1, \,1, \,{\displaystyle 
\frac {1}{2}} \,b + {\displaystyle \frac {1}{2}} \,a - 1, \,0 \! 
 \right] , \,[1, \,0, \,0, \,0, \,1]\}
\]
\end{maplelatex}

\end{maplegroup}
\begin{maplegroup}
\begin{mapleinput}
\mapleinline{active}{1d}{v:=op(2,vects);}{%
}
\end{mapleinput}

\mapleresult
\begin{maplelatex}
\[
v :=  \left[  \! 0, \,{\displaystyle \frac {1}{2}} \,b + 
{\displaystyle \frac {1}{2}} \,a - 1, \,1, \,{\displaystyle 
\frac {1}{2}} \,b + {\displaystyle \frac {1}{2}} \,a - 1, \,0 \! 
 \right] 
\]
\end{maplelatex}

\end{maplegroup}
\begin{maplegroup}
\begin{mapleinput}
\mapleinline{active}{1d}{eq:=v[2];}{%
}
\end{mapleinput}

\mapleresult
\begin{maplelatex}
\[
{\it eq} := {\displaystyle \frac {1}{2}} \,b + {\displaystyle 
\frac {1}{2}} \,a - 1
\]
\end{maplelatex}

\end{maplegroup}
\begin{maplegroup}
\begin{mapleinput}
\mapleinline{active}{1d}{sol:=solve(eq);}{%
}
\end{mapleinput}

\mapleresult
\begin{maplelatex}
\[
{\it sol} := \{a= - b + 2, \,b=b\}
\]
\end{maplelatex}

\end{maplegroup}

On en d�duit que si $a+b\not=2$, A est diagonalisable.

Si $a+b=2$~:

\begin{maplegroup}
\begin{mapleinput}
\mapleinline{active}{1d}{assign(sol);}{%
}
\end{mapleinput}

\end{maplegroup}
\begin{maplegroup}
\begin{mapleinput}
\mapleinline{active}{1d}{AA:=map(eval,A);}{%
}
\end{mapleinput}

\mapleresult
\begin{maplelatex}
\[
{\it AA} :=  \left[ 
{\begin{array}{ccrcc}
 - b + 2 & 0 & 0 & 0 & b \\
0 &  - b + 2 & 0 & b & 0 \\
0 & 1 & 2 & 1 & 0 \\
0 & b & 0 &  - b + 2 & 0 \\
b & 0 & 0 & 0 &  - b + 2
\end{array}}
 \right] 
\]
\end{maplelatex}

\end{maplegroup}
\begin{maplegroup}
\begin{mapleinput}
\mapleinline{active}{1d}{eigenvects(AA);}{%
}
\end{mapleinput}

\mapleresult
\begin{maplelatex}
\[
[ - 2\,b + 2, \,2, \,\{[0, \,-1, \,0, \,1, \,0], \,[-1, \,0, \,0
, \,0, \,1]\}], \,[2, \,3, \,\{[1, \,0, \,0, \,0, \,1], \,[0, \,0
, \,1, \,0, \,0]\}]
\]
\end{maplelatex}

\end{maplegroup}

\textbf{Conclusion} :
$2$ est une valeur propre d'ordre $3$, mais l'espace propre associ� est de
dimension seulement $2$, $A$ n'est donc pas diagonalisable.

\newpage


{\bf\Large Exercice 5}

\begin{maplegroup}
\begin{mapleinput}
\mapleinline{active}{1d}{A:=matrix(3,3,[1,0,0,0,0,lambda,0,2,3]);}{%
}
\end{mapleinput}

\mapleresult
\begin{maplelatex}
\[
A :=  \left[ 
{\begin{array}{rrc}
1 & 0 & 0 \\
0 & 0 & \lambda  \\
0 & 2 & 3
\end{array}}
 \right] 
\]
\end{maplelatex}

\end{maplegroup}
\begin{maplegroup}
\begin{mapleinput}
\mapleinline{active}{1d}{X:=matrix(3,3,[a,b,c,d,e,f,g,h,i]);}{%
}
\end{mapleinput}

\mapleresult
\begin{maplelatex}
\[
X :=  \left[ 
{\begin{array}{ccc}
a & b & c \\
d & e & f \\
g & h & i
\end{array}}
 \right] 
\]
\end{maplelatex}

\end{maplegroup}
\begin{maplegroup}
\begin{mapleinput}
\mapleinline{active}{1d}{evalm(X&*A - A&*X);}{%
}
\end{mapleinput}

\mapleresult
\begin{maplelatex}
\[
 \left[ 
{\begin{array}{ccc}
0 & 2\,c - b & b\,\lambda  + 2\,c \\
d - \lambda \,g & 2\,f - h\,\lambda  & e\,\lambda  + 3\,f - 
\lambda \,i \\
 - 2\,g - 2\,d & 2\,i - 2\,e - 3\,h & h\,\lambda  - 2\,f
\end{array}}
 \right] 
\]
\end{maplelatex}

\end{maplegroup}
\begin{maplegroup}
\begin{mapleinput}
\mapleinline{active}{1d}{convert(",set);}{%
}
\end{mapleinput}

\mapleresult
\begin{maplelatex}
\begin{eqnarray*}
\lefteqn{\{2\,i - 2\,e - 3\,h, \,h\,\lambda  - 2\,f, \,0, \,e\,
\lambda  + 3\,f - \lambda \,i, \,d - \lambda \,g, \,2\,f - h\,
\lambda , \,b\,\lambda  + 2\,c, \,2\,c - b, \, - 2\,g - 2\,d} \\
 & & \}\mbox{\hspace{425pt}}
\end{eqnarray*}
\end{maplelatex}

\end{maplegroup}
\begin{maplegroup}
\begin{mapleinput}
\mapleinline{active}{1d}{sol:=solve(");}{%
}
\end{mapleinput}

\mapleresult
\begin{maplelatex}
\begin{eqnarray*}
\lefteqn{{\it sol} := \{e=i - {\displaystyle \frac {3}{2}} \,h, 
\,i=i, \,\lambda =\lambda , \,h=h, \,b=0, \,f={\displaystyle 
\frac {1}{2}} \,h\,\lambda , \,d=0, \,c=0, \,g=0\}, } \\
 & & \{c=c, \,g=g, \,e=i - {\displaystyle \frac {3}{2}} \,h, \,i=
i, \,h=h, \,b=2\,c, \,f= - {\displaystyle \frac {1}{2}} \,h, \,d=
 - g, \,\lambda =-1\}\mbox{\hspace{18pt}}
\end{eqnarray*}
\end{maplelatex}

\end{maplegroup}
\begin{maplegroup}
\begin{mapleinput}
\mapleinline{active}{1d}{assign(sol[1]);}{%
}
\end{mapleinput}

\end{maplegroup}
\begin{maplegroup}
\begin{mapleinput}
\mapleinline{active}{1d}{map(eval,X);}{%
}
\end{mapleinput}

\mapleresult
\begin{maplelatex}
\[
 \left[ 
{\begin{array}{ccc}
a & 0 & 0 \\
0 & i - {\displaystyle \frac {3}{2}} \,h & {\displaystyle \frac {
1}{2}} \,h\,\lambda  \\ [2ex]
0 & h & i
\end{array}}
 \right] 
\]
\end{maplelatex}

\end{maplegroup}

{\bf Conclusion} :
Si $\lambda \not= -1$, l'alg�bre des commutants de A est de dimension 3.

\begin{maplegroup}
\begin{mapleinput}
\mapleinline{active}{1d}{unassign('a','b','c','d','e','f','g','h','i',
'lambda');}{%
}
\end{mapleinput}

\end{maplegroup}
\begin{maplegroup}
\begin{mapleinput}
\mapleinline{active}{1d}{assign(sol[2]);}{%
}
\end{mapleinput}

\end{maplegroup}
\begin{maplegroup}
\begin{mapleinput}
\mapleinline{active}{1d}{map(eval,X);}{%
}
\end{mapleinput}

\mapleresult
\begin{maplelatex}
\[
 \left[ 
{\begin{array}{ccc}
a & 2\,c & c \\
 - g & i - {\displaystyle \frac {3}{2}} \,h &  - {\displaystyle 
\frac {1}{2}} \,h \\ [2ex]
g & h & i
\end{array}}
 \right] 
\]
\end{maplelatex}

\end{maplegroup}

{\bf Conclusion} :
Si $\lambda = - 1$, l'alg�bre des commutants de $A$ est de dimension $5$.

{\bf\Large Exercice 6}

\begin{maplegroup}
\begin{mapleinput}
\mapleinline{active}{1d}{A1:=matrix([[a1,b1,c1,d1],
            [0,e1,f1,g1],
            [0,0,h1,i1],
            [0,0,0,j1]]);}{%
}
\end{mapleinput}

\mapleresult
\begin{maplelatex}
\[
{\it A1} :=  \left[ 
{\begin{array}{cccc}
{\it a1} & {\it b1} & {\it c1} & {\it d1} \\
0 & {\it e1} & {\it f1} & {\it g1} \\
0 & 0 & {\it h1} & {\it i1} \\
0 & 0 & 0 & {\it j1}
\end{array}}
 \right] 
\]
\end{maplelatex}

\end{maplegroup}
\begin{maplegroup}
\begin{mapleinput}
\mapleinline{active}{1d}{A2:=matrix([[a2,b2,c2,d2],
            [0,e2,f2,g2],
            [0,0,h2,i2],
            [0,0,0,j2]]);}{%
}
\end{mapleinput}

\mapleresult
\begin{maplelatex}
\[
{\it A2} :=  \left[ 
{\begin{array}{cccc}
{\it a2} & {\it b2} & {\it c2} & {\it d2} \\
0 & {\it e2} & {\it f2} & {\it g2} \\
0 & 0 & {\it h2} & {\it i2} \\
0 & 0 & 0 & {\it j2}
\end{array}}
 \right] 
\]
\end{maplelatex}

\end{maplegroup}
\begin{maplegroup}
\begin{mapleinput}
\mapleinline{active}{1d}{evalm(A1 &* A2);}{%
}
\end{mapleinput}

\mapleresult
\begin{maplelatex}
\[
 \left[ 
{\begin{array}{cccc}
{\it a1}\,{\it a2} & {\it a1}\,{\it b2} + {\it b1}\,{\it e2} & 
{\it a1}\,{\it c2} + {\it b1}\,{\it f2} + {\it c1}\,{\it h2} & 
{\it a1}\,{\it d2} + {\it b1}\,{\it g2} + {\it c1}\,{\it i2} + 
{\it d1}\,{\it j2} \\
0 & {\it e1}\,{\it e2} & {\it e1}\,{\it f2} + {\it f1}\,{\it h2}
 & {\it e1}\,{\it g2} + {\it f1}\,{\it i2} + {\it g1}\,{\it j2}
 \\
0 & 0 & {\it h1}\,{\it h2} & {\it h1}\,{\it i2} + {\it i1}\,{\it 
j2} \\
0 & 0 & 0 & {\it j1}\,{\it j2}
\end{array}}
 \right] 
\]
\end{maplelatex}

\end{maplegroup}
\begin{maplegroup}
\begin{mapleinput}
\mapleinline{active}{1d}{evalm(A1^(-1));}{%
}
\end{mapleinput}

\mapleresult
\begin{maplelatex}
\[
 \left[ 
{\begin{array}{cccc}
{\displaystyle \frac {1}{{\it a1}}}  &  - {\displaystyle \frac {
{\it b1}}{{\it a1}\,{\it e1}}}  &  - {\displaystyle \frac { - 
{\it b1}\,{\it f1} + {\it e1}\,{\it c1}}{{\it a1}\,{\it e1}\,
{\it h1}}}  & {\displaystyle \frac { - {\it b1}\,{\it f1}\,{\it 
i1} + {\it b1}\,{\it g1}\,{\it h1} + {\it e1}\,{\it c1}\,{\it i1}
 - {\it e1}\,{\it d1}\,{\it h1}}{{\it a1}\,{\it e1}\,{\it h1}\,
{\it j1}}}  \\ [2ex]
0 & {\displaystyle \frac {1}{{\it e1}}}  &  - {\displaystyle 
\frac {{\it f1}}{{\it e1}\,{\it h1}}}  & {\displaystyle \frac {
{\it f1}\,{\it i1} - {\it g1}\,{\it h1}}{{\it e1}\,{\it h1}\,
{\it j1}}}  \\ [2ex]
0 & 0 & {\displaystyle \frac {1}{{\it h1}}}  &  - {\displaystyle 
\frac {{\it i1}}{{\it h1}\,{\it j1}}}  \\ [2ex]
0 & 0 & 0 & {\displaystyle \frac {1}{{\it j1}}} 
\end{array}}
 \right] 
\]
\end{maplelatex}

\end{maplegroup}
\begin{maplegroup}
\begin{mapleinput}
\mapleinline{active}{1d}{A:=matrix([[1,2,3,4],
            [0,1,2,3],
            [0,0,1,2],
            [0,0,0,1]]);}{%
}
\end{mapleinput}

\mapleresult
\begin{maplelatex}
\[
A :=  \left[ 
{\begin{array}{rrrr}
1 & 2 & 3 & 4 \\
0 & 1 & 2 & 3 \\
0 & 0 & 1 & 2 \\
0 & 0 & 0 & 1
\end{array}}
 \right] 
\]
\end{maplelatex}

\end{maplegroup}
\begin{maplegroup}
\begin{mapleinput}
\mapleinline{active}{1d}{B:=matrix([[1,1,0,0],
            [0,1,1,0],
            [0,0,1,1],
            [0,0,0,1]]);}{%
}
\end{mapleinput}

\mapleresult
\begin{maplelatex}
\[
B :=  \left[ 
{\begin{array}{rrrr}
1 & 1 & 0 & 0 \\
0 & 1 & 1 & 0 \\
0 & 0 & 1 & 1 \\
0 & 0 & 0 & 1
\end{array}}
 \right] 
\]
\end{maplelatex}

\end{maplegroup}

On cherche $A_1\in\mathcal{T}(\mathbf{R},4)$, inversible, tel que 
$A_1\cdot A = B\cdot A_1$.

\begin{maplegroup}
\begin{mapleinput}
\mapleinline{active}{1d}{convert(evalm(A1 &* A - B &* A1),set);}{%
}
\end{mapleinput}

\mapleresult
\begin{maplelatex}
\begin{eqnarray*}
\lefteqn{\{2\,{\it a1} - {\it e1}, \,3\,{\it a1} + 2\,{\it b1} - 
{\it f1}, \,4\,{\it a1} + 3\,{\it b1} + 2\,{\it c1} - {\it g1}, 
\,2\,{\it e1} - {\it h1}, \,3\,{\it e1} + 2\,{\it f1} - {\it i1}
, \,0, } \\
 & & 2\,{\it h1} - {\it j1}\}\mbox{\hspace{347pt}}
\end{eqnarray*}
\end{maplelatex}

\end{maplegroup}
\begin{maplegroup}
\begin{mapleinput}
\mapleinline{active}{1d}{solve(");}{%
}
\end{mapleinput}

\mapleresult
\begin{maplelatex}
\begin{eqnarray*}
\lefteqn{\{{\it j1}=8\,{\it a1}, \,{\it e1}=2\,{\it a1}, \,{\it 
a1}={\it a1}, \,{\it c1}={\it c1}, \,{\it b1}={\it b1}, \,{\it f1
}=3\,{\it a1} + 2\,{\it b1}, \,{\it h1}=4\,{\it a1}, } \\
 & & {\it i1}=12\,{\it a1} + 4\,{\it b1}, \,{\it g1}=4\,{\it a1}
 + 3\,{\it b1} + 2\,{\it c1}\}\mbox{\hspace{156pt}}
\end{eqnarray*}
\end{maplelatex}

\end{maplegroup}
\begin{maplegroup}
\begin{mapleinput}
\mapleinline{active}{1d}{assign(");}{%
}
\end{mapleinput}

\end{maplegroup}
\begin{maplegroup}
\begin{mapleinput}
\mapleinline{active}{1d}{map(eval,A1);}{%
}
\end{mapleinput}

\mapleresult
\begin{maplelatex}
\[
 \left[ 
{\begin{array}{cccc}
{\it a1} & {\it b1} & {\it c1} & {\it d1} \\
0 & 2\,{\it a1} & 3\,{\it a1} + 2\,{\it b1} & 4\,{\it a1} + 3\,
{\it b1} + 2\,{\it c1} \\
0 & 0 & 4\,{\it a1} & 12\,{\it a1} + 4\,{\it b1} \\
0 & 0 & 0 & 8\,{\it a1}
\end{array}}
 \right] 
\]
\end{maplelatex}

\end{maplegroup}

On a donc plusieurs solutions, dont celle-ci~:
\begin{maplegroup}
\begin{mapleinput}
\mapleinline{active}{1d}{P:=subs(\{a1=1,b1=0,c1=0,d1=0\},");}{%
}
\end{mapleinput}

\mapleresult
\begin{maplelatex}
\[
P :=  \left[ 
{\begin{array}{rrrr}
1 & 0 & 0 & 0 \\
0 & 2 & 3 & 4 \\
0 & 0 & 4 & 12 \\
0 & 0 & 0 & 8
\end{array}}
 \right] 
\]
\end{maplelatex}

\end{maplegroup}
\begin{maplegroup}
\begin{mapleinput}
\mapleinline{active}{1d}{iszero(evalm(P &* A * P^(-1) - B));}{%
}
\end{mapleinput}

\mapleresult
\begin{maplelatex}
\[
{\it true}
\]
\end{maplelatex}

\end{maplegroup}

{\bf\Large Exercice 7}

\begin{maplegroup}
\begin{mapleinput}
\mapleinline{active}{1d}{A:=matrix(3,3,[a,1,b,1,c,d,e,f,-1]);}{%
}
\end{mapleinput}

\mapleresult
\begin{maplelatex}
\[
A :=  \left[ 
{\begin{array}{ccc}
a & 1 & b \\
1 & c & d \\
e & f & -1
\end{array}}
 \right] 
\]
\end{maplelatex}

\end{maplegroup}
\begin{maplegroup}
\begin{mapleinput}
\mapleinline{active}{1d}{X:=vector([1,1,0]);}{%
}
\end{mapleinput}

\mapleresult
\begin{maplelatex}
\[
X := [1, \,1, \,0]
\]
\end{maplelatex}

\end{maplegroup}
\begin{maplegroup}
\begin{mapleinput}
\mapleinline{active}{1d}{Y:=vector([1,2,1]);}{%
}
\end{mapleinput}

\mapleresult
\begin{maplelatex}
\[
Y := [1, \,2, \,1]
\]
\end{maplelatex}

\end{maplegroup}
\begin{maplegroup}
\begin{mapleinput}
\mapleinline{active}{1d}{Z:=vector([1,1,2]);}{%
}
\end{mapleinput}

\mapleresult
\begin{maplelatex}
\[
Z := [1, \,1, \,2]
\]
\end{maplelatex}

\end{maplegroup}
\begin{maplegroup}
\begin{mapleinput}
\mapleinline{active}{1d}{evalm(A &* X - lambda[X] * X);}{%
}
\end{mapleinput}

\mapleresult
\begin{maplelatex}
\[
 \left[  \! a + 1 - {\lambda _{X}}, \,1 + c - {\lambda _{X}}, \,e
 + f \!  \right] 
\]
\end{maplelatex}

\end{maplegroup}
\begin{maplegroup}
\begin{mapleinput}
\mapleinline{active}{1d}{S:=convert(",set);}{%
}
\end{mapleinput}

\mapleresult
\begin{maplelatex}
\[
S := \{a + 1 - {\lambda _{X}}, \,1 + c - {\lambda _{X}}, \,e + f
\}
\]
\end{maplelatex}

\end{maplegroup}
\begin{maplegroup}
\begin{mapleinput}
\mapleinline{active}{1d}{evalm(A &* Y - lambda[Y] * Y);}{%
}
\end{mapleinput}

\mapleresult
\begin{maplelatex}
\[
 \left[  \! a + 2 + b - {\lambda _{Y}}, \,1 + 2\,c + d - 2\,{
\lambda _{Y}}, \,e + 2\,f - 1 - {\lambda _{Y}} \!  \right] 
\]
\end{maplelatex}

\end{maplegroup}
\begin{maplegroup}
\begin{mapleinput}
\mapleinline{active}{1d}{S:=S union convert(",set);}{%
}
\end{mapleinput}

\mapleresult
\begin{maplelatex}
\[
S := \{a + 2 + b - {\lambda _{Y}}, \,1 + 2\,c + d - 2\,{\lambda 
_{Y}}, \,e + 2\,f - 1 - {\lambda _{Y}}, \,a + 1 - {\lambda _{X}}
, \,1 + c - {\lambda _{X}}, \,e + f\}
\]
\end{maplelatex}

\end{maplegroup}
\begin{maplegroup}
\begin{mapleinput}
\mapleinline{active}{1d}{evalm(A &* Z - lambda[Z] * Z);}{%
}
\end{mapleinput}

\mapleresult
\begin{maplelatex}
\[
 \left[  \! a + 1 + 2\,b - {\lambda _{Z}}, \,1 + c + 2\,d - {
\lambda _{Z}}, \,e + f - 2 - 2\,{\lambda _{Z}} \!  \right] 
\]
\end{maplelatex}

\end{maplegroup}
\begin{maplegroup}
\begin{mapleinput}
\mapleinline{active}{1d}{S:=S union convert(",set);}{%
}
\end{mapleinput}

\mapleresult
\begin{maplelatex}
\begin{eqnarray*}
\lefteqn{S := \{a + 1 + 2\,b - {\lambda _{Z}}, \,1 + c + 2\,d - {
\lambda _{Z}}, \,e + f - 2 - 2\,{\lambda _{Z}}, \,a + 2 + b - {
\lambda _{Y}}, \,1 + 2\,c + d - 2\,{\lambda _{Y}}, } \\
 & & e + 2\,f - 1 - {\lambda _{Y}}, \,a + 1 - {\lambda _{X}}, \,1
 + c - {\lambda _{X}}, \,e + f\}\mbox{\hspace{214pt}}
\end{eqnarray*}
\end{maplelatex}

\end{maplegroup}
\begin{maplegroup}
\begin{mapleinput}
\mapleinline{active}{1d}{solve(S);}{%
}
\end{mapleinput}

\mapleresult
\begin{maplelatex}
\[
\{d=-3, \,{\lambda _{X}}=5, \,{\lambda _{Y}}=3, \,e=-4, \,b=-3, 
\,c=4, \,a=4, \,{\lambda _{Z}}=-1, \,f=4\}
\]
\end{maplelatex}

\end{maplegroup}
\begin{maplegroup}
\begin{mapleinput}
\mapleinline{active}{1d}{assign(");}{%
}
\end{mapleinput}

\end{maplegroup}
\begin{maplegroup}
\begin{mapleinput}
\mapleinline{active}{1d}{map(eval,A);}{%
}
\end{mapleinput}

\mapleresult
\begin{maplelatex}
\[
 \left[ 
{\begin{array}{rrr}
4 & 1 & -3 \\
1 & 4 & -3 \\
-4 & 4 & -1
\end{array}}
 \right] 
\]
\end{maplelatex}

\end{maplegroup}
\begin{maplegroup}
\begin{mapleinput}
\mapleinline{active}{1d}{charpoly(",x);}{%
}
\end{mapleinput}

\mapleresult
\begin{maplelatex}
\[
x^{3} - 7\,x^{2} + 7\,x + 15
\]
\end{maplelatex}

\end{maplegroup}
\begin{maplegroup}
\begin{mapleinput}
\mapleinline{active}{1d}{factor(");}{%
}
\end{mapleinput}

\mapleresult
\begin{maplelatex}
\[
(x - 5)\,(x - 3)\,(x + 1)
\]
\end{maplelatex}

\end{maplegroup}

{\bf\Large Exercice 8}

\begin{maplegroup}
\begin{mapleinput}
\mapleinline{active}{1d}{A:=matrix(3,3,[x,0,0,0,y,sqrt(3),0,sqrt(3),z]
);}{%
}
\end{mapleinput}

\mapleresult
\begin{maplelatex}
\[
A :=  \left[ 
{\begin{array}{ccc}
x & 0 & 0 \\
0 & y & \sqrt{3} \\
0 & \sqrt{3} & z
\end{array}}
 \right] 
\]
\end{maplelatex}

\end{maplegroup}
\begin{maplegroup}
\begin{mapleinput}
\mapleinline{active}{1d}{P:= X -> charpoly(A,X);}{%
}
\end{mapleinput}

\mapleresult
\begin{maplelatex}
\[
P := X\rightarrow {\rm charpoly}(A, \,X)
\]
\end{maplelatex}

\end{maplegroup}
\begin{maplegroup}
\begin{mapleinput}
\mapleinline{active}{1d}{solve(\{P(4),P(8),P(12)\});}{%
}
\end{mapleinput}

\mapleresult
\begin{maplelatex}
\begin{eqnarray*}
\lefteqn{\{x=8, \,z={\rm RootOf}(51 - 16\,{\it \_Z} + {\it \_Z}^{
2}), \,y=16 - {\rm RootOf}(51 - 16\,{\it \_Z} + {\it \_Z}^{2})\}
, } \\
 & & \{x=12, \,z=5, \,y=7\}, \,\{x=12, \,z=7, \,y=5\}, \,\{z=9, 
\,x=4, \,y=11\}, \,\{x=4, \,y=9, \,z=11\}
\end{eqnarray*}
\end{maplelatex}

\end{maplegroup}

Il y a donc $4$ triplets solutions.

\vspace{.3cm}
{\bf\Large Exercice 9}

\begin{maplegroup}
\begin{mapleinput}
\mapleinline{active}{1d}{A:=matrix(4,4,[0,1,0,0,0,0,1,0,0,0,0,1,0,0,0,
0]);}{%
}
\end{mapleinput}

\mapleresult
\begin{maplelatex}
\[
A :=  \left[ 
{\begin{array}{rrrr}
0 & 1 & 0 & 0 \\
0 & 0 & 1 & 0 \\
0 & 0 & 0 & 1 \\
0 & 0 & 0 & 0
\end{array}}
 \right] 
\]
\end{maplelatex}

\end{maplegroup}
\begin{maplegroup}
\begin{mapleinput}
\mapleinline{active}{1d}{est_nil:=proc(A)}{}
\end{mapleinput}
\vspace{-.5cm}
\begin{verbatim}
       local n,i,R;
       n:=coldim(A);
       R:=A;
       for i from 1 to n-1 while not iszero(R) do
         R:= evalm(R &* A);
       od;
       if i<n-1 then true
       else iszero(R)
       fi;
      end:
\end{verbatim}


\end{maplegroup}
\begin{maplegroup}
\begin{mapleinput}
\mapleinline{active}{1d}{est_nil(A);}{%
}
\end{mapleinput}

\mapleresult
\begin{maplelatex}
\[
{\it true}
\]
\end{maplelatex}

\end{maplegroup}

\end{document}

