\documentclass[12pt,a4paper]{article}
\usepackage{fancyheadings}
\usepackage{pstricks,pst-node,pst-tree}
\usepackage{amsmath}
\usepackage[ansinew]{inputenc}

\title{TP1 {\sc Maple} : Alg�bre lin�aire}
\author{}
\date{}


\setlength{\oddsidemargin}{0cm}
\addtolength{\textwidth}{70pt}
\setlength{\topmargin}{0cm}
\addtolength{\textheight}{2cm}
\setlength{\parindent}{0cm}


\pagestyle{fancy}
\lhead{TP1 {\sc Maple}}
\rhead{Alg�bre lin�aire}


\newcounter{numquestion}
\setcounter{numquestion}{1}
\newenvironment{question}{\noindent{\bf Exercice \thenumquestion.}}%
{\stepcounter{numquestion}\medskip}


\begin{document}

\maketitle

\begin{question}
  Diagonaliser la matrice
$$M =  \left( 
\begin{array}{cccc}
0 & 1 & 1  & 1 \\
1 & 0 & -1 & -1 \\
1 & -1 & 0 & -1 \\
1 & -1 & -1 & 0 
\end{array} \right)$$ 
Dans un premier temps on s'interdira l'usage des commandes {\tt
  jordan}, {\tt eigenvals} et {\tt eigenvects}. Donner la matrice de
  passage et v�rifier votre r�sultat.
\end{question}

\begin{question}
Soit
$$A =  \left( 
\begin{array}{ccc}
a - b - c & 2a & 2a \\
2b & b - a - c & 2b \\
2c & 2c & c - a - b
\end{array} \right)$$ 
On suppose que $(a,b,c)\not=0$. Montrer que $A$ est diagonalisable si
et seulement si $A$ est inversible.
\end{question}

\begin{question}
On pose 
$$A =  \left( 
\begin{array}{ccc}
58 & 52 & 36 \\
-29 & 187 & 9 \\
-145 & 65 & 219
\end{array} \right)$$ 
Calculer $A^n$ pour $n\in\mathbf{N}$ de deux mani�res diff�rentes.
\end{question}

\begin{question}
Trouver une condition n�cessaire et suffisante pour que la matrice
suivante soit diagonalisable.   
$$A =  \left( 
\begin{array}{ccccc}
a & 0 & 0 & 0 & b \\
0 & a & 0 & b & 0 \\
0 & 1 & 2 & 1 & 0 \\
0 & b & 0 & a & 0 \\
b & 0 & 0 & 0 & a
\end{array} \right)$$ 
\end{question}

\begin{question}
Calculer l'ensemble des matrices qui commutent avec la matrice
suivante.
$$A =  \left( 
\begin{array}{ccc}
1 & 0 & 0 \\
0 & 0 & \lambda \\
0 & 2 & 3
\end{array} \right)$$ 
Quel est la dimension de l'espace vectoriel obtenu ?
\end{question}

\begin{question}
Montrer que l'ensemble $\mathcal{T}(\mathbf{R},4)$ des matrices
triangulaires sup�rieures d'ordre $4$ est stable par multiplication et
inverse.   

On pose
$$A =  \left( 
\begin{array}{ccccc}
1 & 2 & 3 & 4  \\
0 & 1 & 2 & 3  \\
0 & 0 & 1 & 2  \\
0 & 0 & 0 & 1  
\end{array} \right)\text{ et }
B =  \left( 
\begin{array}{ccccc}
1 & 1 & 0 & 0  \\
0 & 1 & 1 & 0  \\
0 & 0 & 1 & 1  \\
0 & 0 & 0 & 1  
\end{array}\right)
$$ 
Ces deux matrices sont-elles semblables ?
\end{question}

\begin{question}
D�terminer $a,b,c,d,e$ et $f$ pour que les vecteurs
$$\left(\begin{array}{c}
   1 \\ 1 \\ 0 \end{array}\right),
\left(\begin{array}{c}
   1 \\ 2 \\ 1 \end{array}\right),
\left(\begin{array}{c}
   1 \\ 1 \\ 2 \end{array}\right)
$$ 
  forment une base de vecteurs propres de la matrice
$$A = \left( 
\begin{array}{ccc}
a & 1 & b \\
1 & c & d \\
e & f & -1 
\end{array} \right)$$ 
\end{question}

\begin{question}
Trouver tous les triplets $(x,y,z)\in\mathbf{R}^3$ tels que la matrice
$$A = \left( 
\begin{array}{ccc}
x & 0 & 0 \\
0 & y & \sqrt{3} \\
0 & \sqrt{3} & z 
\end{array} \right)$$ 
admette $4,8$ et $12$ comme valeurs propres.
\end{question}

\begin{question}
  Ecrire une proc�dure qui teste si une matrice est nilpotente.
\end{question}

\end{document}

